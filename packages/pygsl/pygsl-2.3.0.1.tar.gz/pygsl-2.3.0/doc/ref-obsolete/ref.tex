% Complete documentation on the extended LaTeX markup used for Python
% documentation is available in ``Documenting Python'', which is part
% of the standard documentation for Python.  It may be found online
% at:
%
%     http://www.python.org/doc/current/doc/doc.html

\documentclass[hyperref]{manual}

% latex2html doesn't know [T1]{fontenc}, so we cannot use that:(
\usepackage{amsmath}
\usepackage[latin1]{inputenc}
\usepackage{textcomp}
\usepackage{hyperref}

% this version does not reset module names at section level
%begin{latexonly}
\makeatletter
\let\py@OldOldChapter=\chapter
\renewcommand{\chapter}{\py@reset%
                        \py@OldOldChapter}
\renewcommand{\section}{\@startsection{section}{1}{\z@}%
   {-3.5ex \@plus -1ex \@minus -.2ex}%
   {2.3ex \@plus.2ex}%
   {\reset@font\Large\py@HeaderFamily}}
\makeatother
%end{latexonly}


% some convenience declarations
\newcommand{\gsl}{GSL}
\newcommand{\GSL}{GNU Scientific Library}
\newcommand{\numpy}{NumPy}
\newcommand{\NUMPY}{Numerical Python}
\newcommand{\pygsl}{PyGSL}
\newcommand{\PYGSL}{PyGSL: Python wrapper of the GNU Scientific Library}

\makeatletter
\newenvironment{pytypedesc}[2]{
  % Using \renewcommand doesn't work for this, for unknown reasons:
  \global\def\py@thisclass{#1}
  \begin{fulllineitems}
    \py@sigline{\strong{pytype }\bfcode{#1}}{#2}%
    \index{#1@{\py@idxcode{#1}} (pytype in \py@thismodule)}
}{\end{fulllineitems}}
\makeatother


\title{PyGSL Reference Manual}

\ifhtml
\author{
  \ulink{Achim G\"adke}{mailto:achimgaedke@users.sourceforge.net}\\
  Technische Universit\"at Darmstadt, Darmstadt, Germany
}
\author{
  \ulink{Pierre Schnizer}{mailto:schnizer@users.sourceforge.net}\\
  Gesellschaft f\"ur Schwerionenforschung, Darmstadt, Germany
}
%\author{
%  \ulink{Jochen K\"upper}{mailto:jochen@jochen-kuepper.de}\\
%  Fritz-Haber-Institut der MPG, Berlin, Germany
%}
%\author{
%  \ulink{S\'ebastien Maret}{mailto:schnizer@users.sourceforge.net}\\
%  Department of Astronomy, University of Michigan, Ann Arbor, USA
%}
\else
%begin{latexonly}
%% pdfelatex (TeXLive 7) doesn't handle \footnotemark in here...
\author{Achim G\"adke \\ 
          Jochen K\"upper \\ 
         %S\'ebastien Maret \\
        Pierre Schnizer}
% Please at least include a long-lived email address!
\authoraddress{
   Technische Universit\"at Darmstadt, Darmstadt, Germany \\
   \email{achimgaedke@users.sourceforge.net} \\
   Gesellschaft f\"ur Schwerionenforschung, Darmstadt, Germany \\
   \email{schnizer@users.sourceforge.net} \\
%   Fritz-Haber-Institut der MPG, Berlin, Germany \\
%   \email{jochen@jochen-kuepper.de} \\
%   Department of Astronomy, University of Michigan, Ann Arbor, USA \\
%   \email{bmaret@users.sourceforge.et} \\
}
%end{latexonly}
\fi

\date{October, 2008}            % update before release!
                                % Use an explicit date so that reformatting
                                % doesn't cause a new date to be used.  Setting
                                % the date to \today can be used during draft
                                % stages to make it easier to handle versions.
\release{0.9}                   % release version; this is used to define the
\setshortversion{0.9}           % \version macro
\makeindex                      % tell \index to actually write the .idx file


\begin{document}

\maketitle

% This makes the contents more accessible from the front page of the HTML.
\ifhtml
\chapter*{Front Matter}
\label{front}
\fi

Copyright \copyright{} 2002,2005 The pygsl Team.

Permission is granted to copy, distribute and/or modify this document under the
terms of the GNU Free Documentation License, Version 1.1 or any later version
published by the Free Software Foundation; with no Invariant Sections, no
Front-Cover Texts, and no Back-Cover Texts.  A copy of the license is included
in section \ref{cha:free-documentation-license} entitled ``GNU Free
Documentation License''.


%% Local Variables:
%% mode: LaTeX
%% mode: auto-fill
%% fill-column: 79
%% indent-tabs-mode: nil
%% ispell-dictionary: "american"
%% reftex-fref-is-default: nil
%% TeX-auto-save: t
%% TeX-command-default: "pdfeLaTeX"
%% TeX-master: "pygsl"
%% TeX-parse-self: t
%% End:


\begin{abstract}
   \noindent
   pygsl grants python users access to the GNU scientific library.  The latest
   version can be found at the project homepage, \url{http://pygsl.sf.net}.

   \textbf{Implemented features:} \\
   \begin{tabular}{ll}
     \module{pygsl.blas}                & basic linear algebra system\\
     \module{pygsl.chebyshev}           & chebyshev approximations\\
     \module{pygsl.combination}         & combinations  \\
     \module{pygsl.const}               & $>200$ often used mathematical and
                                          scientific constants. \\
     \module{pygsl.diff}                & (Deprecated. Use pygsl.deriv instead). \\
     \module{pygsl.deriv}               & Numerical differentiation. \\
     \module{pygsl.eigen}               &\\
     \module{pygsl.fit}                 &\\
     \module{pygsl.histogram}          & 1d and 2d histograms and operations
                                          on histograms. \\
     \module{pygsl.ieee}                & Access to the ieee-arithmetics layer
                                          of gsl. \\ 
     \module{pygsl.integrate}           &\\
     \module{pygsl.interpolation}       &\\ 
     \module{pygsl.linalg}              &\\
     \module{pygsl.math}                &\\
     \module{pygsl.monte}               &\\
     \module{pygsl.minimize}            &\\
     \module{pygsl.multifit}            &\\
     \module{pygsl.multifit_nlin}       &\\    
     \module{pygsl.multimin}            &\\
     \module{pygsl.multiroots}          &\\ 
     \module{pygsl.odeiv}               &\\
     \module{pygsl.permutation}         &\\  
     \module{pygsl.poly}                &\\
     \module{pygsl.qrng}                &\\     
     \module{pygsl.rng}                 & random number generators and probability densities. \\
     \module{pygsl.roots}               &\\
     \module{pygsl.siman}               &Simulated anealing\\
     \module{pygsl.sum}                 & \\
     \module{pygsl.sf}                  & $>200$ special functions. \\
     \module{pygsl.statistics}          & Statistical functions. \\
   \end{tabular}
\end{abstract}


\tableofcontents


\chapter{System Requirements, Installation}
\label{cha:system-req-installation}
\section{Status}

\paragraph*{Status of GSL-Library}
The gsl-library is since version 1.0 stable and for general use.
More information about it at \url{http://www.gnu.org/software/gsl/}.

\paragraph*{Status of this interface}
Nearly all modules are wrapped. A lot of tests are
covering various functionality. Please report to the mailing list
\url{pygsl-discuss@lists.sourceforge.net} if you find a bug.

The hankel modules have been
wrapped. Please write to the mailing list
\url{pygsl-discuss@lists.sourceforge.net} 
if you require one of the modules
and are willing to help with a simple example. 
If any other function is missing or some other module (e.g. ntuple) or
function, do not hesitate to write to the list.

\paragraph*{Retriving the Interface}
You can download it here: \url{http://sourceforge.net/projects/pygsl}

\section{Requirements}

To build the interface, you will need
\begin{itemize}
\item \ulink{gsl-1.x}{http://sources.redhat.com/gsl},
\item \ulink{python2.6}{http://www.python.org} or better,
\item \ulink{NumPy}{http://numpy.sf.net}, and
\item a c compiler (like \ulink{gcc}{http://gcc.gnu.org}).
\end{itemize}

Supported Platforms are:
\begin{itemize}
\item Linux (Redhat/Debian/SuSE) with python2.* and gsl-1.*
\item Win32
\end{itemize}
It was tested and is tested on an irregular basis on the following platforms
\begin{itemize}
\item SUN
\item Cygwin
\item MacOS X
\end{itemize}
but is supposed to build on any POSIX platforms.

\section{Installing the pygsl interface}

\program{gsl-config} must be on your path:\nopagebreak
\begin{verbatim}
# unpack the source distribution
gzip -d -c pygsl-x.y.z.tar.gz|tar xvf-
cd pygsl-x.y.z
# do this with your prefered python version
# to set the gsl location explicitly use setup.py --gsl-prefix=/path/to/gsl
python setup.py build
# change to an user id, that is allowed to do installation
python setup.py install
\end{verbatim}
Ready....

{\bf Do not test the interface in the distribution root or in the directories
 \file{src} or \file{pygsl}.}

If you find unresolved symbols later on, delete the C source in the
swig_src files. Check that swig can be called from the command line. 
Then start the build process again. 

In this case swig will rebuild the C files. The swig_src files
distributed with pygsl are to an up to date version of GSL (1.16 as of
this writing). Swig parses partly some header header files and builds
the appropriate interface functions. If you have an older GSL version 
locally installed, the sources in the swig_src directory can contain 
links to symbols which are not defined by the locally installed GSL
version.

\subsection{Building on win32}

Windows by default does not allow to run a posix shell. Here a different path
is required. First change into the directory \file{gsl_dist}. Copy the file 
\file{gsl_site_example.py}
and edit it to reflect your installation of GSL and SWIG if you want to run it
yourself. The pygsl windows binaries distributed over 
\url{http://sourceforge.net/projects/pygsl/} are built using the mingw32 
compiler. 

\paragraph*{Uninstall GSL interface}
\code{rm -r }"python install path"\code{/lib/python}"version"\code{/site-packages/pygsl}

\paragraph*{Testing}
the directory \file{tests} contains several testsuites, based on python
\module{unittest}.
The script \file{run_test.py} in this directory will run one after the other.

\paragraph*{Support}
Please send mails to our mailinglist at
\email{pygsl-discuss@lists.sourceforge.net}.

\paragraph*{Developement}
You can browse our cvs tree at
\url{http://cvs.sourceforge.net/cgi-bin/viewcvs.cgi/pygsl/pygsl/}.
\\
Type this to check out the actual version:
\begin{verbatim}
cvs -d:pserver:anonymous@cvs.pygsl.sourceforge.net:/cvsroot/pygsl login
#Hit return for no password.
cvs -z3 -d:pserver:anonymous@cvs.pygsl.sourceforge.net:/cvsroot/pygsl co pygsl
\end{verbatim}
The script \program{tools/extract_tool.py} generates most of the special 
function code.

%\input{install_advanced.tex}
\paragraph*{ToDo}
Implement other parts:


\paragraph*{History}
\begin{itemize}
\item a gsl-interface for python was needed for a project at
\ulink{Center for Applied Informatics Cologne}{http://www.zaik.uni-koeln.de/AFS}.
\item \file{gsl-0.0.3} was released at May 23, 2001
\item \file{gsl-0.0.4} was released at January 8, 2002
\item \file{gsl-0.0.5} is growing since January, 2002
\item \file{gsl-0.2.0} was released at 
\item \file{gsl-0.3.0} was released at 
\item \file{gsl-0.3.1} was released at 
\item \file{gsl-0.3.2} was released at 
\item \file{gsl-0.9.4} was released at 25. October 2008
\end{itemize}

\paragraph*{Thanks}
Jochen K\"upper (\email{jochen@jochen-kuepper.de}) for 
\module{pygsl.statistics} part\\
Fabian Jakobs for \module{pygsl.blas}, \module{pygsl.eigen}
\module{pygsl.linalg}, \module{pygsl.permutation}\\ 
Leonardo Milano for rpm build\\
Eric Gurrola and  Peter Stoltz for testing and supporting the port of pygsl to
the MAC\\
Sebastien Maret for supporting the Fink \url{http://fink.sourceforge.net}
port of pygsl.


\paragraph*{Maintainers}
Achim G\"adke (\email{AchimGaedke@users.sourceforge.net}),\\
Pierre Schnizer (\email{schnizer@users.sourceforge.net})

\input{installadvanced.tex}
\chapter{Design of the \pygsl{} interface}

The GSL library was implemented using the C language. This implies that 
each function uses a certain type for the different variables and are fixed
 to one specific type. The wrapper will try to convert each argument to the approbriate
C type. 
The \pygsl{} interface
tries to follow it as much as possible but only as far as resonable. 
For example the definition of the poly_eval function in C is given by
\begin{funcdesc}{\texttt{double} gsl_poly_eval}
                {\texttt{const double} C[], \texttt{const int} LEN, \texttt{const double} X}
\end{funcdesc}

The corresponding python wrapper was implemented by
\begin{funcdesc}{poly.poly_eval}{C, x}
\end{funcdesc}
as the wrapper can get the length of any python object and then fill the len variable. 
The mathematical calculation is performed by the GSL library. Thus the calculation is limited 
to the precision provided by the underlying hardware.

Default arguments are used to allocate workspaces on the fly if not provided by the user. 
Consider for example the fft module. The function for the real forward transform is
named 

\begin{funcdesc}{\texttt{int} gsl_fft_real_transform}
{\texttt{double DATA[]}, 
 \texttt{size_t STRIDE},
 \texttt{size_t N}, 
 \texttt{const gsl_fft_real_wavetable * WAVETABLE},
 \texttt{gsl_fft_real_workspace * WORK}
}
\end{funcdesc}

The corresponding python wrapper is found in the fft module called
real_transform
\begin{funcdesc}{real_transform}
{data, \optional{space, 
    table, 
    output}}
\end{funcdesc}
The wrapper will get the stride and size information from the data object provided
by the user. If space or table are not provided, these objects will be generated on 
the fly. As the GSL function applies the transformation in space, an internal copy is 
made of the data and only then the object is passed to the \gsl{} function. If an output
object is provided the data will be copied there instead. \pygsl{} will always make copies
of objects which would be otherwise modified in place.

\section{Callbacks}

Solvers require as one argument a user function to work on which have to be provided by the
user. These callbacks typically are of the form
\begin{funcdesc}{f}{x, params}
  \dots\\
  return result
\end{funcdesc}
Please note that this function must return the exact number of arguments
as given in the example. The wrappers around callbacks go a long way to try to provide
meaningfull error messages. If a solver fails, please check that the number of input and 
output arguments it takes are correct

\section{Error handling}
\label{sec:interface-error-handling}
As GSL is a C library error handling is implemented using an error handler and return values.
\pygsl{} generates python exceptions out of these values. See \module{pygsl.errors} 
(chapter~\ref{cha:error-module}) for a list of the exceptions.

\section{Exception handling}
\index{exception handling!initialisation} GSL provides a selectable error
handler, that is called for occuring errors (like domain errors, division by
zero, etc. ).  This is switched off. Instead each wrapper function will check
the error return value and in case of error an python exception is created. 

Here is a python level example:
\begin{verbatim}
import pygsl.histogram
import pygsl.errors
hist=pygsl.histogram.histogram2d(100,100)
try:
   hist[-1,-1]=0
except pygsl.errors.gsl_Error,err:
   print err
\end{verbatim}
Will result
\begin{verbatim}
input domain error: index i lies outside valid range of 0 .. nx - 1
\end{verbatim}


An exception are ufuncs in the testings.sf module (see section\ref{sec:ufuncs}).


\subsection{Change of internal error handling.}
Before a error handler was installed by init_pygsl into gsl which translated
the error code (and the message) to a python exception.
This required that the GIL was available, which numpy ufuncs dispose. Thus
now this gsl error handler is deactivated and instead the C error code
returned by the C function is translated to an error code by the wrapper
called from python.

UFuncs do not call this handler now at all.


\section{The documentation gap}

\pygsl{} does still lack an approbriate documentation. Most documentation is accessible over
the internal documentation strings. These are accessible as \code{__doc__} attributes (the help
function does not always show them).  It can be sometimes necessary to create an 
object to see its methods as well as the documentation of the methods
 (e.g.a random number generator in the rng module to see its methods). 
The \file{example} directory contains examples for (nearly each) module.

Please feel welcome to add to the documentation!


\paragraph*{Acknowledgment}
\label{sec:acknowledgment}
Parts of this this manual are based on the \GSL{} reference manual.
The authors want to thank all for contribution of code,
support material for generating distribution packages, bug reports
and example scripts.


\chapter[\protect\module{pygsl.errors} --- Error and warning classes]
{\protect\module{pygsl.errors} \\ Error and warning classes} 
\label{cha:error-module}
\declaremodule{standard}{pygsl.errors}
\moduleauthor{Pierre Schnizer}{schnizer@users.sourceforge.net}
\moduleauthor{Original Author: Achim G\"adke}{achimgaedke@users.sourceforge.net}

This chapter provides information about the \exception{gsl_Error} exception class that comes with this module.

\section{Exception Classes}


\begin{excclassdesc} {gsl_Error}{}
derived from \exception{Exception}, can be constructed with any object as parameter.
It is baseclass to all other \gsl{} Exceptions
\end{excclassdesc}
These classes are translations of the \file{<gsl/gsl_errno.h>} to python
exceptions.


\begin{excclassdesc}{gsl_ArithmeticError}{}
derived from \exception{gsl_Error} and \exception{exceptions.ArithmeticError},
base of all common arithmetic exceptions
\end{excclassdesc}

\begin{excclassdesc}{gsl_OverflowError}{}
derived from \exception{gsl_Error} and \exception{exceptions.OverflowError}
\end{excclassdesc}

\begin{excclassdesc}{gsl_ZeroDivisionError}{}
derived from \exception{gsl_Error} and \exception{exceptions.ZeroDivisionError}
\end{excclassdesc}

\begin{excclassdesc}{gsl_FloatingPointError}{}
derived from \exception{gsl_Error} and \exception{exceptions.FloatingPointError}
\end{excclassdesc}

\begin{excclassdesc}{gsl_ArithmeticError}{}
is derived from  \exception{gsl_Error} and from  \exception{ArithmeticError} .
This exception is the    base of all common arithmetic exceptions.
\end{excclassdesc}

\begin{excclassdesc}{gsl_AccuracyLossError}{}
is derived from  \exception{gsl_ArithmeticError} .
This exception is raised if the failed to reach the specified tolerance.
\end{excclassdesc}
\begin{excclassdesc}{gsl_BadFuncError}{}
is derived from  \exception{gsl_Error} .
This exception is raised if problem with a user-supplied function occur.
\end{excclassdesc}
\begin{excclassdesc}{gsl_BadLength}{}
is derived from  \exception{gsl_Error} .
This exception is raised if  matrix or  vector lengths are not conformant.
\end{excclassdesc}
\begin{excclassdesc}{gsl_BadToleranceError}{}
is derived from  \exception{gsl_Error} .
This exception is raised if user specified an tolerance which can not be reached.
\end{excclassdesc}
\begin{excclassdesc}{gsl_CacheLimitError}{}
is derived from  \exception{gsl_Error} .
This exception is raised if the    cache limit is exceeded.
\end{excclassdesc}
\begin{excclassdesc}{gsl_DivergeError}{}
is derived from  \exception{gsl_ArithmeticError} .
This exception is raised if an   integral or series is divergent.
\end{excclassdesc}
\begin{excclassdesc}{gsl_DomainError}{}
is derived from  \exception{gsl_Error} .
This exception is raised if    domain errors occure. e.g. sqrt(-1).
\end{excclassdesc}
\begin{excclassdesc}{gsl_EOFError}{}
is derived from  \exception{gsl_Error} and from  \exception{EOFError} .
This exception is raised if 
    end of file
     .
\end{excclassdesc}
\begin{excclassdesc}{gsl_FactorizationError}{}
is derived from  \exception{gsl_Error} .
This exception is raised if     factorization failed.
\end{excclassdesc}
\begin{excclassdesc}{gsl_FloatingPointError}{}
is derived from  \exception{gsl_Error} and from  \exception{FloatingPointError} .
\end{excclassdesc}
\begin{excclassdesc}{gsl_GenericError}{}
is derived from  \exception{gsl_Error} .
\end{excclassdesc}
\begin{excclassdesc}{gsl_InvalidArgumentError}{}
is derived from  \exception{gsl_Error} .
This exception is raised if an invalid argument is supplied by the user.
\end{excclassdesc}
\begin{excclassdesc}{gsl_JacobianEvaluationError}{}
is derived from  \exception{gsl_ArithmeticError} .
This exception is raised if jacobian evaluations are not improving the solution.
\end{excclassdesc}
\begin{excclassdesc}{gsl_MatrixNotSquare}{}
is derived from  \exception{gsl_Error} .
This exception is raised if the given matrix is not square.
\end{excclassdesc}
\begin{excclassdesc}{gsl_MaximumIterationError}{}
is derived from  \exception{gsl_ArithmeticError} .
This exception is raised if    the maximum number  of iterations is exceeded.
\end{excclassdesc}
\begin{excclassdesc}{gsl_NoHardwareSupportError}{}
is derived from  \exception{gsl_Error} .
This exception is raised if the requested feature is not supported by the hardware.
\end{excclassdesc}
\begin{excclassdesc}{gsl_NoProgressError}{}
is derived from  \exception{gsl_ArithmeticError} .
This exception is raised if the  iteration is not making progress towards solution.
\end{excclassdesc}
\begin{excclassdesc}{gsl_NotImplementedError}{}
is derived from  \exception{gsl_Error} and from  \exception{NotImplementedError} .
This exception is raised if  a requested feature is not (yet) implemented .
\end{excclassdesc}
\begin{excclassdesc}{gsl_OverflowError}{}
is derived from  \exception{gsl_Error} and from  \exception{OverflowError} .
\end{excclassdesc}
\begin{excclassdesc}{gsl_PointerError}{}
is derived from  \exception{gsl_Error} .
This exception is raised if an invalid pointer is found by the C wrapper code
or by the GSL library.
\end{excclassdesc}
\begin{excclassdesc}{gsl_RangeError}{}
is derived from  \exception{gsl_ArithmeticError} .
This exception is raised if     output would be out or range, e.g. exp(1e100)
     .
\end{excclassdesc}
\begin{excclassdesc}{gsl_RoundOffError}{}
is derived from  \exception{gsl_ArithmeticError} .
This exception is raised if  arithmetic failed because of roundoff error.
\end{excclassdesc}
\begin{excclassdesc}{gsl_RunAwayError}{}
is derived from  \exception{gsl_ArithmeticError} .
This exception is raised if   iterative process is out of control.
\end{excclassdesc}
\begin{excclassdesc}{gsl_SanityCheckError}{}
is derived from  \exception{gsl_Error} .
This exception is raised if a sanity check failed - shouldn't happen.
\end{excclassdesc}
\begin{excclassdesc}{gsl_SingularityError}{}
is derived from  \exception{gsl_ArithmeticError} .
This exception is raised if  an   apparent singularity is detected.
\end{excclassdesc}
\begin{excclassdesc}{gsl_TableLimitError}{}
is derived from  \exception{gsl_Error} .
This exception is raised if the table limit is exceeded.
\end{excclassdesc}
\begin{excclassdesc}{gsl_ToleranceError}{}
is derived from  \exception{gsl_ArithmeticError} .
This exception is raised if  the alghorithm failed to reach the specified tolerance.
\end{excclassdesc}
\begin{excclassdesc}{gsl_ToleranceFError}{}
is derived from  \exception{gsl_ArithmeticError} .
This exception is raised if  the alghorithm cannot reach the specified
tolerance in F (typically the variation of the evaluated function).
\end{excclassdesc}
\begin{excclassdesc}{gsl_ToleranceGradientError}{}
is derived from  \exception{gsl_ArithmeticError} .
This exception is raised if  cannot reach the specified tolerance for the gradient.
\end{excclassdesc}
\begin{excclassdesc}{gsl_ToleranceXError}{}
is derived from  \exception{gsl_ArithmeticError} .
This exception is raised if cannot reach the specified tolerance in X
(typically a search result).
\end{excclassdesc}
\begin{excclassdesc}{gsl_UnderflowError}{}
is derived from  \exception{gsl_Error} and from  \exception{OverflowError} .
\end{excclassdesc}
\begin{excclassdesc}{gsl_ZeroDivisionError}{}
is derived from  \exception{gsl_Error} and from  \exception{ZeroDivisionError} .
\end{excclassdesc}

All the above errors are just translations of the errno to python exceptions.

The following two are specific to pygsl:
\begin{excclassdesc}{pygsl.errors.pygsl_NotImplementedError}{}
is derived from  \exception{gsl_Error} and from  \exception{NotImplementedError} .
This exception is raised if a feature is requested but not
implemented. Currently only used if a module requests the debugging enviroment
of the init module, but the init module was not compiled with \code{\#define DEBUG=1}
\end{excclassdesc}
\begin{excclassdesc}{pygsl.errors.pygsl_StrideError}{}
is derived from  \exception{gsl_SanityCheckError} .
GSL uses as strides multiples of the basis type; for a vector or doubles, one
means from one double to the next. Numpy or numarray count the stride in
multiples of the size of a char. Therefore the stride has to be recalculated
before the approbriate \gsl{} function can be called. If that fails this
exception is raised.
\end{excclassdesc}

\section{Warning Classes}

\begin{excclassdesc} {gsl_Warning}{}
The dedicated warning class for \gsl{} has \exception{Warning} as base class.
\end{excclassdesc}

\begin{excclassdesc}{gsl_DomainWarning}{}
derived from \exception{gsl_Warning}, used by some \module{pygsl.histogram} functions
\end{excclassdesc}


\chapter[\protect\module{pygsl.const} --- Mathematical and scientific
constants]{\protect\module{pygsl.const} \\ Mathematical and scientific
constants} 
\label{cha:const-module}
\declaremodule{extension}{pygsl.const} 
\moduleauthor{Achim  G\"adke}{achimgaedke@users.sourceforge.net}

In this module some usefull constants are defined.  There are four groups of
constants:

\begin{itemize}
\item mathematical,
\item physical in mks unit system,
\item physical in cgs unit system and
\item physical number constants (e.g. fine structure)
\end{itemize}

The other modules are created during the initialisation of
\module{pygsl.const}.  For convenience the mathematical, physical mks
constants and number constants also are available in the namespace of
\module{pygsl.const}.  If the used GSL version is before gsl1.4, see
\begin{verbatim}
pygsl.compiled_gsl_version
\end{verbatim}
than the module names are cgs and mks. Form gsl1.5 these modules have been
renamed to cgsm and mksa. So to use cgs constants one has to write
\begin{verbatim}
import pygsl.const
import pygsl.const.cgs
print pygsl.const.cgs.speed_of_light/pygsl.const.speed_of_light
\end{verbatim}
for gsl $<$ 1.5 and
\begin{verbatim}
import pygsl.const
import pygsl.const.cgsm
print pygsl.const.cgsm.speed_of_light/pygsl.const.speed_of_light
\end{verbatim}.
Of course the result is \constant{100.0}.
Short examples are given at top of each section.

\begin{seealso}
  The actual values are taken form the \gsl{} headers.  The \GSL{} reference
  provides a more detailed description of these constants.
\end{seealso}

\section[\protect\module{pygsl.const.m} --- Mathematical constants]
{\protect\module{pygsl.const.m} \\ Mathematical constants} 
\label{cha:const-math-module}

\begin{verbatim}
from pygsl.const.m import *
print sqrt2*sqrt2
\end{verbatim}\\
Prints \constant{2.0}.\\
 Here comes the list:\nopagebreak
\begin{longtableiii}{l|l|l}{constant}{Name}{\gsl{} Name}{value}
\lineiii{e}{\protect\constant{M\_E}}{e}
\lineiii{log2e}{\constant{M\_LOG2E}}{$\log_2 e$}
\lineiii{log10e}{\constant{M\_LOG10E}}{$\log_{10} e$}
\lineiii{sqrt2}{\constant{M\_SQRT2}}{$\sqrt{2}$}
\lineiii{sqrt1\_2}{\constant{M\_SQRT1\_2}}{$\sqrt{1/2}$}
\lineiii{sqrt3}{\constant{M\_SQRT3}}{$\sqrt{3}$}
\lineiii{pi}{\constant{M\_PI}}{$\pi$}
\lineiii{pi\_2}{\constant{M\_PI\_2}}{$\pi/2$}
\lineiii{pi\_4}{\constant{M\_PI\_4}}{$\pi/4$}
\lineiii{sqrtpi}{\constant{M\_SQRTPI}}{$\sqrt{\pi}$}
\lineiii{2\_sqrtpi}{\constant{M\_2\_SQRTPI}}{$2/\sqrt{\pi}$}
\lineiii{1\_pi}{\constant{M\_1\_PI}}{$1/\pi$}
\lineiii{2\_pi}{\constant{M\_2\_PI}}{$2/\pi$}
\lineiii{ln10}{\constant{M\_LN10}}{$\ln 10$}
\lineiii{ln2}{\constant{M\_LN2}}{$\ln 2$}
\lineiii{lnpi}{\constant{M\_LNPI}}{$\ln{\pi}$}
\lineiii{euler}{\constant{M\_EULER}}{Euler constant}
\end{longtableiii}

\section[\protect\module{pygsl.const.mksa} --- Scientific constants in mksa units]
{\protect\module{pygsl.const.mksa} \\ Scientific constants in mksa units} 
\label{cha:const-mks-module}

\begin{verbatim}
from pygsl.const import cgsm
print "a teaspoon contains %g m^3"%mks.teaspoon
\end{verbatim}

These are the provided constants:\nopagebreak
\begin{longtableiii}{l|l|l}{constant}{Name}{gsl Name}{unit}
\lineiii{speed\_of\_light}{\constant{GSL\_CONST\_MKSA\_SPEED\_OF\_LIGHT}}{m / s}
\lineiii{gravitational\_constant}{\constant{GSL\_CONST\_MKSA\_GRAVITATIONAL\_CONSTANT}}{m\^{}3 / kg s\^{}2}
\lineiii{plancks\_constant\_h}{\constant{GSL\_CONST\_MKSA\_PLANCKS\_CONSTANT\_H}}{kg m\^{}2 / s}
\lineiii{plancks\_constant\_hbar}{\constant{GSL\_CONST\_MKSA\_PLANCKS\_CONSTANT\_HBAR}}{kg m\^{}2 / s}
\lineiii{vacuum\_permeability}{\constant{GSL\_CONST\_MKSA\_VACUUM\_PERMEABILITY}}{kg m / A\^{}2 s\^{}2}
\lineiii{astronomical\_unit}{\constant{GSL\_CONST\_MKSA\_ASTRONOMICAL\_UNIT}}{m}
\lineiii{light\_year}{\constant{GSL\_CONST\_MKSA\_LIGHT\_YEAR}}{m}
\lineiii{parsec}{\constant{GSL\_CONST\_MKSA\_PARSEC}}{m}
\lineiii{grav\_accel}{\constant{GSL\_CONST\_MKSA\_GRAV\_ACCEL}}{m / s\^{}2}
\lineiii{electron\_volt}{\constant{GSL\_CONST\_MKSA\_ELECTRON\_VOLT}}{kg m\^{}2 / s\^{}2}
\lineiii{mass\_electron}{\constant{GSL\_CONST\_MKSA\_MASS\_ELECTRON}}{kg}
\lineiii{mass\_muon}{\constant{GSL\_CONST\_MKSA\_MASS\_MUON}}{kg}
\lineiii{mass\_proton}{\constant{GSL\_CONST\_MKSA\_MASS\_PROTON}}{kg}
\lineiii{mass\_neutron}{\constant{GSL\_CONST\_MKSA\_MASS\_NEUTRON}}{kg}
\lineiii{rydberg}{\constant{GSL\_CONST\_MKSA\_RYDBERG}}{kg m\^{}2 / s\^{}2}
\lineiii{boltzmann}{\constant{GSL\_CONST\_MKSA\_BOLTZMANN}}{kg m\^{}2 / K s\^{}2}
\lineiii{bohr\_magneton}{\constant{GSL\_CONST\_MKSA\_BOHR\_MAGNETON}}{A m\^{}2}
\lineiii{nuclear\_magneton}{\constant{GSL\_CONST\_MKSA\_NUCLEAR\_MAGNETON}}{A m\^{}2}
\lineiii{electron\_magnetic\_moment}{\constant{GSL\_CONST\_MKSA\_ELECTRON\_MAGNETIC\_MOMENT}}{A m\^{}2}
\lineiii{proton\_magnetic\_moment}{\constant{GSL\_CONST\_MKSA\_PROTON\_MAGNETIC\_MOMENT}}{A m\^{}2}
\lineiii{molar\_gas}{\constant{GSL\_CONST\_MKSA\_MOLAR\_GAS}}{kg m\^{}2 / K mol s\^{}2}
\lineiii{standard\_gas\_volume}{\constant{GSL\_CONST\_MKSA\_STANDARD\_GAS\_VOLUME}}{m\^{}3 / mol}
\lineiii{minute}{\constant{GSL\_CONST\_MKSA\_MINUTE}}{s}
\lineiii{hour}{\constant{GSL\_CONST\_MKSA\_HOUR}}{s}
\lineiii{day}{\constant{GSL\_CONST\_MKSA\_DAY}}{s}
\lineiii{week}{\constant{GSL\_CONST\_MKSA\_WEEK}}{s}
\lineiii{inch}{\constant{GSL\_CONST\_MKSA\_INCH}}{m}
\lineiii{foot}{\constant{GSL\_CONST\_MKSA\_FOOT}}{m}
\lineiii{yard}{\constant{GSL\_CONST\_MKSA\_YARD}}{m}
\lineiii{mile}{\constant{GSL\_CONST\_MKSA\_MILE}}{m}
\lineiii{nautical\_mile}{\constant{GSL\_CONST\_MKSA\_NAUTICAL\_MILE}}{m}
\lineiii{fathom}{\constant{GSL\_CONST\_MKSA\_FATHOM}}{m}
\lineiii{mil}{\constant{GSL\_CONST\_MKSA\_MIL}}{m}
\lineiii{point}{\constant{GSL\_CONST\_MKSA\_POINT}}{m}
\lineiii{texpoint}{\constant{GSL\_CONST\_MKSA\_TEXPOINT}}{m}
\lineiii{micron}{\constant{GSL\_CONST\_MKSA\_MICRON}}{m}
\lineiii{angstrom}{\constant{GSL\_CONST\_MKSA\_ANGSTROM}}{m}
\lineiii{hectare}{\constant{GSL\_CONST\_MKSA\_HECTARE}}{m\^{}2}
\lineiii{acre}{\constant{GSL\_CONST\_MKSA\_ACRE}}{m\^{}2}
\lineiii{barn}{\constant{GSL\_CONST\_MKSA\_BARN}}{m\^{}2}
\lineiii{liter}{\constant{GSL\_CONST\_MKSA\_LITER}}{m\^{}3}
\lineiii{us\_gallon}{\constant{GSL\_CONST\_MKSA\_US\_GALLON}}{m\^{}3}
\lineiii{quart}{\constant{GSL\_CONST\_MKSA\_QUART}}{m\^{}3}
\lineiii{pint}{\constant{GSL\_CONST\_MKSA\_PINT}}{m\^{}3}
\lineiii{cup}{\constant{GSL\_CONST\_MKSA\_CUP}}{m\^{}3}
\lineiii{fluid\_ounce}{\constant{GSL\_CONST\_MKSA\_FLUID\_OUNCE}}{m\^{}3}
\lineiii{tablespoon}{\constant{GSL\_CONST\_MKSA\_TABLESPOON}}{m\^{}3}
\lineiii{teaspoon}{\constant{GSL\_CONST\_MKSA\_TEASPOON}}{m\^{}3}
\lineiii{canadian\_gallon}{\constant{GSL\_CONST\_MKSA\_CANADIAN\_GALLON}}{m\^{}3}
\lineiii{uk\_gallon}{\constant{GSL\_CONST\_MKSA\_UK\_GALLON}}{m\^{}3}
\lineiii{miles\_per\_hour}{\constant{GSL\_CONST\_MKSA\_MILES\_PER\_HOUR}}{m / s}
\lineiii{kilometers\_per\_hour}{\constant{GSL\_CONST\_MKSA\_KILOMETERS\_PER\_HOUR}}{m / s}
\lineiii{knot}{\constant{GSL\_CONST\_MKSA\_KNOT}}{m / s}
\lineiii{pound\_mass}{\constant{GSL\_CONST\_MKSA\_POUND\_MASS}}{kg}
\lineiii{ounce\_mass}{\constant{GSL\_CONST\_MKSA\_OUNCE\_MASS}}{kg}
\lineiii{ton}{\constant{GSL\_CONST\_MKSA\_TON}}{kg}
\lineiii{metric\_ton}{\constant{GSL\_CONST\_MKSA\_METRIC\_TON}}{kg}
\lineiii{uk\_ton}{\constant{GSL\_CONST\_MKSA\_UK\_TON}}{kg}
\lineiii{troy\_ounce}{\constant{GSL\_CONST\_MKSA\_TROY\_OUNCE}}{kg}
\lineiii{carat}{\constant{GSL\_CONST\_MKSA\_CARAT}}{kg}
\lineiii{unified\_atomic\_mass}{\constant{GSL\_CONST\_MKSA\_UNIFIED\_ATOMIC\_MASS}}{kg}
\lineiii{gram\_force}{\constant{GSL\_CONST\_MKSA\_GRAM\_FORCE}}{kg m / s\^{}2}
\lineiii{pound\_force}{\constant{GSL\_CONST\_MKSA\_POUND\_FORCE}}{kg m / s\^{}2}
\lineiii{kilopound\_force}{\constant{GSL\_CONST\_MKSA\_KILOPOUND\_FORCE}}{kg m / s\^{}2}
\lineiii{poundal}{\constant{GSL\_CONST\_MKSA\_POUNDAL}}{kg m / s\^{}2}
\lineiii{calorie}{\constant{GSL\_CONST\_MKSA\_CALORIE}}{kg m\^{}2 / s\^{}2}
\lineiii{btu}{\constant{GSL\_CONST\_MKSA\_BTU}}{kg m\^{}2 / s\^{}2}
\lineiii{therm}{\constant{GSL\_CONST\_MKSA\_THERM}}{kg m\^{}2 / s\^{}2}
\lineiii{horsepower}{\constant{GSL\_CONST\_MKSA\_HORSEPOWER}}{kg m\^{}2 / s\^{}3}
\lineiii{bar}{\constant{GSL\_CONST\_MKSA\_BAR}}{kg / m s\^{}2}
\lineiii{std\_atmosphere}{\constant{GSL\_CONST\_MKSA\_STD\_ATMOSPHERE}}{kg / m s\^{}2}
\lineiii{torr}{\constant{GSL\_CONST\_MKSA\_TORR}}{kg / m s\^{}2}
\lineiii{meter\_of\_mercury}{\constant{GSL\_CONST\_MKSA\_METER\_OF\_MERCURY}}{kg / m s\^{}2}
\lineiii{inch\_of\_mercury}{\constant{GSL\_CONST\_MKSA\_INCH\_OF\_MERCURY}}{kg / m s\^{}2}
\lineiii{inch\_of\_water}{\constant{GSL\_CONST\_MKSA\_INCH\_OF\_WATER}}{kg / m s\^{}2}
\lineiii{psi}{\constant{GSL\_CONST\_MKSA\_PSI}}{kg / m s\^{}2}
\lineiii{poise}{\constant{GSL\_CONST\_MKSA\_POISE}}{kg / m / s}
\lineiii{stokes}{\constant{GSL\_CONST\_MKSA\_STOKES}}{m\^{}2 / s}
\lineiii{faraday}{\constant{GSL\_CONST\_MKSA\_FARADAY}}{A s / mol}
\lineiii{electron\_charge}{\constant{GSL\_CONST\_MKSA\_ELECTRON\_CHARGE}}{A s}
\lineiii{gauss}{\constant{GSL\_CONST\_MKSA\_GAUSS}}{kg / A s\^{}2}
\lineiii{stilb}{\constant{GSL\_CONST\_MKSA\_STILB}}{cd / m\^{}2}
\lineiii{lumen}{\constant{GSL\_CONST\_MKSA\_LUMEN}}{cd sr}
\lineiii{lux}{\constant{GSL\_CONST\_MKSA\_LUX}}{cd sr / m\^{}2}
\lineiii{phot}{\constant{GSL\_CONST\_MKSA\_PHOT}}{cd sr / m\^{}2}
\lineiii{footcandle}{\constant{GSL\_CONST\_MKSA\_FOOTCANDLE}}{cd sr / m\^{}2}
\lineiii{lambert}{\constant{GSL\_CONST\_MKSA\_LAMBERT}}{cd sr / m\^{}2}
\lineiii{footlambert}{\constant{GSL\_CONST\_MKSA\_FOOTLAMBERT}}{cd sr / m\^{}2}
\lineiii{curie}{\constant{GSL\_CONST\_MKSA\_CURIE}}{1 / s}
\lineiii{roentgen}{\constant{GSL\_CONST\_MKSA\_ROENTGEN}}{A s / kg}
\lineiii{rad}{\constant{GSL\_CONST\_MKSA\_RAD}}{m\^{}2 / s\^{}2}
\lineiii{solar\_mass}{\constant{GSL\_CONST\_MKSA\_SOLAR\_MASS}}{kg}
\lineiii{bohr\_radius}{\constant{GSL\_CONST\_MKSA\_BOHR\_RADIUS}}{m}
\lineiii{vacuum\_permittivity}{\constant{GSL\_CONST\_MKSA\_VACUUM\_PERMITTIVITY}}{A\^{}2 s\^{}4 / kg m\^{}3}
\end{longtableiii}

\section[\protect\module{pygsl.const.cgsm} --- Scientific constants in cgsm units]
{\protect\module{pygsl.const.cgsm} \\ Scientific constants in cgsm units} 
\label{cha:const-cgs-module}

\begin{verbatim}
from pygsl.const import cgsm
print "a teaspoon contains %g ml"%cgs.teaspoon
\end{verbatim}

You can access the following constants:\nopagebreak
\begin{longtableiii}{l|l|l}{constant}{Name}{gsl Name}{unit or value}
\lineiii{speed\_of\_light}{\constant{GSL\_CONST\_CGSM\_SPEED\_OF\_LIGHT}}{cm / s}
\lineiii{gravitational\_constant}{\constant{GSL\_CONST\_CGSM\_GRAVITATIONAL\_CONSTANT}}{cm\^{}3 / g s\^{} 2}
\lineiii{plancks\_constant\_h}{\constant{GSL\_CONST\_CGSM\_PLANCKS\_CONSTANT\_H}}{g cm\^{}2 / s}
\lineiii{plancks\_constant\_hbar}{\constant{GSL\_CONST\_CGSM\_PLANCKS\_CONSTANT\_HBAR}}{g cm\^{}2 / s}
\lineiii{vacuum\_permeability}{\constant{GSL\_CONST\_CGSM\_VACUUM\_PERMEABILITY}}{cm g / A\^{}2 s\^{}2}
\lineiii{astronomical\_unit}{\constant{GSL\_CONST\_CGSM\_ASTRONOMICAL\_UNIT}}{cm}
\lineiii{light\_year}{\constant{GSL\_CONST\_CGSM\_LIGHT\_YEAR}}{cm}
\lineiii{parsec}{\constant{GSL\_CONST\_CGSM\_PARSEC}}{cm}
\lineiii{grav\_accel}{\constant{GSL\_CONST\_CGSM\_GRAV\_ACCEL}}{cm / s\^{}2}
\lineiii{electron\_volt}{\constant{GSL\_CONST\_CGSM\_ELECTRON\_VOLT}}{g cm\^{}2 / s\^{}2}
\lineiii{mass\_electron}{\constant{GSL\_CONST\_CGSM\_MASS\_ELECTRON}}{g}
\lineiii{mass\_muon}{\constant{GSL\_CONST\_CGSM\_MASS\_MUON}}{g}
\lineiii{mass\_proton}{\constant{GSL\_CONST\_CGSM\_MASS\_PROTON}}{g}
\lineiii{mass\_neutron}{\constant{GSL\_CONST\_CGSM\_MASS\_NEUTRON}}{g}
\lineiii{rydberg}{\constant{GSL\_CONST\_CGSM\_RYDBERG}}{g cm\^{}2 / s\^{}2}
\lineiii{boltzmann}{\constant{GSL\_CONST\_CGSM\_BOLTZMANN}}{g cm\^{}2 / K s\^{}2}
\lineiii{bohr\_magneton}{\constant{GSL\_CONST\_CGSM\_BOHR\_MAGNETON}}{A cm\^{}2}
\lineiii{nuclear\_magneton}{\constant{GSL\_CONST\_CGSM\_NUCLEAR\_MAGNETON}}{A cm\^{}2}
\lineiii{electron\_magnetic\_moment}{\constant{GSL\_CONST\_CGSM\_ELECTRON\_MAGNETIC\_MOMENT}}{A cm\^{}2}
\lineiii{proton\_magnetic\_moment}{\constant{GSL\_CONST\_CGSM\_PROTON\_MAGNETIC\_MOMENT}}{A cm\^{}2}
\lineiii{molar\_gas}{\constant{GSL\_CONST\_CGSM\_MOLAR\_GAS}}{g cm\^{}2 / K mol s\^{}2}
\lineiii{standard\_gas\_volume}{\constant{GSL\_CONST\_CGSM\_STANDARD\_GAS\_VOLUME}}{cm\^{}3 / mol}
\lineiii{minute}{\constant{GSL\_CONST\_CGSM\_MINUTE}}{s}
\lineiii{hour}{\constant{GSL\_CONST\_CGSM\_HOUR}}{s}
\lineiii{day}{\constant{GSL\_CONST\_CGSM\_DAY}}{s}
\lineiii{week}{\constant{GSL\_CONST\_CGSM\_WEEK}}{s}
\lineiii{inch}{\constant{GSL\_CONST\_CGSM\_INCH}}{cm}
\lineiii{foot}{\constant{GSL\_CONST\_CGSM\_FOOT}}{cm}
\lineiii{yard}{\constant{GSL\_CONST\_CGSM\_YARD}}{cm}
\lineiii{mile}{\constant{GSL\_CONST\_CGSM\_MILE}}{cm}
\lineiii{nautical\_mile}{\constant{GSL\_CONST\_CGSM\_NAUTICAL\_MILE}}{cm}
\lineiii{fathom}{\constant{GSL\_CONST\_CGSM\_FATHOM}}{cm}
\lineiii{mil}{\constant{GSL\_CONST\_CGSM\_MIL}}{cm}
\lineiii{point}{\constant{GSL\_CONST\_CGSM\_POINT}}{cm}
\lineiii{texpoint}{\constant{GSL\_CONST\_CGSM\_TEXPOINT}}{cm}
\lineiii{micron}{\constant{GSL\_CONST\_CGSM\_MICRON}}{cm}
\lineiii{angstrom}{\constant{GSL\_CONST\_CGSM\_ANGSTROM}}{cm}
\lineiii{hectare}{\constant{GSL\_CONST\_CGSM\_HECTARE}}{cm\^{}2}
\lineiii{acre}{\constant{GSL\_CONST\_CGSM\_ACRE}}{cm\^{}2}
\lineiii{barn}{\constant{GSL\_CONST\_CGSM\_BARN}}{cm\^{}2}
\lineiii{liter}{\constant{GSL\_CONST\_CGSM\_LITER}}{cm\^{}3}
\lineiii{us\_gallon}{\constant{GSL\_CONST\_CGSM\_US\_GALLON}}{cm\^{}3}
\lineiii{quart}{\constant{GSL\_CONST\_CGSM\_QUART}}{cm\^{}3}
\lineiii{pint}{\constant{GSL\_CONST\_CGSM\_PINT}}{cm\^{}3}
\lineiii{cup}{\constant{GSL\_CONST\_CGSM\_CUP}}{cm\^{}3}
\lineiii{fluid\_ounce}{\constant{GSL\_CONST\_CGSM\_FLUID\_OUNCE}}{cm\^{}3}
\lineiii{tablespoon}{\constant{GSL\_CONST\_CGSM\_TABLESPOON}}{cm\^{}3}
\lineiii{teaspoon}{\constant{GSL\_CONST\_CGSM\_TEASPOON}}{cm\^{}3}
\lineiii{canadian\_gallon}{\constant{GSL\_CONST\_CGSM\_CANADIAN\_GALLON}}{cm\^{}3}
\lineiii{uk\_gallon}{\constant{GSL\_CONST\_CGSM\_UK\_GALLON}}{cm\^{}3}
\lineiii{miles\_per\_hour}{\constant{GSL\_CONST\_CGSM\_MILES\_PER\_HOUR}}{cm / s}
\lineiii{kilometers\_per\_hour}{\constant{GSL\_CONST\_CGSM\_KILOMETERS\_PER\_HOUR}}{cm / s}
\lineiii{knot}{\constant{GSL\_CONST\_CGSM\_KNOT}}{cm / s}
\lineiii{pound\_mass}{\constant{GSL\_CONST\_CGSM\_POUND\_MASS}}{g}
\lineiii{ounce\_mass}{\constant{GSL\_CONST\_CGSM\_OUNCE\_MASS}}{g}
\lineiii{ton}{\constant{GSL\_CONST\_CGSM\_TON}}{g}
\lineiii{metric\_ton}{\constant{GSL\_CONST\_CGSM\_METRIC\_TON}}{g}
\lineiii{uk\_ton}{\constant{GSL\_CONST\_CGSM\_UK\_TON}}{g}
\lineiii{troy\_ounce}{\constant{GSL\_CONST\_CGSM\_TROY\_OUNCE}}{g}
\lineiii{carat}{\constant{GSL\_CONST\_CGSM\_CARAT}}{g}
\lineiii{unified\_atomic\_mass}{\constant{GSL\_CONST\_CGSM\_UNIFIED\_ATOMIC\_MASS}}{g}
\lineiii{gram\_force}{\constant{GSL\_CONST\_CGSM\_GRAM\_FORCE}}{cm g / s\^{}2}
\lineiii{pound\_force}{\constant{GSL\_CONST\_CGSM\_POUND\_FORCE}}{cm g / s\^{}2}
\lineiii{kilopound\_force}{\constant{GSL\_CONST\_CGSM\_KILOPOUND\_FORCE}}{cm g / s\^{}2}
\lineiii{poundal}{\constant{GSL\_CONST\_CGSM\_POUNDAL}}{cm g / s\^{}2}
\lineiii{calorie}{\constant{GSL\_CONST\_CGSM\_CALORIE}}{g cm\^{}2 / s\^{}2}
\lineiii{btu}{\constant{GSL\_CONST\_CGSM\_BTU}}{g cm\^{}2 / s\^{}2}
\lineiii{therm}{\constant{GSL\_CONST\_CGSM\_THERM}}{g cm\^{}2 / s\^{}2}
\lineiii{horsepower}{\constant{GSL\_CONST\_CGSM\_HORSEPOWER}}{g cm\^{}2 / s\^{}3}
\lineiii{bar}{\constant{GSL\_CONST\_CGSM\_BAR}}{g / cm s\^{}2}
\lineiii{std\_atmosphere}{\constant{GSL\_CONST\_CGSM\_STD\_ATMOSPHERE}}{g / cm s\^{}2}
\lineiii{torr}{\constant{GSL\_CONST\_CGSM\_TORR}}{g / cm s\^{}2}
\lineiii{meter\_of\_mercury}{\constant{GSL\_CONST\_CGSM\_METER\_OF\_MERCURY}}{g / cm s\^{}2}
\lineiii{inch\_of\_mercury}{\constant{GSL\_CONST\_CGSM\_INCH\_OF\_MERCURY}}{g / cm s\^{}2}
\lineiii{inch\_of\_water}{\constant{GSL\_CONST\_CGSM\_INCH\_OF\_WATER}}{g / cm s\^{}2}
\lineiii{psi}{\constant{GSL\_CONST\_CGSM\_PSI}}{g / cm s\^{}2}
\lineiii{poise}{\constant{GSL\_CONST\_CGSM\_POISE}}{g / cm s}
\lineiii{stokes}{\constant{GSL\_CONST\_CGSM\_STOKES}}{cm\^{}2 / s}
\lineiii{faraday}{\constant{GSL\_CONST\_CGSM\_FARADAY}}{A s / mol}
\lineiii{electron\_charge}{\constant{GSL\_CONST\_CGSM\_ELECTRON\_CHARGE}}{A s}
\lineiii{gauss}{\constant{GSL\_CONST\_CGSM\_GAUSS}}{g / A s\^{}2}
\lineiii{stilb}{\constant{GSL\_CONST\_CGSM\_STILB}}{cd / cm\^{}2}
\lineiii{lumen}{\constant{GSL\_CONST\_CGSM\_LUMEN}}{cd sr}
\lineiii{lux}{\constant{GSL\_CONST\_CGSM\_LUX}}{cd sr / cm\^{}2}
\lineiii{phot}{\constant{GSL\_CONST\_CGSM\_PHOT}}{cd sr / cm\^{}2}
\lineiii{footcandle}{\constant{GSL\_CONST\_CGSM\_FOOTCANDLE}}{cd sr / cm\^{}2}
\lineiii{lambert}{\constant{GSL\_CONST\_CGSM\_LAMBERT}}{cd sr / cm\^{}2}
\lineiii{footlambert}{\constant{GSL\_CONST\_CGSM\_FOOTLAMBERT}}{cd sr / cm\^{}2}
\lineiii{curie}{\constant{GSL\_CONST\_CGSM\_CURIE}}{1 / s}
\lineiii{roentgen}{\constant{GSL\_CONST\_CGSM\_ROENTGEN}}{A s / g}
\lineiii{rad}{\constant{GSL\_CONST\_CGSM\_RAD}}{cm\^{}2 / s\^{}2}
\lineiii{solar\_mass}{\constant{GSL\_CONST\_CGSM\_SOLAR\_MASS}}{g}
\lineiii{bohr\_radius}{\constant{GSL\_CONST\_CGSM\_BOHR\_RADIUS}}{cm}
\lineiii{vacuum\_permittivity}{\constant{GSL\_CONST\_CGSM\_VACUUM\_PERMITTIVITY}}{A\^{}2 s\^{}4 / g cm\^{}3}
\end{longtableiii}

\section[\protect\module{pygsl.const.num} --- Scientific number constants]
{\protect\module{pygsl.const.num} \\ Scientific number constants} 
\label{cha:const-num-module}

\begin{verbatim}
from pygsl.const import *
# an alternative to
# from pygsl.const.num import *
print "fine structure is 1/137 with an error of %g%%"%(abs(1.0/137.0/fine_structure-1.0)*100.0)
\end{verbatim}

Only two constants are available:\nopagebreak
\begin{longtableiii}{l|l|l}{constant}{Name}{gsl Name}{unit}
\lineiii{fine\_structure}{\constant{GSL\_CONST\_NUM\_FINE\_STRUCTURE}}{1}
\lineiii{avogadro}{\constant{GSL\_CONST\_NUM\_AVOGADRO}}{1 / mol}
\end{longtableiii}


\chapter[\protect\module{pygsl.chebyshev}]
{\protect\module{pygsl.chebyshev}}
\label{cha:statistics-module}

\declaremodule{standard}{pygsl.chebyshev}
\moduleauthor{Pierre Schnizer}{schnizer@users.sourceforge.net}

\begin{classdesc}{cheb_series}{}
  This base class can be instantiated by its name
\end{classdesc}
\begin{verbatim}
import pygsl.chebyshev
s=pygsl.chebyshev.cheb_series()
\end{verbatim}

\begin{methoddesc}{__init__}{n}\index{__init__}
            n ... number of coefficients        
\end{methoddesc}
\begin{methoddesc}{init}{f, a, b}\index{init}
        This function computes the Chebyshev approximation for the
        function F over the range (a,b) to the previously specified order.
        The computation of the Chebyshev approximation is an O($n^2$)
        process, and requires n function evaluations.

            f ... a gsl_function
            a ... lower limit
            b ... upper limit
        
\end{methoddesc}
\begin{methoddesc}{eval}{x}\index{eval}
        This function evaluates the Chebyshev series at a given point X.
\end{methoddesc}
\begin{methoddesc}{eval_err}{x}\index{eval_err}
         This function computes the Chebyshev series  at a given point X,
         estimating both the series RESULT and its absolute error ABSERR.
         The error estimate is made from the first neglected term in the
         series.
\end{methoddesc}
\begin{methoddesc}{eval_n}{n, x}\index{eval_n}
         This function evaluates the Chebyshev series at a given point
         x, to (at most) the given order n
\end{methoddesc}
\begin{methoddesc}{eval_n_err}{n, x}\index{eval_n_err}
        This function evaluates a Chebyshev series at a given point X,
        estimating both the series RESULT and its absolute error ABSERR,
        to (at most) the given order ORDER.  The error estimate is made
        from the first neglected term in the series.
\end{methoddesc}

\begin{methoddesc}{calc_deriv}{}\index{calc_deriv}
        This method computes the derivative of the series CS. It returns
        a new instance of the cheb_series class.
\end{methoddesc}
\begin{methoddesc}{calc_integ}{}\index{calc_integ}
        This method computes the integral of the series CS. It returns
        a new instance of the cheb_series class.
\end{methoddesc}
\begin{methoddesc}{get_a}{}\index{get_a}
        Get the lower boundary of the current representation       
\end{methoddesc}
\begin{methoddesc}{get_b}{}\index{get_b}
        Get the upper boundary of the current representation        
\end{methoddesc}
\begin{methoddesc}{get_coefficients}{}\index{get_coefficients}
        Get the chebyshev coefficients.         
\end{methoddesc}
\begin{methoddesc}{get_f}{}\index{get_f}
        Get the value f (what is it ?) The documentation does not tell anything
        about it.        
\end{methoddesc}
\begin{methoddesc}{get_order_sp}{}\index{get_order_sp}
        Get the value f (what is it ?) The documentation does not tell anything
        about it.        
\end{methoddesc}
\begin{methoddesc}{set_a}{}\index{set_a}
        Set the lower boundary of the current representation        
\end{methoddesc}
\begin{methoddesc}{set_b}{}\index{set_b}
        Set the upper boundary of the current         
\end{methoddesc}
\begin{methoddesc}{set_coefficients}{}\index{set_coefficients}
        Sets the chebyshev coefficients. 
\end{methoddesc}
\begin{methoddesc}{set_f}{f}\index{set_f}
        Set the value f (what is it ?)        
\end{methoddesc}
\begin{methoddesc}{set_order_sp}{...}\index{set_order_sp}
        Set the value f (what is it ?)        
\end{methoddesc}


\begin{funcdesc}{gsl_function}{f, params}\index{gsl_function}

    This class defines the callbacks known as gsl_function to
    gsl.

    e.g to supply the function f:
    
    def f(x, params):
        a = params[0]
        b = parmas[1]
        c = params[3]
        return a * x ** 2 + b * x + c

    to some solver, use

    function = gsl_function(f, params)
    
\end{funcdesc}

%%% Local Variables: 
%%% mode: latex
%%% TeX-master: "ref"
%%% End: 

\chapter[\protect\module{pygsl.deriv} --- NumericalDifferentiation]%
{\protect\module{pygsl.deriv} \\ Numerical Differentiation}
\label{cha:diff-module}

\declaremodule{extension}{pygsl.deriv}%
 \moduleauthor{Pierre  Schnizer}{schnizer@users.sourceforge.net}%
 \modulesynopsis{Numerical  Differentiation}%

\begin{quote}
  This chapter describes the available functions for numerical differentiation.
\end{quote}

The functions described in this chapter compute numerical derivatives by finite
differencing.  An adaptive algorithm is used to find the best choice of finite
difference and to estimate the error in the derivative. This module supersedes
the diff module which has been deprecated with the release of GSL-1. XXX


\begin{funcdesc}{central}{func, x, h}
  This function computes the numerical derivative of the function \var{func} at
  the point \var{x} using an adaptive central difference algorithm with a step
  size of h.  A tuple \code{(result, error)} is returned with the derivative
  and its estimated absolute error.
\end{funcdesc}

\begin{funcdesc}{backward}{func, x, h}
  This function computes the numerical derivative of the function \var{func} at
  the point \var{x} using an adaptive backward difference algorithm with a step
  size of h.  The function \var{func} is evaluated only at points smaller than
  \var{x} and at \var{x} itself.  A tuple \code{(result, error)} is returned
  with the derivative and its estimated absolute error.
\end{funcdesc}

\begin{funcdesc}{forward}{func, x, h}
  This function computes the numerical derivative of the function \var{func} at
  the point \var{x} using an adaptive forward difference algorithm with a step
  size of h.  The function \var{func} is evaluated only at points greater than
  \var{x} and at \var{x} itself.  A tuple \code{(result, error)} is returned
  with the derivative and its estimated absolute error.
\end{funcdesc}


\begin{seealso}
  The algorithms used by these functions are described in the following book:
  \seetext{S.D.\ Conte and Carl de Boor, \emph{Elementary Numerical Analysis:
      An Algorithmic Approach}, McGraw-Hill, 1972.}
\end{seealso}



%% Local Variables:
%% mode: LaTeX
%% mode: auto-fill
%% fill-column: 79
%% indent-tabs-mode: nil
%% ispell-dictionary: "british"
%% reftex-fref-is-default: nil
%% TeX-auto-save: t
%% TeX-command-default: "pdfeLaTeX"
%% TeX-master: "pygsl"
%% TeX-parse-self: t
%% End:


\chapter[\protect\module{pygsl.histogram} --- Histogram Types]
{\protect\module{pygsl.histogram} \\ Histogram Types}
\label{cha:histogram-module}
\declaremodule{extension}{pygsl.histogram}
\moduleauthor{Achim G\"adke}{achimgaedke@users.sourceforge.net}

This chapter is about the \class{histogram} and \class{histogram2d} type that
are contained in \module{pygsl.histogram}.

\section{\protect\class{histogram} --- 1-dimensional histograms}

\begin{classdesc}{histogram}{\texttt{long} size \code{|} \class{histogram} h}
This type implements all methods on \ctype{struct gsl_histogram}.
\end{classdesc}

\begin{methoddesc}{alloc}{\texttt{long} length}
allocate necessary space, \hfill returns \texttt{None}
\end{methoddesc}
\begin{methoddesc}{set_ranges_uniform}{\texttt{float} upper, \texttt{float} lower}
set the ranges to uniform distance, \hfill returns \texttt{None}
\end{methoddesc}
\begin{methoddesc}{reset}{}
sets all bin values to 0, \hfill returns \texttt{None}
\end{methoddesc}
\begin{methoddesc}{increment}{\texttt{float} x}
increments corresponding bin, \hfill returns \texttt{None}
\end{methoddesc}
\begin{methoddesc}{accumulate}{\texttt{float} x, \texttt{float} weight}
adds the weight to corresponding bin, \hfill returns \texttt{None}
\end{methoddesc}
\begin{methoddesc}{max}{}
returns upper range, \hfill as \texttt{float}
\end{methoddesc}
\begin{methoddesc}{min}{}
returns lower range, \hfill as \texttt{float}
\end{methoddesc}
\begin{methoddesc}{bins}{}
returns number of bins, \hfill as \texttt{long}
\end{methoddesc}
\begin{methoddesc}{get}{\texttt{long} n}
gets value of indexed bin, \hfill returns \texttt{float}
\end{methoddesc}
\begin{methoddesc}{get_range}{\texttt{long} n}
gets upper and lower range of indexed bin, \hfill returns \texttt{(float,float)}
\end{methoddesc}
\begin{methoddesc}{find}{\texttt{float} x}
finds index of corresponding bin, \hfill returns \texttt{long}
\end{methoddesc}
\begin{methoddesc}{max_val}{}
returns maximal bin value, \hfill as \texttt{float}
\end{methoddesc}
\begin{methoddesc}{max_bin}{}
returns bin index with maximal value, \hfill as \texttt{long}
\end{methoddesc}
\begin{methoddesc}{min_val}{}
returns minimal bin value, \hfill as \texttt{float}
\end{methoddesc}
\begin{methoddesc}{min_bin}{}
returns bin index with minimal value, \hfill as \texttt{long}
\end{methoddesc}
\begin{methoddesc}{mean}{}
returns mean of histogram, \hfill as \texttt{float}
\end{methoddesc}
\begin{methoddesc}{sigma}{}
returns std deviation of histogram, \hfill as \texttt{float}
\end{methoddesc}
\begin{methoddesc}{sum}{}
returns sum of bin values, \hfill as \texttt{float}
\end{methoddesc}
\begin{methoddesc}{set_ranges}{\texttt{sequence} ranges}
sets range according given sequence, \hfill returns \texttt{None}
\end{methoddesc}
\begin{methoddesc}{shift}{\texttt{float} offset}
shifts the contents of the bins by the given offset, \hfill returns
\texttt{None}
\end{methoddesc}
\begin{methoddesc}{scale}{\texttt{float} scale}
multiplies the contents of the bins by the given scale, \hfill returns \texttt{None}
\end{methoddesc}
\begin{methoddesc}{equal_bins_p}{}
true if the all of the individual bin ranges are identical, \hfill returns \texttt{int}
\end{methoddesc}
\begin{methoddesc}{add}{\texttt{histogram} h}
adds the contents of the bins, \hfill returns \texttt{None}
\end{methoddesc}
\begin{methoddesc}{sub}{\texttt{histogram} h}
substracts the contents of the bins, \hfill returns \texttt{None}
\end{methoddesc}
\begin{methoddesc}{mul}{\texttt{histogram} h}
multiplicates the contents of the bins, \hfill returns \texttt{None}
\end{methoddesc}
\begin{methoddesc}{div}{\texttt{histogram} h}
divides the contents of the bins, \hfill returns \texttt{None}
\end{methoddesc}
\begin{methoddesc}{clone}{\texttt{histogram} h}
returns a new copy of this histogram, \hfill returns new \texttt{histogram}
\end{methoddesc}
\begin{methoddesc}{copy}{\texttt{histogram} h}
copies the given histogram to myself, \hfill returns \texttt{None}
\end{methoddesc}
\begin{methoddesc}{read}{\texttt{file} input}
reads histogram binary data from file, \hfill returns \texttt{None}
\end{methoddesc}
\begin{methoddesc}{write}{\texttt{file} output}
writes histogram binary data to file, \hfill returns \texttt{None}
\end{methoddesc}
\begin{methoddesc}{scanf}{\texttt{file} input}
reads histogram data from file using scanf, \hfill returns \texttt{None}
\end{methoddesc}
\begin{methoddesc}{printf}{\texttt{file} output}
writes histogram data to file using printf, \hfill returns \texttt{None}
\end{methoddesc}


Some mapping operations are supported, too:\nopagebreak
\begin{tableii}{l|l}{texttt}{Mapping syntax}{Effect}
\lineii{histogram[index]}{returns the value of the indexed bin}
\lineii{histogram[index]=value}{sets the value of the indexed bin}
\lineii{len(histogram)}{returns the length of the histogram}
\end{tableii}

\begin{seealso}
For the special meaning and details please consult the GNU Scientific Library
reference.
\end{seealso}


\section{\protect\class{histogram2d} --- 2-dimensional histograms}

\begin{classdesc}{histogram2d}{\texttt{long} size x, \texttt{long} size y
                               \code{|} \class{histogram2d} h}
This class holds a 2d array and 2 sets of ranges for x and y coordinates for a
two paramter statistical event. It can be constructed by size parameters or
as a copy from another histogram. Most of the methods are the same as of
\class{histogram}.
\end{classdesc}

\begin{methoddesc}{set_ranges_uniform}{\texttt{float} xmin, \texttt{float} xmax,
                                       \texttt{float} ymin, \texttt{float} ymax}
set the ranges to uniform distance, \hfill returns \texttt{None}
\end{methoddesc}
\begin{methoddesc}{alloc}{\texttt{long} nx, \texttt{long} ny}
allocate necessary space, \hfill returns \texttt{None}
\end{methoddesc}
\begin{methoddesc}{reset}{}
sets all bin values to 0, \hfill returns \texttt{None}
\end{methoddesc}
\begin{methoddesc}{increment}{\texttt{float} x, \texttt{float} y}
increments corresponding bin, \hfill returns \texttt{None}
\end{methoddesc}
\begin{methoddesc}{accumulate}{\texttt{float} x, \texttt{float} y,
                               \texttt{float} weight}
adds the weight to corresponding bin, \hfill returns \texttt{None}
\end{methoddesc}
\begin{methoddesc}{xmax}{}
returns upper x range \hfill as \texttt{float}
\end{methoddesc}
\begin{methoddesc}{xmin}{}
returns lower x range \hfill as \texttt{float}
\end{methoddesc}
\begin{methoddesc}{ymax}{}
returns upper y range \hfill as \texttt{float}
\end{methoddesc}
\begin{methoddesc}{ymin}{}
returns lower y range \hfill as \texttt{float}
\end{methoddesc}
\begin{methoddesc}{nx}{}
returns number of x bins \hfill as \texttt{long}
\end{methoddesc}
\begin{methoddesc}{ny}{}
returns number of y bins \hfill as \texttt{long}
\end{methoddesc}
\begin{methoddesc}{get}{\texttt{long} i, \texttt{long} j}
gets value of indexed bin,\hfill returns \texttt{float}
\end{methoddesc}
\begin{methoddesc}{get_xrange}{\texttt{long} i}
gets upper and lower x range of indexed bin,
\hfill returns \texttt{(float \textrm{lower}, float \textrm{upper})}
\end{methoddesc}
\begin{methoddesc}{get_yrange}{\texttt{long} j}
gets upper and lower y range of indexed bin,
\hfill returns \texttt{(float \textrm{lower}, float \textrm{upper})}
\end{methoddesc}
\begin{methoddesc}{find}{\texttt{float} x, \texttt{float} y}
finds index pair of corresponding value pair,
\hfill returns (\texttt{long},\texttt{long})
\end{methoddesc}
\begin{methoddesc}{max_val}{}
returns maximal bin value \hfill as \texttt{float}
\end{methoddesc}
\begin{methoddesc}{max_bin}{}
returns bin index with maximal value \hfill as \texttt{long}
\end{methoddesc}
\begin{methoddesc}{min_val}{}
returns minimal bin value \hfill as \texttt{float}
\end{methoddesc}
\begin{methoddesc}{min_bin}{}
returns bin index with minimal value \hfill as \texttt{long}
\end{methoddesc}
\begin{methoddesc}{sum}{}
returns sum of bin values \hfill as \texttt{float}
\end{methoddesc}
\begin{methoddesc}{xmean}{}
returns x mean of histogram \hfill as \texttt{float}
\end{methoddesc}
\begin{methoddesc}{xsigma}{}
returns x std deviation of histogram \hfill as \texttt{float}
\end{methoddesc}
\begin{methoddesc}{ymean}{}
returns y mean of histogram \hfill as\texttt{float}
\end{methoddesc}
\begin{methoddesc}{ysigma}{}
returns y std deviation of histogram \hfill as \texttt{float}
\end{methoddesc}
\begin{methoddesc}{cov}{}
returns covariance of histogram \hfill as \texttt{float}
\end{methoddesc}
\begin{methoddesc}{set_ranges}{sequence xranges, sequence yranges}
set the ranges according to given sequences, \hfill returns \texttt{None}
\end{methoddesc}
\begin{methoddesc}{shift}{\texttt{float} offset}
shifts the contents of the bins by the given offset, \hfill returns \texttt{None}
\end{methoddesc}
\begin{methoddesc}{scale}{\texttt{float} scale}
multiplies the contents of the bins by the given scale, \hfill returns \texttt{None}
\end{methoddesc}
\begin{methoddesc}{equal_bins_p}{}
true if the all of the individual bin ranges are identical, \hfill returns \texttt{int}
\end{methoddesc}
\begin{methoddesc}{add}{\class{histogram2d} h}
adds the contents of the bins, \hfill returns \texttt{None}
\end{methoddesc}
\begin{methoddesc}{sub}{\class{histogram2d} h}
substracts the contents of the bins, \hfill returns \texttt{None}
\end{methoddesc}
\begin{methoddesc}{mul}{\class{histogram2d} h}
multiplicates the contents of the bins, \hfill returns \texttt{None}
\end{methoddesc}
\begin{methoddesc}{div}{\class{histogram2d} h}
divides the contents of the bins, \hfill returns \texttt{None}
\end{methoddesc}
\begin{methoddesc}{clone}{}
returns a copy instance of \hfill\class{histogram2d}
\end{methoddesc}
\begin{methoddesc}{copy}{\class{histogram2d} h}
copies the given histogram to myself, \hfill returns \texttt{None}
\end{methoddesc}
\begin{methoddesc}{read}{file input}
reads histogram binary data from file, \hfill returns \texttt{None}
\end{methoddesc}
\begin{methoddesc}{writew}{file output}
writes histogram binary data to file, \hfill returns \texttt{None}
\end{methoddesc}
\begin{methoddesc}{scanf}{file input}
reads histogram data from file using scanf, \hfill returns \texttt{None}
\end{methoddesc}
\begin{methoddesc}{printf}{file input}
writes histogram data to file using printf, \hfill returns \texttt{None}
\end{methoddesc}

Some mapping operations are supported, too:\nopagebreak
\begin{tableii}{l|l}{code}{Mapping syntax}{Effect}
\lineii{histogram[x\_index,y\_index]}{returns the value of the indexed bin}
\lineii{histogram[x\_index,y\_index]=value}{sets the value of the indexed bin}
\lineii{len(histogram)}{returns the size of the histogram, i.e nx$\times$ny}
\end{tableii}


\begin{seealso}
For the special meaning and details please consult the GNU Scientific Library
reference.
\end{seealso}

\section{\protect\class{histogram_pdf} and \protect\class{histogram2d_pdf}}

To be implemented\dots

%% Local Variables:
%% mode: LaTeX
%% mode: auto-fill
%% fill-column: 90
%% indent-tabs-mode: nil
%% ispell-dictionary: "american"
%% reftex-fref-is-default: nil
%% TeX-auto-save: t
%% TeX-command-default: "pdfeLaTeX"
%% TeX-master: "pygsl"
%% TeX-parse-self: t
%% End:


\chapter[\protect\module{pygsl.rng} --- Random Number Generators]
{\protect\module{pygsl.rng} \\ Random Number Generators}
\label{cha:rng-module}
% $Id$
% pygsl/doc/rng.tex

\declaremodule{standard}{pygsl.rng}%
\moduleauthor{Pierre Schnizer}{schnizer@users.sourceforge.net}%
\moduleauthor{Original Author: Achim G\"adke}{achimgaedke@users.sourceforge.net}%

This chapter introduces the random number generator type provided by \module{pygsl}.

\section{Random Number Generators}

All random number generatores are the same python type (PyGSL_rng), but using the
approbriate GSL random generator for generating the random numbers. Use the method
\code{name} to get the name of the rng used internally.

Methods of
this type \pytype{rng} provide the transformation to different probability
distributions and give access to basic properties of random number generators. 
All methods allow to pass one optional integer. Then the method will be evaluated n times and the result
will be returned as an array.

\begin{pytypedesc}{rng}{\texttt{string} typenamme \code{|} \class{rng} r}
  This base class can be instantiated by its name
\begin{verbatim}
import pygsl.rng
my_ran0=pygsl.rng.ran0()
\end{verbatim}
.
\end{pytypedesc}
The type of the allocated generator is given by the method
\begin{methoddesc}{name}{}
  which returns its name as string.
\end{methoddesc}
All generators can be seeded with
\begin{methoddesc}{set}{seed}
  which sets the internal seed according to the positive integer {\tt seed}. Zero as seed
  has a special meaning, please read details in the gsl reference.
\end{methoddesc}
The basic returned number type is integer, these are generated by
\begin{methoddesc}{get}{}
  which returns the next number of the pseudo random sequence.
\end{methoddesc}
All methods support internal sampling; i.e each method has an optional integer. 
If given it will return a sample of the approbriate size.
\begin{methoddesc}{get}{|n}
  will return the next n numbers of the pseudo random sequence.
\end{methoddesc}

Basic information about these numbers can be obtained by
\begin{methoddesc}{max}{}
  maximum number of this sequence and
\end{methoddesc}
\begin{methoddesc}{min}{}
  minimum number of this sequence.
\end{methoddesc}
Implemented uniform probability densities are:
\begin{methoddesc}{uniform}{}
  returns a real number between $[0,1)$.
\end{methoddesc}
\begin{methoddesc}{uniform_pos}{}
  returns a real number between $(0,1)$ --- this excludes 0.
\end{methoddesc}
\begin{methoddesc}{uniform_int}{upper limit}
  returns an integer from 0 to the upper limit (exclusive). If this limit is larger than
  the number of return values of the underlying generator, \exception{pygsl.gsl_Error} is
  raised.
\end{methoddesc}
Furthermore a lot of derived probability densities can be used:
\begin{methoddesc}{gaussian}{sigma}
  gaussian distribution with mean 0 and given sigma \hfill returns {\tt float}
\end{methoddesc}
\begin{methoddesc}{gaussian\_ratio\_method}{sigma}
  gaussian distribution with mean 0 and given sigma.  This variate uses the
  Kinderman-Monahan ratio method.  \hfill returns {\tt float}
\end{methoddesc}
\begin{methoddesc}{ugaussian}{}
  gaussian distribution with unit sigma and mean 0.  \hfill returns {\tt float}
\end{methoddesc}
\begin{methoddesc}{ugaussian\_ratio\_method}{}
  gaussian distribution with unit sigma and mean 0.  This variate uses the
  Kinderman-Monahan ratio method.  \hfill returns {\tt float}
\end{methoddesc}
\begin{methoddesc}{gaussian\_tail}{sigma, a}
  upper tail of a Gaussian distribution with standard deviation sigma>0.  \hfill returns
  {\tt float}
\end{methoddesc}
\begin{methoddesc}{ugaussian\_tail}{a}
  upper tail of a Gaussian distribution with unit standard deviation.  \hfill returns {\tt
    float}
\end{methoddesc}
\begin{methoddesc}{bivariate\_gaussian}{sigma\_x, sigma\_y, rho}
  pair of correlated gaussian variates, with mean zero, correlation coefficient rho and
  standard deviations sigma\_x and sigma\_y in the x and y directions \hfill returns~{\tt
    (float,float)}
\end{methoddesc}
\begin{methoddesc}{exponential}{mu}
  \hfill returns {\tt float}
\end{methoddesc}
\begin{methoddesc}{laplace}{mu}
  \hfill returns {\tt float}
\end{methoddesc}
\begin{methoddesc}{exppow}{mu, a}
  \hfill returns {\tt float}
\end{methoddesc}
\begin{methoddesc}{cauchy}{mu}
  \hfill returns {\tt float}
\end{methoddesc}
\begin{methoddesc}{rayleigh}{sigma}
  \hfill returns {\tt float}
\end{methoddesc}
\begin{methoddesc}{rayleigh\_tail}{a, sigma}
  \hfill returns {\tt float}
\end{methoddesc}
\begin{methoddesc}{levy}{mu,a}
  \hfill returns {\tt float}
\end{methoddesc}
\begin{methoddesc}{levy_skew}{mu,a,beta}
  \hfill returns {\tt float}
\end{methoddesc}
\begin{methoddesc}{gamma}{a, b}
  \hfill returns {\tt float}
\end{methoddesc}
\begin{methoddesc}{gamma\_int}{long a}
  \hfill returns {\tt float}
\end{methoddesc}
\begin{methoddesc}{flat}{a, b}
  \hfill returns {\tt float}
\end{methoddesc}
\begin{methoddesc}{lognormal}{zeta, sigma}
  \hfill returns {\tt float}
\end{methoddesc}
\begin{methoddesc}{chisq}{nu}
  \hfill returns {\tt float}
\end{methoddesc}
\begin{methoddesc}{fdist}{nu1, nu2}
  \hfill returns {\tt float}
\end{methoddesc}
\begin{methoddesc}{tdist}{nu}
  \hfill returns {\tt float}
\end{methoddesc}
\begin{methoddesc}{beta}{a, b}
  \hfill returns {\tt float}
\end{methoddesc}
\begin{methoddesc}{logistic}{mu}
  \hfill returns {\tt float}
\end{methoddesc}
\begin{methoddesc}{pareto}{a, b}
  \hfill returns {\tt float}
\end{methoddesc}
\begin{methoddesc}{dir\_2d}{}
  \hfill returns {\tt (float, float)}
\end{methoddesc}
\begin{methoddesc}{dir\_2d\_trig\_method}{}
  \hfill returns {\tt (float, float)}
\end{methoddesc}
\begin{methoddesc}{dir\_3d}{}
  \hfill returns {\tt (float, float, float)}
\end{methoddesc}
\begin{methoddesc}{dir\_nd}{int n}
  \hfill returns {\tt (float, \dots, float)}
\end{methoddesc}
\begin{methoddesc}{weibull}{mu, a}
  \hfill returns {\tt float}
\end{methoddesc}
\begin{methoddesc}{gumbel1}{a, b}
  \hfill returns {\tt float}
\end{methoddesc}
\begin{methoddesc}{gumbel2}{}
\end{methoddesc}
\begin{methoddesc}{poisson}{}
\end{methoddesc}
\begin{methoddesc}{bernoulli}{}
\end{methoddesc}
\begin{methoddesc}{binomial}{}
\end{methoddesc}
\begin{methoddesc}{negative\_binomial}{}
\end{methoddesc}
\begin{methoddesc}{pascal}{}
\end{methoddesc}
\begin{methoddesc}{geometric}{}
\end{methoddesc}
\begin{methoddesc}{hypergeometric}{}
\end{methoddesc}
\begin{methoddesc}{logarithmic}{}
\end{methoddesc}
\begin{methoddesc}{landau}{}
\end{methoddesc}
\begin{methoddesc}{erlang}{}
\end{methoddesc}


The different generator classes are created according to the output of
\code{gsl_rng_types_setup()} when the \module{pygsl.rng} is loaded. Here is the list of
children from \class{rng} for gsl-1.2: \newline \class{rng_borosh13}, \class{rng_coveyou},
\class{rng_cmrg}, \class{rng_fishman18}, \class{rng_fishman20}, \class{rng_fishman2x},
\class{rng_gfsr4}, \class{rng_knuthran}, \class{rng_knuthran2}, \class{rng_lecuyer21},
\class{rng_minstd}, \class{rng_mrg}, \class{rng_mt19937}, \class{rng_mt19937_1999},
\class{rng_mt19937_1998}, \class{rng_r250}, \class{rng_ran0}, \class{rng_ran1},
\class{rng_ran2}, \class{rng_ran3}, \class{rng_rand}, \class{rng_rand48},
\class{rng_random128_bsd}, \class{rng_random128_glibc2}, \class{rng_random128_libc5},
\class{rng_random256_bsd}, \class{rng_random256_glibc2}, \class{rng_random256_libc5},
\class{rng_random32_bsd}, \class{rng_random32_glibc2}, \class{rng_random32_libc5},
\class{rng_random64_bsd}, \class{rng_random64_glibc2}, \class{rng_random64_libc5},
\class{rng_random8_bsd}, \class{rng_random8_glibc2}, \class{rng_random8_libc5},
\class{rng_random_bsd}, \class{rng_random_glibc2}, \class{rng_random_libc5},
\class{rng_randu}, \class{rng_ranf}, \class{rng_ranlux}, \class{rng_ranlux389},
\class{rng_ranlxd1}, \class{rng_ranlxd2}, \class{rng_ranlxs0}, \class{rng_ranlxs1},
\class{rng_ranlxs2}, \class{rng_ranmar}, \class{rng_slatec}, \class{rng_taus},
\class{rng_taus2}, \class{rng_taus113}, \class{rng_transputer}, \class{rng_tt800},
\class{rng_uni}, \class{rng_uni32}, \class{rng_vax}, \class{rng_waterman14}, and
\class{rng_zuf}.  
\newline 

The default generator of the \class{rng} defaults to {\tt rng_mt19937} but can be set from the
environment variable \envvar{GSL_RNG_TYPE} using the function \function{rng.env_setup()}.

\section{Probability Density Functions}


\section{Using probability densities with random number generators}


%% Local Variables:
%% mode: LaTeX
%% mode: auto-fill
%% fill-column: 90
%% indent-tabs-mode: nil
%% ispell-dictionary: "british"
%% reftex-fref-is-default: nil
%% TeX-auto-save: t
%% TeX-command-default: "pdfeLaTeX"
%% TeX-master: "pygsl"
%% TeX-parse-self: t
%% End:


%\chapter[\protect\module{pygsl.sf} --- Special Functions]
%{\protect\module{pygsl.sf} \\ Special Functions}
%\label{cha:sf-module}
%\declaremodule{extension}{pygsl.sf}
\moduleauthor{Achim G\"adke}{achimgaedke@users.sourceforge.net}

This chapter shows you the list of implemented special function and explains
details of error handling and return values.

\section{Function list}

\begin{longtableii}{l|l}{texttt}{Function}{Description}
\lineii{}{ToDo}
\end{longtableii}

\section{Return values}

\section{Error handling}


\chapter[\protect\module{pygsl.sum} --- Series acceleration]{
  \protect\module{pygsl.sum} \\ Series acceleration}
\label{cha:sum-module}

\declaremodule{extension}{pygsl.sum}
\modulesynopsis{Series acceleration.}

This chapter describes the use of the series acceleration tools based
on the Levin $u$-transform.  This method takes a small number of terms
from the start of a series and uses a systematic approximation to
compute an extrapolated value and an estimate of its error. The
$u$-transform works for both convergent and divergent series,
including asymptotic series.

\begin{equation}
  \label{eq:levin}
  \function{levin_sum}\code{(a)} = (A, \epsilon)
  \qquad\text{where}
  \qquad
  A \approx \sum_{n=0}^{\infty} a_{n} \pm \epsilon, 
\end{equation}
$\code{a} = [a_{0}, a_{1}, \ldots, a_{n}]$, and $\epsilon$ is an
estimate of the absolute error.

Note: This function is intended for summing analytic series where each
term is known to high accuracy, and the rounding errors are assumed to
originate from finite precision. They are taken to be relative errors
of order \constant{GSL_DBL_EPSILON} for each term (as defined in the
\GSL{} source code).

\section{Function list}
\begin{funcdesc}{levin_sum}{a, truncate=False, info_dict=None}
  Return ($A, \epsilon$) where $A$ is the approximated sum of the
  series~(\ref{eq:levin}) and $\epsilon$ is its absolute error
  estimate.

  The calculation of the error in the extrapolated value is an
  O$(N^2)$ process, which is expensive in time and memory.  A full
  table of intermediate values and derivatives through to O$(N)$ must
  be computed and stored, but this does give a reliable error
  estimate.

  A faster but less reliable method which estimates the error from the
  convergence of the extrapolated value is employed if \var{truncate}
  is \code{True}.  This attempts to estimate the error from the
  ``truncation error'' in the extrapolation, the difference between
  the final two approximations. Using this method avoids the need to
  compute an intermediate table of derivatives because the error is
  estimated from the behavior of the extrapolated value
  itself. Consequently this algorithm is an O$(N)$ process and only
  requires O$(N)$ terms of storage. If the series converges sufficiently
  fast then this procedure can be acceptable. It is appropriate to use
  this method when there is a need to compute many extrapolations of
  series with similar convergence properties at high-speed. For
  example, when numerically integrating a function defined by a
  parameterized series where the parameter varies only slightly. A
  reliable error estimate should be computed first using the full
  algorithm described above in order to verify the consistency of the
  results.

  If a dictionary is passed as \var{info_dict}, then two entries will
  be added: \var{info_dict}\code{['terms_used']} will be the number of
  terms used\footnote{Note that it appears that this is the number of
    terms \emph{beyond} the first term that are used.  I.e.\ there are
    a total of $\var{terms_used}+1$ terms:
    \begin{equation}
      \var{sum_plain} = 
      \sum_{n=0}^{\var{terms_used}}
      a_{n}
    \end{equation}}
  and \var{info_dict}\code{['sum_plain']} will be the sum of these terms without
  acceleration.
\end{funcdesc}

\section{Further Reading}
For details on the underlying implementation of these functions please
consult the \GSL{} reference manual.  The algorithms used by these
functions are described Fessler \textit{et al.} (1983).  The theory of
the $u$-transform was presented Levin in 1973, and a review paper from
2000 by Homeier is available online.

\begin{seealso}
  \seetext{T.~Fessler, W.~F.~Ford, D.~A.~Smith, \textit{hurry: An
      acceleration algorithm for scalar sequences and series}. ACM
    Transactions on Mathematical Software, \textbf{9}(3):346--354,
    (1983), and Algorithm 602 9(3):355--357, 1983.}
  \seetext{D.~Levin, \textit{Development of Non-Linear Transformations
      for Improving Convergence of Sequences,} Intern.~J.~Computer
    Math. \textbf{B3}:371--388, (1973).}
  \seetitle[http://arXiv.org/abs/math/0005209]{Herbert H.~H.~Homeier,
    \textit{Scalar Levin-Type Sequence Transformations.}}{}
\end{seealso}
%%% Local Variables: 
%%% mode: latex
%%% TeX-master: "ref"
%%% End: 

\chapter[\protect\module{pygsl.statistics} --- Statistics
functions]{\protect\module{pygsl.statistics} \\ Statistics functions}
\label{cha:statistics-module}

\declaremodule{extension}{pygsl.statistics}
\moduleauthor{Pierre Schnizer}{schnizer@users.sourceforge.net}
\moduleauthor{Original Author: Jochen K\"upper}{jochen@jochen-kuepper.de}
\modulesynopsis{Statistical functions.}

\index{mean}
\index{standard deviation}
\index{variance}
\index{estimated standard deviation}
\index{estimated variance}
\index{t-test}
\index{range}
\index{min}
\index{max}
\index{kurtosis}
\index{skewness}
\index{autocorrelation}
\index{covariance}

\begin{quote}
   This chapter describes the statistical functions in the library.  The basic
   statistical functions include routines to compute the mean, variance and
   standard deviation. More advanced functions allow you to calculate absolute
   deviations, skewness, and kurtosis as well as the median and arbitrary
   percentiles.
\end{quote}

The algorithms provided here use recurrence relations to compute average
quantities in a stable way, without large intermediate values that might
overflow.  All functions work on any Python sequence (of appropriate
data-type), but see section \ref{sec:stat:speed-considerations} for advantages
and drawbacks of different kinds of input data.

\begin{seealso}
   For details on the underlying implementation of these functions please
   consult the \GSL{} reference manual.
\end{seealso}



\section{Organization of the module}
\label{sec:stat:organization}

Individual parts of the \gsl{} functions names, providing artificial namespaces
in C, are mapped to modules and submodules in \pygsl{}.  That is,
\cfunction{gsl_stats_mean} can be found as \function{pygsl.statistics.mean} and
\cfunction{gsl_stats_long_mean} as \function{pygsl.statistics.long.mean}.

The functions in the module are available in versions for datasets in the
standard and \numpy{} floating-point and integer types. The generic versions
available in the \module{pygsl.statistics} module are using the generic \gsl{}
\ctype{double} versions.  The submodules use \gsl{} functions according to the
submodule name, e.g. long for \module{pygsl.statistics.long}.

Implemented submodules are \module{char}, \module{uchar}, \module{short},
\module{int}, \module{long}, \module{float}, and \module{double}. The latter
one also serves as default and is used whenever you don't expclicitely state a
different datatype. In most cases it is appropriate to simply use the default
implementation as it covers the widest range of the real space, offers high
precision, and as such is simple to use. If you have a sequence of all integer
values it is straightforward to use \module{pygsl.statistics.long} functions as
these use an implementation corresponding to Pythons \class{Float}-type. These
implemented submodules represent all numeric datatypes available in Python
(\class{Int}, \class{Float}) besides \class{Long Int} which has no
representation in standard C, as well as all numeric datatypes available in
\numpy{} that have corresponding implementations in \gsl{} (on 32 bit systems
these are: Character, UnsigendInt8, Int16, Int32, Int, Float32, Float).



\section{Available functions}
\label{sec:stat:available-functions}

\subsection{Mean, Standard Deviation, and Variance}
\label{sec:stat:mean-stddev-var}

\begin{funcdesc}{mean}{x}\index{mean}
   Arithmetic mean (\emph{sample mean}) of \var{x}:
   \begin{equation}
      \hat\mu = \frac{1}{N} \sum x_i
   \end{equation}
\end{funcdesc}

\begin{funcdesc}{variance}{x}\index{variance}
   Estimated (\emph{sample}) variance of \var{x}:
   \begin{equation}
      \hat\sigma^2 = \frac{1}{N-1} \sum (x_i - \hat\mu)^2
   \end{equation}
   This function computes the mean via a call to \function{mean}.  If you have
   already computed the mean then you can pass it directly to
   \function{variance_m}.
\end{funcdesc}

\begin{funcdesc}{variance_m}{x, mean}\index{variance}
   Estimated (\emph{sample}) variance of \var{x} relative to \var{mean}:
   \begin{equation}
      \hat\sigma^2 = \frac{1}{N-1} \sum (x_i - mean)^2
   \end{equation}
\end{funcdesc}

\begin{funcdesc}{sd}{x}
\end{funcdesc}
\begin{funcdesc}{sd_m}{x, mean}\index{sd}\index{mean}
   The standard deviation is defined as the square root of the variance of
   \var{x}.  These functions returns the square root of the respective
   variance-functions above.
\end{funcdesc}

\begin{funcdesc}{variance_with_fixed_mean}{x, mean}\index{variance}\index{mean}
   Compute an unbiased estimate of the variance of \var{x} when the population
   mean \var{mean} of the underlying distribution is known \emph{a priori}.  In
   this case the estimator for the variance uses the factor $1/N$ and the
   sample mean $\hat\mu$ is replaced by the known population mean $\mu$:
   \begin{equation}
      \hat\sigma^2 = \frac{1}{N} \sum (x_i - \mu)^2
   \end{equation}
\end{funcdesc}


\subsection{Absolute deviation}
\label{sec:stat:absolute-deviation}

\begin{funcdesc}{absdev}{data}
   Compute the absolute deviation from the mean of \var{data} The absolute
   deviation from the mean is defined as
   \begin{equation}
      absdev  = (1/N) \sum |x_i - \hat\mu|
   \end{equation}
   where $x_i$ are the elements of the dataset \var{data}.  The absolute
   deviation from the mean provides a more robust measure of the width of a
   distribution than the variance.  This function computes the mean of
   \var{data} via a call to \function{mean}.
\end{funcdesc}

\begin{funcdesc}{absdev_m}{data, mean}
   Compute the absolute deviation of the dataset \var{data} relative to the
   given value of \var{mean}
   \begin{equation}
      absdev  = (1/N) \sum |x_i - mean|
   \end{equation}
   This function is useful if you have already computed the mean of \var{data}
   (and want to avoid recomputing it), or wish to calculate the absolute
   deviation relative to another value (such as zero, or the median).
\end{funcdesc}


\subsection{Higher moments (skewness and kurtosis)}
\label{sec:stat:higher-moments}

\begin{funcdesc}{skew}{data}
   Compute the skewness of \var{data}.  The skewness is defined as
   \begin{equation}
      skew = (1/N) \sum ((x_i - \hat\mu)/\hat\sigma)^3
   \end{equation}
   where $x_i$ are the elements of the dataset \var{data}.  The skewness
   measures the asymmetry of the tails of a distribution.
   
   The function computes the mean and estimated standard deviation of
   \var{data} via calls to \function{mean} and \function{sd}.
\end{funcdesc}


\begin{funcdesc}{skew_m_sd}{data, mean, sd}
   Compute the skewness of the dataset \var{data} using the given values of the
   mean \var{mean} and standard deviation var{sd}
   \begin{equation}
      skew = (1/N) \sum ((x_i - mean)/sd)^3
   \end{equation}
   These functions are useful if you have already computed the mean and
   standard deviation of \var{data} and want to avoid recomputing them.
\end{funcdesc}


\begin{funcdesc}{kurtosis}{data}
   Compute the kurtosis of \var{data}.  The kurtosis is defined as
   \begin{equation}
      kurtosis = ((1/N) \sum ((x_i - \hat\mu)/\hat\sigma)^4) - 3
   \end{equation}
   The kurtosis measures how sharply peaked a distribution is, relative to its
   width.  The kurtosis is normalized to zero for a gaussian distribution.
\end{funcdesc}


\begin{funcdesc}{kurtosis_m_sd}{data, mean, sd}
   This function computes the kurtosis of the dataset \var{data} using the
   given values of the mean \var{mean} and standard deviation \var{sd}
   \begin{equation}
      kurtosis = ((1/N) \sum ((x_i - mean)/sd)^4) - 3
   \end{equation}
   This function is useful if you have already computed the mean and standard
   deviation of \var{data} and want to avoid recomputing them.
\end{funcdesc}



\subsection{Autocorrelation}
\label{sec:stat:autocorrelation}

\begin{funcdesc}{lag1_autocorrelation}{x}
   Computes the lag-1 autocorrelation of the dataset \var{x}
   \begin{equation}
      a_1 = \frac{\sum^{n}_{i = 1} (x_{i} - \hat\mu) (x_{i-1} - \hat\mu)}{
         \sum^{n}_{i = 1} (x_{i} - \hat\mu) (x_{i} - \hat\mu)}
   \end{equation}
 \end{funcdesc}

\begin{funcdesc}{lag1_autocorrelation_m}{x, mean}
   Computes the lag-1 autocorrelation of the dataset \var{x} using the given
   value of the mean \var{mean}.
   \begin{equation}
      a_1 = \frac{\sum_{i = 1}^{n} (x_{i} - \var{mean}) (x_{i-1} - \var{mean})}{
         \sum^{n}_{i = 1} (x_{i} - \var{mean}) (x_{i} - \var{mean})}
   \end{equation}
\end{funcdesc}



\subsection{Covariance}
\label{sec:stat:covariance}

\begin{funcdesc}{covariance}{x, y}
   Computes the covariance of the datasets \var{x} and \var{y} which must be of
   same length.
   \begin{equation}
      c = \frac{1}{n-1} \sum^{n}_{i=1} (x_i - \hat x) (y_i - \hat y)
   \end{equation}
\end{funcdesc}

\begin{funcdesc}{lag1_autocorrelation_m}{x, y, mean\_x, mean\_y}
   Computes the covariance of the datasets \var{x} and \var{y} using the given
   values of the means \var{mean\_x} and \var{mean\_y}. The datasets \var{x}
   and \var{y} must be of equal length.
   \begin{equation}
      c = \frac{1}{n-1} \sum^{n}_{i=1} (x_i - \var{mean\_x}) (y_i -
      \var{mean\_y})
   \end{equation}
\end{funcdesc}




\subsection{Maximum and Minimum values}
\label{sec:stat:max-min-value}


\begin{funcdesc}{max}{data}
   This function returns the maximum value in \var{data}.  The maximum value is
   defined as the value of the element $x_i$ which satisfies $x_i \ge x_j$ for
   all $j$.
   
   If you want instead to find the element with the largest absolute magnitude
   you will need to apply `fabs' or `abs' to your data before calling this
   function.
\end{funcdesc}

\begin{funcdesc}{min}{data}
   This function returns the minimum value in \var{data}. The maximum value is
   defined as the value of the element $x_i$ which satisfies $x_i \le x_j$ for
   all $j$.
   
   If you want instead to find the element with the smallest absolute magnitude
   you will need to apply `fabs' or `abs' to your data before calling this
   function.
\end{funcdesc}

\begin{funcdesc}{minmax}{data}
   This function returns both the minimum and maximum values of \var{data},
   determined in a single pass.
\end{funcdesc}

\begin{funcdesc}{max_index}{data}
   This function returns the index of the maximum value in \var{data}.  The
   maximum value is defined as the value of the element $x_i$ which satisfies
   $x_i \ge x_j$ for all $j$.  When there are several equal maximum elements
   then the first one is chosen.
\end{funcdesc}

\begin{funcdesc}{min_index}{data}
   This function returns the index of the minimum value in \var{data}.  The
   minimum value is defined as the value of the element $x_i$ which satisfies
   $x_i \le x_j$ for all $j$.  When there are several equal minimum elements
   then the first one is chosen.
\end{funcdesc}

\begin{funcdesc}{minmax_index}{data}
   This function returns the indexes of the minimum and maximum values of
   \var{data}, determined in a single pass.
\end{funcdesc}



\subsection{Median and Percentiles}
\label{sec:stat:median-percentiles}

The median and percentile functions described in this section operate on sorted
data.  For convenience we use "quantiles", measured on a scale of 0 to 1,
instead of percentiles (which use a scale of 0 to 100).

\begin{funcdesc}{median_from_sorted_data}{data}
   This function returns the median value of \var{data}.  The elements of the
   array must be in ascending numerical order.  There are no checks to see
   whether the data are sorted, so the function \function{sort} should always
   be used first.
   
   When the dataset has an odd number of elements the median is the value of
   element (n-1)/2.  When the dataset has an even number of elements the median
   is the mean of the two nearest middle values, elements (n-1)/2 and n/2.
   Since the algorithm for computing the median involves interpolation this
   function always returns a floating-point number, even for integer data
   types.
\end{funcdesc}

\begin{funcdesc}{quantile_from_sorted_data}{data, F}
   This function returns a quantile value of \var{data}.  The elements of the
   array must be in ascending numerical order.  The quantile is determined by
   the \var{F}, a fraction between 0 and 1.  For example, to compute the value
   of the 75th percentile \var{F} should have the value 0.75.
   
   There are no checks to see whether the data are sorted, so the function
   \function{sort} should always be used first.
   
   The quantile is found by interpolation, using the formula
   \begin{equation}
      quantile = (1 - \delta) x_i + \delta x_{i+1}
   \end{equation}
   where $i$ is $floor((n - 1)f)$ and $\delta$ is $(n-1)f - i$.
   
   Thus the minimum value of the array (\var{data[0]}) is given by \var{F}
   equal to zero, the maximum value (\var{data[-1]}) is given by \var{F} equal
   to one and the median value is given by \var{F} equal to 0.5.  Since the
   algorithm for computing quantiles involves interpolation this function
   always returns a floating-point number, even for integer data types.
\end{funcdesc}


\subsection{Weighted Samples}
\label{sec:weighted-samples}

The functions described in this section allow the computation of statistics for
weighted samples.  The functions accept an array of samples, $x_i$, with
associated weights, $w_i$.  Each sample $x_i$ is considered as having been
drawn from a Gaussian distribution with variance $\sigma_i^2$.  The sample
weight $w_i$ is defined as the reciprocal of this variance, $w_i =
1/\sigma_i^2$.  Setting a weight to zero corresponds to removing a sample from
a dataset.

\begin{funcdesc}{wmean}{w, data}
   This function returns the weighted mean of the dataset \var{data} using the
   set of weights \var{w}.  The weighted mean is defined as
   \begin{equation}
      \hat\mu = (\sum w_i x_i) / (\sum w_i)
   \end{equation}
\end{funcdesc}

\begin{funcdesc}{wvariance }{w, data}
   This function returns the estimated variance of the dataset \var{data},
   using the set of weights \var{w}.  The estimated variance of a weighted
   dataset is defined as
   \begin{equation}
      \hat\sigma^2 = ((\sum w_i)/((\sum w_i)^2 - \sum (w_i^2))) \sum w_i (x_i - \hat\mu)^2
   \end{equation}
   Note that this expression reduces to an unweighted variance with the
   familiar $1/(N-1)$ factor when there are $N$ equal non-zero weights.
\end{funcdesc}

\begin{funcdesc}{wvariance_m}{w, data, wmean}
   This function returns the estimated variance of the weighted dataset
   \var{data} using the given weighted mean \var{wmean}.
\end{funcdesc}

\begin{funcdesc}{wsd}{w, data}
   The standard deviation is defined as the square root of the variance.  This
   function returns the square root of the corresponding variance function
   \function{wvariance} above.
\end{funcdesc}

\begin{funcdesc}{wsd_m}{w, data, wmean}
   This function returns the square root of the corresponding variance function
   \function{wvariance_m} above.
\end{funcdesc}

\begin{funcdesc}{wvariance_with_fixed_mean}{w, data, mean}
   This function computes an unbiased estimate of the variance of weighted
   dataset \var{data} when the population mean \var{mean} of the underlying
   distribution is known _a priori_.  In this case the estimator for the
   variance replaces the sample mean $\hat\mu$ by the known population mean
   $\mu$,
   \begin{equation}
      \hat\sigma^2 = (\sum w_i (x_i - \mu)^2) / (\sum w_i)
   \end{equation}
\end{funcdesc}

\begin{funcdesc}{wsd_with_fixed_mean}{w, data, mean}
   The standard deviation is defined as the square root of the variance.  This
   function returns the square root of the corresponding variance function
   above.
\end{funcdesc}

\begin{funcdesc}{wabsdev}{w, data}
   This function computes the weighted absolute deviation from the weighted
   mean of \var{data}.  The absolute deviation from the mean is defined as
   \begin{equation}
      absdev = (\sum w_i |x_i - \hat\mu|) / (\sum w_i)
   \end{equation}
\end{funcdesc}

\begin{funcdesc}{wabsdev_m}{w, data, wmean}
   This function computes the absolute deviation of the weighted dataset DATA
   about the given weighted mean WMEAN.
\end{funcdesc}

\begin{funcdesc}{wskew}{w, data}
   This function computes the weighted skewness of the dataset DATA.
   \begin{equation}
      skew = (\sum w_i ((x_i - xbar)/\sigma)^3) / (\sum w_i)
   \end{equation}
\end{funcdesc}

\begin{funcdesc}{wskew_m_sd}{w, data, mean, wsd}
   This function computes the weighted skewness of the dataset \var{data} using
   the given values of the weighted mean and weighted standard deviation,
   \var{wmean} and \var{wsd}.
\end{funcdesc}

\begin{funcdesc}{wkurtosis}{w, data}
   This function computes the weighted kurtosis of the dataset \var{data}. The
   kurtosis is defined as 
   \begin{equation}
      kurtosis = ((\sum w_i ((x_i - xbar)/sigma)^4) / (\sum w_i)) - 3
   \end{equation}
\end{funcdesc}

\begin{funcdesc}{wkurtosis_m_sd}{w, data, mean, wsd}
   This function computes the weighted kurtosis of the dataset \var{data} using
   the given values of the weighted mean and weighted standard deviation,
   \var{wmean} and \var{wsd}.
\end{funcdesc}





\section{Further Reading}
\label{sec:stat:further-reading}

See the \gsl{} reference manual for a description of all available functions
and the calculations they perform.

The standard reference for almost any topic in statistics is the multi-volume
\emph{Advanced Theory of Statistics} by Kendall and Stuart.  Many statistical
concepts can be more easily understood by a Bayesian approach.  The book by
Gelman, Carlin, Stern and Rubin gives a comprehensive coverage of the subject.
For physicists the Particle Data Group provides useful reviews of Probability
and Statistics in the "Mathematical Tools" section of its Annual Review of
Particle Physics.
   
\begin{seealso}
   \seetext{Maurice Kendall, Alan Stuart, and J.\ Keith Ord: \emph{The Advanced
         Theory of Statistics} (multiple volumes) reprinted as \emph{Kendall's
         Advanced Theory of Statistics}.  Wiley, ISBN 047023380X.}
   
   \seetext{Andrew Gelman, John B.\ Carlin, Hal S.\ Stern, Donald B.\ Rubin:
      \emph{Bayesian Data Analysis}.  Chapman \& Hall, ISBN 0412039915.}
   
   \seetext{R.M.\ Barnett et al: Review of Particle Properties. \emph{Physical
         Review} \textbf{D54}, 1 (1996).}
   
   \seetext{D.E.\ Groom et al., \emph{The European Physical Journal}
      \textbf{C15}, 1 (2000) and \emph{2001 off-year partial update for the
         2002 edition} available on the PDG WWW pages (URL:
      \url{http://pdg.lbl.gov/}).}
   
   \seetext{Siegmund Brandt: \emph{Datenanalyse}, 4th ed. 1999, Spektrum,
      Heidelberg, ISBN 3827401585.}  
   
   \seetext{Siegmund Brandt: \emph{Data Analysis}. 3rd ed. 1998, Springer,
      Berlin, ISBN 0387984984.}
\end{seealso}


%% Local Variables:
%% mode: LaTeX
%% mode: auto-fill
%% fill-column: 79
%% indent-tabs-mode: nil
%% ispell-dictionary: "british"
%% reftex-fref-is-default: nil
%% TeX-auto-save: t
%% TeX-command-default: "pdfeLaTeX"
%% TeX-master: "pygsl"
%% TeX-parse-self: t
%% End:


[common]
sensorid = default

[virtual_file_system]
data_fs_url = default
fs_url = default

[session]
timeout = 30

[daemon]
;user = conpot
;group = conpot

[json]
enabled = False
filename = /var/log/conpot.json

[sqlite]
enabled = False

[mysql]
enabled = False
device = /tmp/mysql.sock
host = localhost
port = 3306
db = conpot
username = conpot
passphrase = conpot
socket = tcp        ; tcp (sends to host:port), dev (sends to mysql device/socket file)

[syslog]
enabled = False
device = /dev/log
host = localhost
port = 514
facility = local0
socket = dev        ; udp (sends to host:port), dev (sends to device)

[hpfriends]
enabled = False
host = hpfriends.honeycloud.net
port = 20000
ident = 3Ykf9Znv
secret = 4nFRhpm44QkG9cvD
channels = ["conpot.events", ]

[taxii]
enabled = False
host = taxiitest.mitre.org
port = 80
inbox_path = /services/inbox/default/
use_https = False

[fetch_public_ip]
enabled = True
urls = ["http://whatismyip.akamai.com/", "http://wgetip.com/"]

[change_mac_addr]
enabled = False
iface = eth0
addr = 00:de:ad:be:ef:00


\appendix

\chapter[\protect\module{pygsl.ieee} --- Floating Point Unit Support]
{\protect\module{pygsl.ieee} \\ Floating Point Unit Support}
\label{cha:ieee-module}
\declaremodule{extension}{pygsl.ieee}
\moduleauthor{Achim G\"adke}{achimgaedke@users.sourceforge.net}

This chapter lists features to configure the ``Floating Point Unit'' of your machine.
The exact behaviour of your Floating Point Unit can't be discussed here in general --- its just machine type dependent.

\begin{funcdesc} {set_mode}{int precision, int rounding, int exception\_mask}
the mode has effect on the behaviour during calcualtion, e.g. division by zero or rounding.

The following constants are used as precision argument:
\begin{tableii}{l|l}{constant}{mode value}{definition via gsl}
\lineii{single\_precision}{\code{GSL\_IEEE\_SINGLE\_PRECISION}}
\lineii{double\_precision}{\code{GSL\_IEEE\_DOUBLE\_PRECISION}}
\lineii{extended\_precision}{\code{GSL\_IEEE\_EXTENDED\_PRECISION}}
\end{tableii}
Possible round arguments are:
\begin{tableii}{l|l}{constant}{mode value}{definition via gsl}
\lineii{round\_to\_nearest}{\code{GSL\_IEEE\_ROUND\_TO\_NEAREST}}
\lineii{round\_down}{\code{GSL\_IEEE\_ROUND\_DOWN}}
\lineii{round\_up}{\code{GSL\_IEEE\_ROUND\_UP}}
\lineii{round\_to\_zero}{\code{GSL\_IEEE\_ROUND\_TO\_ZERO}}
\end{tableii}
These exception arguments can be added.
\constant{mask\_all} is the sum of all 5 \constant{mask\_*} constants.
\begin{tableii}{l|l}{constant}{mode value}{definition via gsl}
\lineii{mask\_invalid}{\code{GSL\_IEEE\_MASK\_INVALID}}
\lineii{mask\_denormalized}{\code{GSL\_IEEE\_MASK\_DENORMALIZED}}
\lineii{mask\_division\_by\_zero}{\code{GSL\_IEEE\_MASK\_DIVISION\_BY\_ZERO}}
\lineii{mask\_overflow}{\code{GSL\_IEEE\_MASK\_OVERFLOW}}
\lineii{mask\_underflow}{\code{GSL\_IEEE\_MASK\_UNDERFLOW}}
\lineii{mask\_all}{\code{GSL\_IEEE\_MASK\_ALL}}
\lineii{trap\_inexact}{\code{GSL\_IEEE\_TRAP\_INEXACT}}
\end{tableii}
\end{funcdesc}

\begin{funcdesc} {env\_setup}{}
sets the ieee mode from \envvar{GSL\_IEEE\_MODE}. This is not called any more
automatically  when importing the  \module{pygsl}.
\end{funcdesc}

\begin{funcdesc} {bin\_repr}{float value}
%\cfunction{gsl_ieee_double_to_rep}
returns the binary representation as tuple with the following contents:
\code{(int sign, string mantissa, int exponent, int type)}
These values are used as \constant{type} in \function{bin\_repr}:
\begin{tableii}{l|l}{constant}{type value}{definition via gsl}
\lineii{type\_nan}{\code{GSL\_IEEE\_TYPE\_NAN}}
\lineii{type\_inf}{\code{GSL\_IEEE\_TYPE\_INF}}
\lineii{type\_normal}{\code{GSL\_IEEE\_TYPE\_NORMAL}}
\lineii{type\_denormal}{\code{GSL\_IEEE\_TYPE\_DENORMAL}}
\lineii{type\_zero}{\code{GSL\_IEEE\_TYPE\_ZERO}}
\end{tableii}
\end{funcdesc}

\begin{funcdesc}{isnan}{float value}
determines if the argument is not a valid number
\end{funcdesc}

\begin{funcdesc}{nan}{}
generates a ``not-a-number'' value. This is implemented as function, because of the potential exception generation by your floating-point unit.
\end{funcdesc}

\begin{funcdesc}{isinf}{float value}
returns -1 if the argument represents a negative infinite value and +1 if positive, 0 otherwise
\end{funcdesc}

\begin{funcdesc}{posinf}{}
gives you the representation of ``positive infinity''
\end{funcdesc}

\begin{funcdesc}{neginf}{}
the same as posinf, but negative
\end{funcdesc}

\begin{funcdesc}{finite}{float value}
results in 1 if the value is finite, 0 if it is not a number or infinite
\end{funcdesc}


%\chapter[\protect\module{pygsl.init} --- Library initialisation]
%{\protect\module{pygsl.init} \\ Library initialisation}
%\label{cha:library-initialisation}
%\declaremodule{extension}{pygsl.init}
\moduleauthor{Pierre Schnizer}{schnizer@users.sourceforge.net}
\moduleauthor{Achim G\"adke}{achimgaedke@users.sourceforge.net}

This module is called the first time when loading \module{pygsl}.
All following procedures are called once and before everything other.

\section{Exception handling}
\index{exception handling!initialisation} GSL provides a selectable error
handler, that is called for occuring errors (like domain errors, division by
zero, etc. ).  This is switched off. Instead each wrapper function will check
the error return value and in case of error an python exception is created. 

Here is a python level example:
\begin{verbatim}
import pygsl.histogram
import pygsl.errors
hist=pygsl.histogram.histogram2d(100,100)
try:
   hist[-1,-1]=0
except pygsl.errors.gsl_Error,err:
   print err
\end{verbatim}
Will result
\begin{verbatim}
input domain error: index i lies outside valid range of 0 .. nx - 1
\end{verbatim}


An exception are ufuncs in the testings.sf module (see section\ref{sec:ufuncs}).

%\module{pygsl.init} installs a handler by calling
%\cfunction{gsl_set_error_handler} to set an appropiate exception from
%\module{pygsl.errors}.  A \module{pygsl} interface function should return
%\code{NULL} in case of an error, so the exception is raised.  If this handler
%is called more than once before returning to python, only the first set
%exception is raised.
%
%
% 
% \section{IEEE-mode}
% \index{ieee-mode!initialisation}
% The IEEE mode is set from the environment variable
%  \envvar{GSL_IEEE_MODE} via \cfunction{gsl_ieee_env_setup()}.
% After the initialisation use \module{pygsl.ieee} for manipulation.
% 
% \section{random number generators}
% \index{random number generator!initialisation}
% Also the random number generator can be initialised from the environment variables
%  \envvar{GSL_RNG_TYPE}
% and \envvar{GSL_RNG_SEED} using the gsl function \cfunction{gsl_rng_env_setup()}.
% Each random number generators are initialised with \envvar{GSL_RNG_SEED}.
% 
% The default generator can be created by:\nopagebreak
% \begin{verbatim}
% import pygsl.rng
% my_rng=pygsl.rng.rng()
% print my_rng.name()
% \end{verbatim}




\chapter{GNU Free Documentation License}
\label{cha:free-documentation-license}

Version 1.1, March 2000\\

 Copyright \copyright\ 2000  Free Software Foundation, Inc.\\
     59 Temple Place, Suite 330, Boston, MA  02111-1307  USA\\
 Everyone is permitted to copy and distribute verbatim copies
 of this license document, but changing it is not allowed.

\section*{Preamble}

The purpose of this License is to make a manual, textbook, or other
written document ``free'' in the sense of freedom: to assure everyone
the effective freedom to copy and redistribute it, with or without
modifying it, either commercially or noncommercially.  Secondarily,
this License preserves for the author and publisher a way to get
credit for their work, while not being considered responsible for
modifications made by others.

This License is a kind of ``copyleft'', which means that derivative
works of the document must themselves be free in the same sense.  It
complements the GNU General Public License, which is a copyleft
license designed for free software.

We have designed this License in order to use it for manuals for free
software, because free software needs free documentation: a free
program should come with manuals providing the same freedoms that the
software does.  But this License is not limited to software manuals;
it can be used for any textual work, regardless of subject matter or
whether it is published as a printed book.  We recommend this License
principally for works whose purpose is instruction or reference.

\section{Applicability and Definitions}

This License applies to any manual or other work that contains a
notice placed by the copyright holder saying it can be distributed
under the terms of this License.  The ``Document'', below, refers to any
such manual or work.  Any member of the public is a licensee, and is
addressed as ``you''.

A ``Modified Version'' of the Document means any work containing the
Document or a portion of it, either copied verbatim, or with
modifications and/or translated into another language.

A ``Secondary Section'' is a named appendix or a front-matter section of
the Document that deals exclusively with the relationship of the
publishers or authors of the Document to the Document's overall subject
(or to related matters) and contains nothing that could fall directly
within that overall subject.  (For example, if the Document is in part a
textbook of mathematics, a Secondary Section may not explain any
mathematics.)  The relationship could be a matter of historical
connection with the subject or with related matters, or of legal,
commercial, philosophical, ethical or political position regarding
them.

The ``Invariant Sections'' are certain Secondary Sections whose titles
are designated, as being those of Invariant Sections, in the notice
that says that the Document is released under this License.

The ``Cover Texts'' are certain short passages of text that are listed,
as Front-Cover Texts or Back-Cover Texts, in the notice that says that
the Document is released under this License.

A ``Transparent'' copy of the Document means a machine-readable copy,
represented in a format whose specification is available to the
general public, whose contents can be viewed and edited directly and
straightforwardly with generic text editors or (for images composed of
pixels) generic paint programs or (for drawings) some widely available
drawing editor, and that is suitable for input to text formatters or
for automatic translation to a variety of formats suitable for input
to text formatters.  A copy made in an otherwise Transparent file
format whose markup has been designed to thwart or discourage
subsequent modification by readers is not Transparent.  A copy that is
not ``Transparent'' is called ``Opaque''.

Examples of suitable formats for Transparent copies include plain
ASCII without markup, Texinfo input format, \LaTeX~input format, SGML
or XML using a publicly available DTD, and standard-conforming simple
HTML designed for human modification.  Opaque formats include
PostScript, PDF, proprietary formats that can be read and edited only
by proprietary word processors, SGML or XML for which the DTD and/or
processing tools are not generally available, and the
machine-generated HTML produced by some word processors for output
purposes only.

The ``Title Page'' means, for a printed book, the title page itself,
plus such following pages as are needed to hold, legibly, the material
this License requires to appear in the title page.  For works in
formats which do not have any title page as such, ``Title Page'' means
the text near the most prominent appearance of the work's title,
preceding the beginning of the body of the text.


\section{Verbatim Copying}

You may copy and distribute the Document in any medium, either
commercially or noncommercially, provided that this License, the
copyright notices, and the license notice saying this License applies
to the Document are reproduced in all copies, and that you add no other
conditions whatsoever to those of this License.  You may not use
technical measures to obstruct or control the reading or further
copying of the copies you make or distribute.  However, you may accept
compensation in exchange for copies.  If you distribute a large enough
number of copies you must also follow the conditions in section 3.

You may also lend copies, under the same conditions stated above, and
you may publicly display copies.


\section{Copying in Quantity}

If you publish printed copies of the Document numbering more than 100,
and the Document's license notice requires Cover Texts, you must enclose
the copies in covers that carry, clearly and legibly, all these Cover
Texts: Front-Cover Texts on the front cover, and Back-Cover Texts on
the back cover.  Both covers must also clearly and legibly identify
you as the publisher of these copies.  The front cover must present
the full title with all words of the title equally prominent and
visible.  You may add other material on the covers in addition.
Copying with changes limited to the covers, as long as they preserve
the title of the Document and satisfy these conditions, can be treated
as verbatim copying in other respects.

If the required texts for either cover are too voluminous to fit
legibly, you should put the first ones listed (as many as fit
reasonably) on the actual cover, and continue the rest onto adjacent
pages.

If you publish or distribute Opaque copies of the Document numbering
more than 100, you must either include a machine-readable Transparent
copy along with each Opaque copy, or state in or with each Opaque copy
a publicly-accessible computer-network location containing a complete
Transparent copy of the Document, free of added material, which the
general network-using public has access to download anonymously at no
charge using public-standard network protocols.  If you use the latter
option, you must take reasonably prudent steps, when you begin
distribution of Opaque copies in quantity, to ensure that this
Transparent copy will remain thus accessible at the stated location
until at least one year after the last time you distribute an Opaque
copy (directly or through your agents or retailers) of that edition to
the public.

It is requested, but not required, that you contact the authors of the
Document well before redistributing any large number of copies, to give
them a chance to provide you with an updated version of the Document.


\section{Modifications}

You may copy and distribute a Modified Version of the Document under
the conditions of sections 2 and 3 above, provided that you release
the Modified Version under precisely this License, with the Modified
Version filling the role of the Document, thus licensing distribution
and modification of the Modified Version to whoever possesses a copy
of it.  In addition, you must do these things in the Modified Version:

\begin{itemize}

\item Use in the Title Page (and on the covers, if any) a title distinct
   from that of the Document, and from those of previous versions
   (which should, if there were any, be listed in the History section
   of the Document).  You may use the same title as a previous version
   if the original publisher of that version gives permission.
\item List on the Title Page, as authors, one or more persons or entities
   responsible for authorship of the modifications in the Modified
   Version, together with at least five of the principal authors of the
   Document (all of its principal authors, if it has less than five).
\item State on the Title page the name of the publisher of the
   Modified Version, as the publisher.
\item Preserve all the copyright notices of the Document.
\item Add an appropriate copyright notice for your modifications
   adjacent to the other copyright notices.
\item Include, immediately after the copyright notices, a license notice
   giving the public permission to use the Modified Version under the
   terms of this License, in the form shown in the Addendum below.
\item Preserve in that license notice the full lists of Invariant Sections
   and required Cover Texts given in the Document's license notice.
\item Include an unaltered copy of this License.
\item Preserve the section entitled ``History'', and its title, and add to
   it an item stating at least the title, year, new authors, and
   publisher of the Modified Version as given on the Title Page.  If
   there is no section entitled ``History'' in the Document, create one
   stating the title, year, authors, and publisher of the Document as
   given on its Title Page, then add an item describing the Modified
   Version as stated in the previous sentence.
\item Preserve the network location, if any, given in the Document for
   public access to a Transparent copy of the Document, and likewise
   the network locations given in the Document for previous versions
   it was based on.  These may be placed in the ``History'' section.
   You may omit a network location for a work that was published at
   least four years before the Document itself, or if the original
   publisher of the version it refers to gives permission.
\item In any section entitled ``Acknowledgements'' or ``Dedications'',
   preserve the section's title, and preserve in the section all the
   substance and tone of each of the contributor acknowledgements
   and/or dedications given therein.
\item Preserve all the Invariant Sections of the Document,
   unaltered in their text and in their titles.  Section numbers
   or the equivalent are not considered part of the section titles.
\item Delete any section entitled ``Endorsements''.  Such a section
   may not be included in the Modified Version.
\item Do not retitle any existing section as ``Endorsements''
   or to conflict in title with any Invariant Section.

\end{itemize}

If the Modified Version includes new front-matter sections or
appendices that qualify as Secondary Sections and contain no material
copied from the Document, you may at your option designate some or all
of these sections as invariant.  To do this, add their titles to the
list of Invariant Sections in the Modified Version's license notice.
These titles must be distinct from any other section titles.

You may add a section entitled ``Endorsements'', provided it contains
nothing but endorsements of your Modified Version by various
parties -- for example, statements of peer review or that the text has
been approved by an organization as the authoritative definition of a
standard.

You may add a passage of up to five words as a Front-Cover Text, and a
passage of up to 25 words as a Back-Cover Text, to the end of the list
of Cover Texts in the Modified Version.  Only one passage of
Front-Cover Text and one of Back-Cover Text may be added by (or
through arrangements made by) any one entity.  If the Document already
includes a cover text for the same cover, previously added by you or
by arrangement made by the same entity you are acting on behalf of,
you may not add another; but you may replace the old one, on explicit
permission from the previous publisher that added the old one.

The author(s) and publisher(s) of the Document do not by this License
give permission to use their names for publicity for or to assert or
imply endorsement of any Modified Version.


\section{Combining Documents}

You may combine the Document with other documents released under this
License, under the terms defined in section 4 above for modified
versions, provided that you include in the combination all of the
Invariant Sections of all of the original documents, unmodified, and
list them all as Invariant Sections of your combined work in its
license notice.

The combined work need only contain one copy of this License, and
multiple identical Invariant Sections may be replaced with a single
copy.  If there are multiple Invariant Sections with the same name but
different contents, make the title of each such section unique by
adding at the end of it, in parentheses, the name of the original
author or publisher of that section if known, or else a unique number.
Make the same adjustment to the section titles in the list of
Invariant Sections in the license notice of the combined work.

In the combination, you must combine any sections entitled ``History''
in the various original documents, forming one section entitled
``History''; likewise combine any sections entitled ``Acknowledgements'',
and any sections entitled ``Dedications''.  You must delete all sections
entitled ``Endorsements.''


\section{Collections of Documents}

You may make a collection consisting of the Document and other documents
released under this License, and replace the individual copies of this
License in the various documents with a single copy that is included in
the collection, provided that you follow the rules of this License for
verbatim copying of each of the documents in all other respects.

You may extract a single document from such a collection, and distribute
it individually under this License, provided you insert a copy of this
License into the extracted document, and follow this License in all
other respects regarding verbatim copying of that document.



\section{Aggregation With Independent Works}

A compilation of the Document or its derivatives with other separate
and independent documents or works, in or on a volume of a storage or
distribution medium, does not as a whole count as a Modified Version
of the Document, provided no compilation copyright is claimed for the
compilation.  Such a compilation is called an ``aggregate'', and this
License does not apply to the other self-contained works thus compiled
with the Document, on account of their being thus compiled, if they
are not themselves derivative works of the Document.

If the Cover Text requirement of section 3 is applicable to these
copies of the Document, then if the Document is less than one quarter
of the entire aggregate, the Document's Cover Texts may be placed on
covers that surround only the Document within the aggregate.
Otherwise they must appear on covers around the whole aggregate.


\section{Translation}

Translation is considered a kind of modification, so you may
distribute translations of the Document under the terms of section 4.
Replacing Invariant Sections with translations requires special
permission from their copyright holders, but you may include
translations of some or all Invariant Sections in addition to the
original versions of these Invariant Sections.  You may include a
translation of this License provided that you also include the
original English version of this License.  In case of a disagreement
between the translation and the original English version of this
License, the original English version will prevail.


\section{Termination}

You may not copy, modify, sublicense, or distribute the Document except
as expressly provided for under this License.  Any other attempt to
copy, modify, sublicense or distribute the Document is void, and will
automatically terminate your rights under this License.  However,
parties who have received copies, or rights, from you under this
License will not have their licenses terminated so long as such
parties remain in full compliance.


\section{Future Revisions of This License}

The Free Software Foundation may publish new, revised versions
of the GNU Free Documentation License from time to time.  Such new
versions will be similar in spirit to the present version, but may
differ in detail to address new problems or concerns. See
http://www.gnu.org/copyleft/.

Each version of the License is given a distinguishing version number.
If the Document specifies that a particular numbered version of this
License "or any later version" applies to it, you have the option of
following the terms and conditions either of that specified version or
of any later version that has been published (not as a draft) by the
Free Software Foundation.  If the Document does not specify a version
number of this License, you may choose any version ever published (not
as a draft) by the Free Software Foundation.


% Complete documentation on the extended LaTeX markup used for Python
% documentation is available in ``Documenting Python'', which is part
% of the standard documentation for Python.  It may be found online
% at:
%
%     http://www.python.org/doc/current/doc/doc.html

\documentclass[hyperref]{manual}

% latex2html doesn't know [T1]{fontenc}, so we cannot use that:(
\usepackage{amsmath}
\usepackage[latin1]{inputenc}
\usepackage{textcomp}
\usepackage{hyperref}

% this version does not reset module names at section level
%begin{latexonly}
\makeatletter
\let\py@OldOldChapter=\chapter
\renewcommand{\chapter}{\py@reset%
                        \py@OldOldChapter}
\renewcommand{\section}{\@startsection{section}{1}{\z@}%
   {-3.5ex \@plus -1ex \@minus -.2ex}%
   {2.3ex \@plus.2ex}%
   {\reset@font\Large\py@HeaderFamily}}
\makeatother
%end{latexonly}


% some convenience declarations
\newcommand{\gsl}{GSL}
\newcommand{\GSL}{GNU Scientific Library}
\newcommand{\numpy}{NumPy}
\newcommand{\NUMPY}{Numerical Python}
\newcommand{\pygsl}{PyGSL}
\newcommand{\PYGSL}{PyGSL: Python wrapper of the GNU Scientific Library}

\makeatletter
\newenvironment{pytypedesc}[2]{
  % Using \renewcommand doesn't work for this, for unknown reasons:
  \global\def\py@thisclass{#1}
  \begin{fulllineitems}
    \py@sigline{\strong{pytype }\bfcode{#1}}{#2}%
    \index{#1@{\py@idxcode{#1}} (pytype in \py@thismodule)}
}{\end{fulllineitems}}
\makeatother


\title{PyGSL Reference Manual}

\ifhtml
\author{
  \ulink{Achim G\"adke}{mailto:achimgaedke@users.sourceforge.net}\\
  Technische Universit\"at Darmstadt, Darmstadt, Germany
}
\author{
  \ulink{Pierre Schnizer}{mailto:schnizer@users.sourceforge.net}\\
  Gesellschaft f\"ur Schwerionenforschung, Darmstadt, Germany
}
%\author{
%  \ulink{Jochen K\"upper}{mailto:jochen@jochen-kuepper.de}\\
%  Fritz-Haber-Institut der MPG, Berlin, Germany
%}
%\author{
%  \ulink{S\'ebastien Maret}{mailto:schnizer@users.sourceforge.net}\\
%  Department of Astronomy, University of Michigan, Ann Arbor, USA
%}
\else
%begin{latexonly}
%% pdfelatex (TeXLive 7) doesn't handle \footnotemark in here...
\author{Achim G\"adke \\ 
          Jochen K\"upper \\ 
         %S\'ebastien Maret \\
        Pierre Schnizer}
% Please at least include a long-lived email address!
\authoraddress{
   Technische Universit\"at Darmstadt, Darmstadt, Germany \\
   \email{achimgaedke@users.sourceforge.net} \\
   Gesellschaft f\"ur Schwerionenforschung, Darmstadt, Germany \\
   \email{schnizer@users.sourceforge.net} \\
%   Fritz-Haber-Institut der MPG, Berlin, Germany \\
%   \email{jochen@jochen-kuepper.de} \\
%   Department of Astronomy, University of Michigan, Ann Arbor, USA \\
%   \email{bmaret@users.sourceforge.et} \\
}
%end{latexonly}
\fi

\date{October, 2008}            % update before release!
                                % Use an explicit date so that reformatting
                                % doesn't cause a new date to be used.  Setting
                                % the date to \today can be used during draft
                                % stages to make it easier to handle versions.
\release{0.9}                   % release version; this is used to define the
\setshortversion{0.9}           % \version macro
\makeindex                      % tell \index to actually write the .idx file


\begin{document}

\maketitle

% This makes the contents more accessible from the front page of the HTML.
\ifhtml
\chapter*{Front Matter}
\label{front}
\fi

Copyright \copyright{} 2002,2005 The pygsl Team.

Permission is granted to copy, distribute and/or modify this document under the
terms of the GNU Free Documentation License, Version 1.1 or any later version
published by the Free Software Foundation; with no Invariant Sections, no
Front-Cover Texts, and no Back-Cover Texts.  A copy of the license is included
in section \ref{cha:free-documentation-license} entitled ``GNU Free
Documentation License''.


%% Local Variables:
%% mode: LaTeX
%% mode: auto-fill
%% fill-column: 79
%% indent-tabs-mode: nil
%% ispell-dictionary: "american"
%% reftex-fref-is-default: nil
%% TeX-auto-save: t
%% TeX-command-default: "pdfeLaTeX"
%% TeX-master: "pygsl"
%% TeX-parse-self: t
%% End:


\begin{abstract}
   \noindent
   pygsl grants python users access to the GNU scientific library.  The latest
   version can be found at the project homepage, \url{http://pygsl.sf.net}.

   \textbf{Implemented features:} \\
   \begin{tabular}{ll}
     \module{pygsl.blas}                & basic linear algebra system\\
     \module{pygsl.chebyshev}           & chebyshev approximations\\
     \module{pygsl.combination}         & combinations  \\
     \module{pygsl.const}               & $>200$ often used mathematical and
                                          scientific constants. \\
     \module{pygsl.diff}                & (Deprecated. Use pygsl.deriv instead). \\
     \module{pygsl.deriv}               & Numerical differentiation. \\
     \module{pygsl.eigen}               &\\
     \module{pygsl.fit}                 &\\
     \module{pygsl.histogram}          & 1d and 2d histograms and operations
                                          on histograms. \\
     \module{pygsl.ieee}                & Access to the ieee-arithmetics layer
                                          of gsl. \\ 
     \module{pygsl.integrate}           &\\
     \module{pygsl.interpolation}       &\\ 
     \module{pygsl.linalg}              &\\
     \module{pygsl.math}                &\\
     \module{pygsl.monte}               &\\
     \module{pygsl.minimize}            &\\
     \module{pygsl.multifit}            &\\
     \module{pygsl.multifit_nlin}       &\\    
     \module{pygsl.multimin}            &\\
     \module{pygsl.multiroots}          &\\ 
     \module{pygsl.odeiv}               &\\
     \module{pygsl.permutation}         &\\  
     \module{pygsl.poly}                &\\
     \module{pygsl.qrng}                &\\     
     \module{pygsl.rng}                 & random number generators and probability densities. \\
     \module{pygsl.roots}               &\\
     \module{pygsl.siman}               &Simulated anealing\\
     \module{pygsl.sum}                 & \\
     \module{pygsl.sf}                  & $>200$ special functions. \\
     \module{pygsl.statistics}          & Statistical functions. \\
   \end{tabular}
\end{abstract}


\tableofcontents


\chapter{System Requirements, Installation}
\label{cha:system-req-installation}
\section{Status}

\paragraph*{Status of GSL-Library}
The gsl-library is since version 1.0 stable and for general use.
More information about it at \url{http://www.gnu.org/software/gsl/}.

\paragraph*{Status of this interface}
Nearly all modules are wrapped. A lot of tests are
covering various functionality. Please report to the mailing list
\url{pygsl-discuss@lists.sourceforge.net} if you find a bug.

The hankel modules have been
wrapped. Please write to the mailing list
\url{pygsl-discuss@lists.sourceforge.net} 
if you require one of the modules
and are willing to help with a simple example. 
If any other function is missing or some other module (e.g. ntuple) or
function, do not hesitate to write to the list.

\paragraph*{Retriving the Interface}
You can download it here: \url{http://sourceforge.net/projects/pygsl}

\section{Requirements}

To build the interface, you will need
\begin{itemize}
\item \ulink{gsl-1.x}{http://sources.redhat.com/gsl},
\item \ulink{python2.6}{http://www.python.org} or better,
\item \ulink{NumPy}{http://numpy.sf.net}, and
\item a c compiler (like \ulink{gcc}{http://gcc.gnu.org}).
\end{itemize}

Supported Platforms are:
\begin{itemize}
\item Linux (Redhat/Debian/SuSE) with python2.* and gsl-1.*
\item Win32
\end{itemize}
It was tested and is tested on an irregular basis on the following platforms
\begin{itemize}
\item SUN
\item Cygwin
\item MacOS X
\end{itemize}
but is supposed to build on any POSIX platforms.

\section{Installing the pygsl interface}

\program{gsl-config} must be on your path:\nopagebreak
\begin{verbatim}
# unpack the source distribution
gzip -d -c pygsl-x.y.z.tar.gz|tar xvf-
cd pygsl-x.y.z
# do this with your prefered python version
# to set the gsl location explicitly use setup.py --gsl-prefix=/path/to/gsl
python setup.py build
# change to an user id, that is allowed to do installation
python setup.py install
\end{verbatim}
Ready....

{\bf Do not test the interface in the distribution root or in the directories
 \file{src} or \file{pygsl}.}

If you find unresolved symbols later on, delete the C source in the
swig_src files. Check that swig can be called from the command line. 
Then start the build process again. 

In this case swig will rebuild the C files. The swig_src files
distributed with pygsl are to an up to date version of GSL (1.16 as of
this writing). Swig parses partly some header header files and builds
the appropriate interface functions. If you have an older GSL version 
locally installed, the sources in the swig_src directory can contain 
links to symbols which are not defined by the locally installed GSL
version.

\subsection{Building on win32}

Windows by default does not allow to run a posix shell. Here a different path
is required. First change into the directory \file{gsl_dist}. Copy the file 
\file{gsl_site_example.py}
and edit it to reflect your installation of GSL and SWIG if you want to run it
yourself. The pygsl windows binaries distributed over 
\url{http://sourceforge.net/projects/pygsl/} are built using the mingw32 
compiler. 

\paragraph*{Uninstall GSL interface}
\code{rm -r }"python install path"\code{/lib/python}"version"\code{/site-packages/pygsl}

\paragraph*{Testing}
the directory \file{tests} contains several testsuites, based on python
\module{unittest}.
The script \file{run_test.py} in this directory will run one after the other.

\paragraph*{Support}
Please send mails to our mailinglist at
\email{pygsl-discuss@lists.sourceforge.net}.

\paragraph*{Developement}
You can browse our cvs tree at
\url{http://cvs.sourceforge.net/cgi-bin/viewcvs.cgi/pygsl/pygsl/}.
\\
Type this to check out the actual version:
\begin{verbatim}
cvs -d:pserver:anonymous@cvs.pygsl.sourceforge.net:/cvsroot/pygsl login
#Hit return for no password.
cvs -z3 -d:pserver:anonymous@cvs.pygsl.sourceforge.net:/cvsroot/pygsl co pygsl
\end{verbatim}
The script \program{tools/extract_tool.py} generates most of the special 
function code.

%\input{install_advanced.tex}
\paragraph*{ToDo}
Implement other parts:


\paragraph*{History}
\begin{itemize}
\item a gsl-interface for python was needed for a project at
\ulink{Center for Applied Informatics Cologne}{http://www.zaik.uni-koeln.de/AFS}.
\item \file{gsl-0.0.3} was released at May 23, 2001
\item \file{gsl-0.0.4} was released at January 8, 2002
\item \file{gsl-0.0.5} is growing since January, 2002
\item \file{gsl-0.2.0} was released at 
\item \file{gsl-0.3.0} was released at 
\item \file{gsl-0.3.1} was released at 
\item \file{gsl-0.3.2} was released at 
\item \file{gsl-0.9.4} was released at 25. October 2008
\end{itemize}

\paragraph*{Thanks}
Jochen K\"upper (\email{jochen@jochen-kuepper.de}) for 
\module{pygsl.statistics} part\\
Fabian Jakobs for \module{pygsl.blas}, \module{pygsl.eigen}
\module{pygsl.linalg}, \module{pygsl.permutation}\\ 
Leonardo Milano for rpm build\\
Eric Gurrola and  Peter Stoltz for testing and supporting the port of pygsl to
the MAC\\
Sebastien Maret for supporting the Fink \url{http://fink.sourceforge.net}
port of pygsl.


\paragraph*{Maintainers}
Achim G\"adke (\email{AchimGaedke@users.sourceforge.net}),\\
Pierre Schnizer (\email{schnizer@users.sourceforge.net})

\input{installadvanced.tex}
\chapter{Design of the \pygsl{} interface}

The GSL library was implemented using the C language. This implies that 
each function uses a certain type for the different variables and are fixed
 to one specific type. The wrapper will try to convert each argument to the approbriate
C type. 
The \pygsl{} interface
tries to follow it as much as possible but only as far as resonable. 
For example the definition of the poly_eval function in C is given by
\begin{funcdesc}{\texttt{double} gsl_poly_eval}
                {\texttt{const double} C[], \texttt{const int} LEN, \texttt{const double} X}
\end{funcdesc}

The corresponding python wrapper was implemented by
\begin{funcdesc}{poly.poly_eval}{C, x}
\end{funcdesc}
as the wrapper can get the length of any python object and then fill the len variable. 
The mathematical calculation is performed by the GSL library. Thus the calculation is limited 
to the precision provided by the underlying hardware.

Default arguments are used to allocate workspaces on the fly if not provided by the user. 
Consider for example the fft module. The function for the real forward transform is
named 

\begin{funcdesc}{\texttt{int} gsl_fft_real_transform}
{\texttt{double DATA[]}, 
 \texttt{size_t STRIDE},
 \texttt{size_t N}, 
 \texttt{const gsl_fft_real_wavetable * WAVETABLE},
 \texttt{gsl_fft_real_workspace * WORK}
}
\end{funcdesc}

The corresponding python wrapper is found in the fft module called
real_transform
\begin{funcdesc}{real_transform}
{data, \optional{space, 
    table, 
    output}}
\end{funcdesc}
The wrapper will get the stride and size information from the data object provided
by the user. If space or table are not provided, these objects will be generated on 
the fly. As the GSL function applies the transformation in space, an internal copy is 
made of the data and only then the object is passed to the \gsl{} function. If an output
object is provided the data will be copied there instead. \pygsl{} will always make copies
of objects which would be otherwise modified in place.

\section{Callbacks}

Solvers require as one argument a user function to work on which have to be provided by the
user. These callbacks typically are of the form
\begin{funcdesc}{f}{x, params}
  \dots\\
  return result
\end{funcdesc}
Please note that this function must return the exact number of arguments
as given in the example. The wrappers around callbacks go a long way to try to provide
meaningfull error messages. If a solver fails, please check that the number of input and 
output arguments it takes are correct

\section{Error handling}
\label{sec:interface-error-handling}
As GSL is a C library error handling is implemented using an error handler and return values.
\pygsl{} generates python exceptions out of these values. See \module{pygsl.errors} 
(chapter~\ref{cha:error-module}) for a list of the exceptions.

\section{Exception handling}
\index{exception handling!initialisation} GSL provides a selectable error
handler, that is called for occuring errors (like domain errors, division by
zero, etc. ).  This is switched off. Instead each wrapper function will check
the error return value and in case of error an python exception is created. 

Here is a python level example:
\begin{verbatim}
import pygsl.histogram
import pygsl.errors
hist=pygsl.histogram.histogram2d(100,100)
try:
   hist[-1,-1]=0
except pygsl.errors.gsl_Error,err:
   print err
\end{verbatim}
Will result
\begin{verbatim}
input domain error: index i lies outside valid range of 0 .. nx - 1
\end{verbatim}


An exception are ufuncs in the testings.sf module (see section\ref{sec:ufuncs}).


\subsection{Change of internal error handling.}
Before a error handler was installed by init_pygsl into gsl which translated
the error code (and the message) to a python exception.
This required that the GIL was available, which numpy ufuncs dispose. Thus
now this gsl error handler is deactivated and instead the C error code
returned by the C function is translated to an error code by the wrapper
called from python.

UFuncs do not call this handler now at all.


\section{The documentation gap}

\pygsl{} does still lack an approbriate documentation. Most documentation is accessible over
the internal documentation strings. These are accessible as \code{__doc__} attributes (the help
function does not always show them).  It can be sometimes necessary to create an 
object to see its methods as well as the documentation of the methods
 (e.g.a random number generator in the rng module to see its methods). 
The \file{example} directory contains examples for (nearly each) module.

Please feel welcome to add to the documentation!


\paragraph*{Acknowledgment}
\label{sec:acknowledgment}
Parts of this this manual are based on the \GSL{} reference manual.
The authors want to thank all for contribution of code,
support material for generating distribution packages, bug reports
and example scripts.


\chapter[\protect\module{pygsl.errors} --- Error and warning classes]
{\protect\module{pygsl.errors} \\ Error and warning classes} 
\label{cha:error-module}
\declaremodule{standard}{pygsl.errors}
\moduleauthor{Pierre Schnizer}{schnizer@users.sourceforge.net}
\moduleauthor{Original Author: Achim G\"adke}{achimgaedke@users.sourceforge.net}

This chapter provides information about the \exception{gsl_Error} exception class that comes with this module.

\section{Exception Classes}


\begin{excclassdesc} {gsl_Error}{}
derived from \exception{Exception}, can be constructed with any object as parameter.
It is baseclass to all other \gsl{} Exceptions
\end{excclassdesc}
These classes are translations of the \file{<gsl/gsl_errno.h>} to python
exceptions.


\begin{excclassdesc}{gsl_ArithmeticError}{}
derived from \exception{gsl_Error} and \exception{exceptions.ArithmeticError},
base of all common arithmetic exceptions
\end{excclassdesc}

\begin{excclassdesc}{gsl_OverflowError}{}
derived from \exception{gsl_Error} and \exception{exceptions.OverflowError}
\end{excclassdesc}

\begin{excclassdesc}{gsl_ZeroDivisionError}{}
derived from \exception{gsl_Error} and \exception{exceptions.ZeroDivisionError}
\end{excclassdesc}

\begin{excclassdesc}{gsl_FloatingPointError}{}
derived from \exception{gsl_Error} and \exception{exceptions.FloatingPointError}
\end{excclassdesc}

\begin{excclassdesc}{gsl_ArithmeticError}{}
is derived from  \exception{gsl_Error} and from  \exception{ArithmeticError} .
This exception is the    base of all common arithmetic exceptions.
\end{excclassdesc}

\begin{excclassdesc}{gsl_AccuracyLossError}{}
is derived from  \exception{gsl_ArithmeticError} .
This exception is raised if the failed to reach the specified tolerance.
\end{excclassdesc}
\begin{excclassdesc}{gsl_BadFuncError}{}
is derived from  \exception{gsl_Error} .
This exception is raised if problem with a user-supplied function occur.
\end{excclassdesc}
\begin{excclassdesc}{gsl_BadLength}{}
is derived from  \exception{gsl_Error} .
This exception is raised if  matrix or  vector lengths are not conformant.
\end{excclassdesc}
\begin{excclassdesc}{gsl_BadToleranceError}{}
is derived from  \exception{gsl_Error} .
This exception is raised if user specified an tolerance which can not be reached.
\end{excclassdesc}
\begin{excclassdesc}{gsl_CacheLimitError}{}
is derived from  \exception{gsl_Error} .
This exception is raised if the    cache limit is exceeded.
\end{excclassdesc}
\begin{excclassdesc}{gsl_DivergeError}{}
is derived from  \exception{gsl_ArithmeticError} .
This exception is raised if an   integral or series is divergent.
\end{excclassdesc}
\begin{excclassdesc}{gsl_DomainError}{}
is derived from  \exception{gsl_Error} .
This exception is raised if    domain errors occure. e.g. sqrt(-1).
\end{excclassdesc}
\begin{excclassdesc}{gsl_EOFError}{}
is derived from  \exception{gsl_Error} and from  \exception{EOFError} .
This exception is raised if 
    end of file
     .
\end{excclassdesc}
\begin{excclassdesc}{gsl_FactorizationError}{}
is derived from  \exception{gsl_Error} .
This exception is raised if     factorization failed.
\end{excclassdesc}
\begin{excclassdesc}{gsl_FloatingPointError}{}
is derived from  \exception{gsl_Error} and from  \exception{FloatingPointError} .
\end{excclassdesc}
\begin{excclassdesc}{gsl_GenericError}{}
is derived from  \exception{gsl_Error} .
\end{excclassdesc}
\begin{excclassdesc}{gsl_InvalidArgumentError}{}
is derived from  \exception{gsl_Error} .
This exception is raised if an invalid argument is supplied by the user.
\end{excclassdesc}
\begin{excclassdesc}{gsl_JacobianEvaluationError}{}
is derived from  \exception{gsl_ArithmeticError} .
This exception is raised if jacobian evaluations are not improving the solution.
\end{excclassdesc}
\begin{excclassdesc}{gsl_MatrixNotSquare}{}
is derived from  \exception{gsl_Error} .
This exception is raised if the given matrix is not square.
\end{excclassdesc}
\begin{excclassdesc}{gsl_MaximumIterationError}{}
is derived from  \exception{gsl_ArithmeticError} .
This exception is raised if    the maximum number  of iterations is exceeded.
\end{excclassdesc}
\begin{excclassdesc}{gsl_NoHardwareSupportError}{}
is derived from  \exception{gsl_Error} .
This exception is raised if the requested feature is not supported by the hardware.
\end{excclassdesc}
\begin{excclassdesc}{gsl_NoProgressError}{}
is derived from  \exception{gsl_ArithmeticError} .
This exception is raised if the  iteration is not making progress towards solution.
\end{excclassdesc}
\begin{excclassdesc}{gsl_NotImplementedError}{}
is derived from  \exception{gsl_Error} and from  \exception{NotImplementedError} .
This exception is raised if  a requested feature is not (yet) implemented .
\end{excclassdesc}
\begin{excclassdesc}{gsl_OverflowError}{}
is derived from  \exception{gsl_Error} and from  \exception{OverflowError} .
\end{excclassdesc}
\begin{excclassdesc}{gsl_PointerError}{}
is derived from  \exception{gsl_Error} .
This exception is raised if an invalid pointer is found by the C wrapper code
or by the GSL library.
\end{excclassdesc}
\begin{excclassdesc}{gsl_RangeError}{}
is derived from  \exception{gsl_ArithmeticError} .
This exception is raised if     output would be out or range, e.g. exp(1e100)
     .
\end{excclassdesc}
\begin{excclassdesc}{gsl_RoundOffError}{}
is derived from  \exception{gsl_ArithmeticError} .
This exception is raised if  arithmetic failed because of roundoff error.
\end{excclassdesc}
\begin{excclassdesc}{gsl_RunAwayError}{}
is derived from  \exception{gsl_ArithmeticError} .
This exception is raised if   iterative process is out of control.
\end{excclassdesc}
\begin{excclassdesc}{gsl_SanityCheckError}{}
is derived from  \exception{gsl_Error} .
This exception is raised if a sanity check failed - shouldn't happen.
\end{excclassdesc}
\begin{excclassdesc}{gsl_SingularityError}{}
is derived from  \exception{gsl_ArithmeticError} .
This exception is raised if  an   apparent singularity is detected.
\end{excclassdesc}
\begin{excclassdesc}{gsl_TableLimitError}{}
is derived from  \exception{gsl_Error} .
This exception is raised if the table limit is exceeded.
\end{excclassdesc}
\begin{excclassdesc}{gsl_ToleranceError}{}
is derived from  \exception{gsl_ArithmeticError} .
This exception is raised if  the alghorithm failed to reach the specified tolerance.
\end{excclassdesc}
\begin{excclassdesc}{gsl_ToleranceFError}{}
is derived from  \exception{gsl_ArithmeticError} .
This exception is raised if  the alghorithm cannot reach the specified
tolerance in F (typically the variation of the evaluated function).
\end{excclassdesc}
\begin{excclassdesc}{gsl_ToleranceGradientError}{}
is derived from  \exception{gsl_ArithmeticError} .
This exception is raised if  cannot reach the specified tolerance for the gradient.
\end{excclassdesc}
\begin{excclassdesc}{gsl_ToleranceXError}{}
is derived from  \exception{gsl_ArithmeticError} .
This exception is raised if cannot reach the specified tolerance in X
(typically a search result).
\end{excclassdesc}
\begin{excclassdesc}{gsl_UnderflowError}{}
is derived from  \exception{gsl_Error} and from  \exception{OverflowError} .
\end{excclassdesc}
\begin{excclassdesc}{gsl_ZeroDivisionError}{}
is derived from  \exception{gsl_Error} and from  \exception{ZeroDivisionError} .
\end{excclassdesc}

All the above errors are just translations of the errno to python exceptions.

The following two are specific to pygsl:
\begin{excclassdesc}{pygsl.errors.pygsl_NotImplementedError}{}
is derived from  \exception{gsl_Error} and from  \exception{NotImplementedError} .
This exception is raised if a feature is requested but not
implemented. Currently only used if a module requests the debugging enviroment
of the init module, but the init module was not compiled with \code{\#define DEBUG=1}
\end{excclassdesc}
\begin{excclassdesc}{pygsl.errors.pygsl_StrideError}{}
is derived from  \exception{gsl_SanityCheckError} .
GSL uses as strides multiples of the basis type; for a vector or doubles, one
means from one double to the next. Numpy or numarray count the stride in
multiples of the size of a char. Therefore the stride has to be recalculated
before the approbriate \gsl{} function can be called. If that fails this
exception is raised.
\end{excclassdesc}

\section{Warning Classes}

\begin{excclassdesc} {gsl_Warning}{}
The dedicated warning class for \gsl{} has \exception{Warning} as base class.
\end{excclassdesc}

\begin{excclassdesc}{gsl_DomainWarning}{}
derived from \exception{gsl_Warning}, used by some \module{pygsl.histogram} functions
\end{excclassdesc}


\chapter[\protect\module{pygsl.const} --- Mathematical and scientific
constants]{\protect\module{pygsl.const} \\ Mathematical and scientific
constants} 
\label{cha:const-module}
\declaremodule{extension}{pygsl.const} 
\moduleauthor{Achim  G\"adke}{achimgaedke@users.sourceforge.net}

In this module some usefull constants are defined.  There are four groups of
constants:

\begin{itemize}
\item mathematical,
\item physical in mks unit system,
\item physical in cgs unit system and
\item physical number constants (e.g. fine structure)
\end{itemize}

The other modules are created during the initialisation of
\module{pygsl.const}.  For convenience the mathematical, physical mks
constants and number constants also are available in the namespace of
\module{pygsl.const}.  If the used GSL version is before gsl1.4, see
\begin{verbatim}
pygsl.compiled_gsl_version
\end{verbatim}
than the module names are cgs and mks. Form gsl1.5 these modules have been
renamed to cgsm and mksa. So to use cgs constants one has to write
\begin{verbatim}
import pygsl.const
import pygsl.const.cgs
print pygsl.const.cgs.speed_of_light/pygsl.const.speed_of_light
\end{verbatim}
for gsl $<$ 1.5 and
\begin{verbatim}
import pygsl.const
import pygsl.const.cgsm
print pygsl.const.cgsm.speed_of_light/pygsl.const.speed_of_light
\end{verbatim}.
Of course the result is \constant{100.0}.
Short examples are given at top of each section.

\begin{seealso}
  The actual values are taken form the \gsl{} headers.  The \GSL{} reference
  provides a more detailed description of these constants.
\end{seealso}

\section[\protect\module{pygsl.const.m} --- Mathematical constants]
{\protect\module{pygsl.const.m} \\ Mathematical constants} 
\label{cha:const-math-module}

\begin{verbatim}
from pygsl.const.m import *
print sqrt2*sqrt2
\end{verbatim}\\
Prints \constant{2.0}.\\
 Here comes the list:\nopagebreak
\begin{longtableiii}{l|l|l}{constant}{Name}{\gsl{} Name}{value}
\lineiii{e}{\protect\constant{M\_E}}{e}
\lineiii{log2e}{\constant{M\_LOG2E}}{$\log_2 e$}
\lineiii{log10e}{\constant{M\_LOG10E}}{$\log_{10} e$}
\lineiii{sqrt2}{\constant{M\_SQRT2}}{$\sqrt{2}$}
\lineiii{sqrt1\_2}{\constant{M\_SQRT1\_2}}{$\sqrt{1/2}$}
\lineiii{sqrt3}{\constant{M\_SQRT3}}{$\sqrt{3}$}
\lineiii{pi}{\constant{M\_PI}}{$\pi$}
\lineiii{pi\_2}{\constant{M\_PI\_2}}{$\pi/2$}
\lineiii{pi\_4}{\constant{M\_PI\_4}}{$\pi/4$}
\lineiii{sqrtpi}{\constant{M\_SQRTPI}}{$\sqrt{\pi}$}
\lineiii{2\_sqrtpi}{\constant{M\_2\_SQRTPI}}{$2/\sqrt{\pi}$}
\lineiii{1\_pi}{\constant{M\_1\_PI}}{$1/\pi$}
\lineiii{2\_pi}{\constant{M\_2\_PI}}{$2/\pi$}
\lineiii{ln10}{\constant{M\_LN10}}{$\ln 10$}
\lineiii{ln2}{\constant{M\_LN2}}{$\ln 2$}
\lineiii{lnpi}{\constant{M\_LNPI}}{$\ln{\pi}$}
\lineiii{euler}{\constant{M\_EULER}}{Euler constant}
\end{longtableiii}

\section[\protect\module{pygsl.const.mksa} --- Scientific constants in mksa units]
{\protect\module{pygsl.const.mksa} \\ Scientific constants in mksa units} 
\label{cha:const-mks-module}

\begin{verbatim}
from pygsl.const import cgsm
print "a teaspoon contains %g m^3"%mks.teaspoon
\end{verbatim}

These are the provided constants:\nopagebreak
\begin{longtableiii}{l|l|l}{constant}{Name}{gsl Name}{unit}
\lineiii{speed\_of\_light}{\constant{GSL\_CONST\_MKSA\_SPEED\_OF\_LIGHT}}{m / s}
\lineiii{gravitational\_constant}{\constant{GSL\_CONST\_MKSA\_GRAVITATIONAL\_CONSTANT}}{m\^{}3 / kg s\^{}2}
\lineiii{plancks\_constant\_h}{\constant{GSL\_CONST\_MKSA\_PLANCKS\_CONSTANT\_H}}{kg m\^{}2 / s}
\lineiii{plancks\_constant\_hbar}{\constant{GSL\_CONST\_MKSA\_PLANCKS\_CONSTANT\_HBAR}}{kg m\^{}2 / s}
\lineiii{vacuum\_permeability}{\constant{GSL\_CONST\_MKSA\_VACUUM\_PERMEABILITY}}{kg m / A\^{}2 s\^{}2}
\lineiii{astronomical\_unit}{\constant{GSL\_CONST\_MKSA\_ASTRONOMICAL\_UNIT}}{m}
\lineiii{light\_year}{\constant{GSL\_CONST\_MKSA\_LIGHT\_YEAR}}{m}
\lineiii{parsec}{\constant{GSL\_CONST\_MKSA\_PARSEC}}{m}
\lineiii{grav\_accel}{\constant{GSL\_CONST\_MKSA\_GRAV\_ACCEL}}{m / s\^{}2}
\lineiii{electron\_volt}{\constant{GSL\_CONST\_MKSA\_ELECTRON\_VOLT}}{kg m\^{}2 / s\^{}2}
\lineiii{mass\_electron}{\constant{GSL\_CONST\_MKSA\_MASS\_ELECTRON}}{kg}
\lineiii{mass\_muon}{\constant{GSL\_CONST\_MKSA\_MASS\_MUON}}{kg}
\lineiii{mass\_proton}{\constant{GSL\_CONST\_MKSA\_MASS\_PROTON}}{kg}
\lineiii{mass\_neutron}{\constant{GSL\_CONST\_MKSA\_MASS\_NEUTRON}}{kg}
\lineiii{rydberg}{\constant{GSL\_CONST\_MKSA\_RYDBERG}}{kg m\^{}2 / s\^{}2}
\lineiii{boltzmann}{\constant{GSL\_CONST\_MKSA\_BOLTZMANN}}{kg m\^{}2 / K s\^{}2}
\lineiii{bohr\_magneton}{\constant{GSL\_CONST\_MKSA\_BOHR\_MAGNETON}}{A m\^{}2}
\lineiii{nuclear\_magneton}{\constant{GSL\_CONST\_MKSA\_NUCLEAR\_MAGNETON}}{A m\^{}2}
\lineiii{electron\_magnetic\_moment}{\constant{GSL\_CONST\_MKSA\_ELECTRON\_MAGNETIC\_MOMENT}}{A m\^{}2}
\lineiii{proton\_magnetic\_moment}{\constant{GSL\_CONST\_MKSA\_PROTON\_MAGNETIC\_MOMENT}}{A m\^{}2}
\lineiii{molar\_gas}{\constant{GSL\_CONST\_MKSA\_MOLAR\_GAS}}{kg m\^{}2 / K mol s\^{}2}
\lineiii{standard\_gas\_volume}{\constant{GSL\_CONST\_MKSA\_STANDARD\_GAS\_VOLUME}}{m\^{}3 / mol}
\lineiii{minute}{\constant{GSL\_CONST\_MKSA\_MINUTE}}{s}
\lineiii{hour}{\constant{GSL\_CONST\_MKSA\_HOUR}}{s}
\lineiii{day}{\constant{GSL\_CONST\_MKSA\_DAY}}{s}
\lineiii{week}{\constant{GSL\_CONST\_MKSA\_WEEK}}{s}
\lineiii{inch}{\constant{GSL\_CONST\_MKSA\_INCH}}{m}
\lineiii{foot}{\constant{GSL\_CONST\_MKSA\_FOOT}}{m}
\lineiii{yard}{\constant{GSL\_CONST\_MKSA\_YARD}}{m}
\lineiii{mile}{\constant{GSL\_CONST\_MKSA\_MILE}}{m}
\lineiii{nautical\_mile}{\constant{GSL\_CONST\_MKSA\_NAUTICAL\_MILE}}{m}
\lineiii{fathom}{\constant{GSL\_CONST\_MKSA\_FATHOM}}{m}
\lineiii{mil}{\constant{GSL\_CONST\_MKSA\_MIL}}{m}
\lineiii{point}{\constant{GSL\_CONST\_MKSA\_POINT}}{m}
\lineiii{texpoint}{\constant{GSL\_CONST\_MKSA\_TEXPOINT}}{m}
\lineiii{micron}{\constant{GSL\_CONST\_MKSA\_MICRON}}{m}
\lineiii{angstrom}{\constant{GSL\_CONST\_MKSA\_ANGSTROM}}{m}
\lineiii{hectare}{\constant{GSL\_CONST\_MKSA\_HECTARE}}{m\^{}2}
\lineiii{acre}{\constant{GSL\_CONST\_MKSA\_ACRE}}{m\^{}2}
\lineiii{barn}{\constant{GSL\_CONST\_MKSA\_BARN}}{m\^{}2}
\lineiii{liter}{\constant{GSL\_CONST\_MKSA\_LITER}}{m\^{}3}
\lineiii{us\_gallon}{\constant{GSL\_CONST\_MKSA\_US\_GALLON}}{m\^{}3}
\lineiii{quart}{\constant{GSL\_CONST\_MKSA\_QUART}}{m\^{}3}
\lineiii{pint}{\constant{GSL\_CONST\_MKSA\_PINT}}{m\^{}3}
\lineiii{cup}{\constant{GSL\_CONST\_MKSA\_CUP}}{m\^{}3}
\lineiii{fluid\_ounce}{\constant{GSL\_CONST\_MKSA\_FLUID\_OUNCE}}{m\^{}3}
\lineiii{tablespoon}{\constant{GSL\_CONST\_MKSA\_TABLESPOON}}{m\^{}3}
\lineiii{teaspoon}{\constant{GSL\_CONST\_MKSA\_TEASPOON}}{m\^{}3}
\lineiii{canadian\_gallon}{\constant{GSL\_CONST\_MKSA\_CANADIAN\_GALLON}}{m\^{}3}
\lineiii{uk\_gallon}{\constant{GSL\_CONST\_MKSA\_UK\_GALLON}}{m\^{}3}
\lineiii{miles\_per\_hour}{\constant{GSL\_CONST\_MKSA\_MILES\_PER\_HOUR}}{m / s}
\lineiii{kilometers\_per\_hour}{\constant{GSL\_CONST\_MKSA\_KILOMETERS\_PER\_HOUR}}{m / s}
\lineiii{knot}{\constant{GSL\_CONST\_MKSA\_KNOT}}{m / s}
\lineiii{pound\_mass}{\constant{GSL\_CONST\_MKSA\_POUND\_MASS}}{kg}
\lineiii{ounce\_mass}{\constant{GSL\_CONST\_MKSA\_OUNCE\_MASS}}{kg}
\lineiii{ton}{\constant{GSL\_CONST\_MKSA\_TON}}{kg}
\lineiii{metric\_ton}{\constant{GSL\_CONST\_MKSA\_METRIC\_TON}}{kg}
\lineiii{uk\_ton}{\constant{GSL\_CONST\_MKSA\_UK\_TON}}{kg}
\lineiii{troy\_ounce}{\constant{GSL\_CONST\_MKSA\_TROY\_OUNCE}}{kg}
\lineiii{carat}{\constant{GSL\_CONST\_MKSA\_CARAT}}{kg}
\lineiii{unified\_atomic\_mass}{\constant{GSL\_CONST\_MKSA\_UNIFIED\_ATOMIC\_MASS}}{kg}
\lineiii{gram\_force}{\constant{GSL\_CONST\_MKSA\_GRAM\_FORCE}}{kg m / s\^{}2}
\lineiii{pound\_force}{\constant{GSL\_CONST\_MKSA\_POUND\_FORCE}}{kg m / s\^{}2}
\lineiii{kilopound\_force}{\constant{GSL\_CONST\_MKSA\_KILOPOUND\_FORCE}}{kg m / s\^{}2}
\lineiii{poundal}{\constant{GSL\_CONST\_MKSA\_POUNDAL}}{kg m / s\^{}2}
\lineiii{calorie}{\constant{GSL\_CONST\_MKSA\_CALORIE}}{kg m\^{}2 / s\^{}2}
\lineiii{btu}{\constant{GSL\_CONST\_MKSA\_BTU}}{kg m\^{}2 / s\^{}2}
\lineiii{therm}{\constant{GSL\_CONST\_MKSA\_THERM}}{kg m\^{}2 / s\^{}2}
\lineiii{horsepower}{\constant{GSL\_CONST\_MKSA\_HORSEPOWER}}{kg m\^{}2 / s\^{}3}
\lineiii{bar}{\constant{GSL\_CONST\_MKSA\_BAR}}{kg / m s\^{}2}
\lineiii{std\_atmosphere}{\constant{GSL\_CONST\_MKSA\_STD\_ATMOSPHERE}}{kg / m s\^{}2}
\lineiii{torr}{\constant{GSL\_CONST\_MKSA\_TORR}}{kg / m s\^{}2}
\lineiii{meter\_of\_mercury}{\constant{GSL\_CONST\_MKSA\_METER\_OF\_MERCURY}}{kg / m s\^{}2}
\lineiii{inch\_of\_mercury}{\constant{GSL\_CONST\_MKSA\_INCH\_OF\_MERCURY}}{kg / m s\^{}2}
\lineiii{inch\_of\_water}{\constant{GSL\_CONST\_MKSA\_INCH\_OF\_WATER}}{kg / m s\^{}2}
\lineiii{psi}{\constant{GSL\_CONST\_MKSA\_PSI}}{kg / m s\^{}2}
\lineiii{poise}{\constant{GSL\_CONST\_MKSA\_POISE}}{kg / m / s}
\lineiii{stokes}{\constant{GSL\_CONST\_MKSA\_STOKES}}{m\^{}2 / s}
\lineiii{faraday}{\constant{GSL\_CONST\_MKSA\_FARADAY}}{A s / mol}
\lineiii{electron\_charge}{\constant{GSL\_CONST\_MKSA\_ELECTRON\_CHARGE}}{A s}
\lineiii{gauss}{\constant{GSL\_CONST\_MKSA\_GAUSS}}{kg / A s\^{}2}
\lineiii{stilb}{\constant{GSL\_CONST\_MKSA\_STILB}}{cd / m\^{}2}
\lineiii{lumen}{\constant{GSL\_CONST\_MKSA\_LUMEN}}{cd sr}
\lineiii{lux}{\constant{GSL\_CONST\_MKSA\_LUX}}{cd sr / m\^{}2}
\lineiii{phot}{\constant{GSL\_CONST\_MKSA\_PHOT}}{cd sr / m\^{}2}
\lineiii{footcandle}{\constant{GSL\_CONST\_MKSA\_FOOTCANDLE}}{cd sr / m\^{}2}
\lineiii{lambert}{\constant{GSL\_CONST\_MKSA\_LAMBERT}}{cd sr / m\^{}2}
\lineiii{footlambert}{\constant{GSL\_CONST\_MKSA\_FOOTLAMBERT}}{cd sr / m\^{}2}
\lineiii{curie}{\constant{GSL\_CONST\_MKSA\_CURIE}}{1 / s}
\lineiii{roentgen}{\constant{GSL\_CONST\_MKSA\_ROENTGEN}}{A s / kg}
\lineiii{rad}{\constant{GSL\_CONST\_MKSA\_RAD}}{m\^{}2 / s\^{}2}
\lineiii{solar\_mass}{\constant{GSL\_CONST\_MKSA\_SOLAR\_MASS}}{kg}
\lineiii{bohr\_radius}{\constant{GSL\_CONST\_MKSA\_BOHR\_RADIUS}}{m}
\lineiii{vacuum\_permittivity}{\constant{GSL\_CONST\_MKSA\_VACUUM\_PERMITTIVITY}}{A\^{}2 s\^{}4 / kg m\^{}3}
\end{longtableiii}

\section[\protect\module{pygsl.const.cgsm} --- Scientific constants in cgsm units]
{\protect\module{pygsl.const.cgsm} \\ Scientific constants in cgsm units} 
\label{cha:const-cgs-module}

\begin{verbatim}
from pygsl.const import cgsm
print "a teaspoon contains %g ml"%cgs.teaspoon
\end{verbatim}

You can access the following constants:\nopagebreak
\begin{longtableiii}{l|l|l}{constant}{Name}{gsl Name}{unit or value}
\lineiii{speed\_of\_light}{\constant{GSL\_CONST\_CGSM\_SPEED\_OF\_LIGHT}}{cm / s}
\lineiii{gravitational\_constant}{\constant{GSL\_CONST\_CGSM\_GRAVITATIONAL\_CONSTANT}}{cm\^{}3 / g s\^{} 2}
\lineiii{plancks\_constant\_h}{\constant{GSL\_CONST\_CGSM\_PLANCKS\_CONSTANT\_H}}{g cm\^{}2 / s}
\lineiii{plancks\_constant\_hbar}{\constant{GSL\_CONST\_CGSM\_PLANCKS\_CONSTANT\_HBAR}}{g cm\^{}2 / s}
\lineiii{vacuum\_permeability}{\constant{GSL\_CONST\_CGSM\_VACUUM\_PERMEABILITY}}{cm g / A\^{}2 s\^{}2}
\lineiii{astronomical\_unit}{\constant{GSL\_CONST\_CGSM\_ASTRONOMICAL\_UNIT}}{cm}
\lineiii{light\_year}{\constant{GSL\_CONST\_CGSM\_LIGHT\_YEAR}}{cm}
\lineiii{parsec}{\constant{GSL\_CONST\_CGSM\_PARSEC}}{cm}
\lineiii{grav\_accel}{\constant{GSL\_CONST\_CGSM\_GRAV\_ACCEL}}{cm / s\^{}2}
\lineiii{electron\_volt}{\constant{GSL\_CONST\_CGSM\_ELECTRON\_VOLT}}{g cm\^{}2 / s\^{}2}
\lineiii{mass\_electron}{\constant{GSL\_CONST\_CGSM\_MASS\_ELECTRON}}{g}
\lineiii{mass\_muon}{\constant{GSL\_CONST\_CGSM\_MASS\_MUON}}{g}
\lineiii{mass\_proton}{\constant{GSL\_CONST\_CGSM\_MASS\_PROTON}}{g}
\lineiii{mass\_neutron}{\constant{GSL\_CONST\_CGSM\_MASS\_NEUTRON}}{g}
\lineiii{rydberg}{\constant{GSL\_CONST\_CGSM\_RYDBERG}}{g cm\^{}2 / s\^{}2}
\lineiii{boltzmann}{\constant{GSL\_CONST\_CGSM\_BOLTZMANN}}{g cm\^{}2 / K s\^{}2}
\lineiii{bohr\_magneton}{\constant{GSL\_CONST\_CGSM\_BOHR\_MAGNETON}}{A cm\^{}2}
\lineiii{nuclear\_magneton}{\constant{GSL\_CONST\_CGSM\_NUCLEAR\_MAGNETON}}{A cm\^{}2}
\lineiii{electron\_magnetic\_moment}{\constant{GSL\_CONST\_CGSM\_ELECTRON\_MAGNETIC\_MOMENT}}{A cm\^{}2}
\lineiii{proton\_magnetic\_moment}{\constant{GSL\_CONST\_CGSM\_PROTON\_MAGNETIC\_MOMENT}}{A cm\^{}2}
\lineiii{molar\_gas}{\constant{GSL\_CONST\_CGSM\_MOLAR\_GAS}}{g cm\^{}2 / K mol s\^{}2}
\lineiii{standard\_gas\_volume}{\constant{GSL\_CONST\_CGSM\_STANDARD\_GAS\_VOLUME}}{cm\^{}3 / mol}
\lineiii{minute}{\constant{GSL\_CONST\_CGSM\_MINUTE}}{s}
\lineiii{hour}{\constant{GSL\_CONST\_CGSM\_HOUR}}{s}
\lineiii{day}{\constant{GSL\_CONST\_CGSM\_DAY}}{s}
\lineiii{week}{\constant{GSL\_CONST\_CGSM\_WEEK}}{s}
\lineiii{inch}{\constant{GSL\_CONST\_CGSM\_INCH}}{cm}
\lineiii{foot}{\constant{GSL\_CONST\_CGSM\_FOOT}}{cm}
\lineiii{yard}{\constant{GSL\_CONST\_CGSM\_YARD}}{cm}
\lineiii{mile}{\constant{GSL\_CONST\_CGSM\_MILE}}{cm}
\lineiii{nautical\_mile}{\constant{GSL\_CONST\_CGSM\_NAUTICAL\_MILE}}{cm}
\lineiii{fathom}{\constant{GSL\_CONST\_CGSM\_FATHOM}}{cm}
\lineiii{mil}{\constant{GSL\_CONST\_CGSM\_MIL}}{cm}
\lineiii{point}{\constant{GSL\_CONST\_CGSM\_POINT}}{cm}
\lineiii{texpoint}{\constant{GSL\_CONST\_CGSM\_TEXPOINT}}{cm}
\lineiii{micron}{\constant{GSL\_CONST\_CGSM\_MICRON}}{cm}
\lineiii{angstrom}{\constant{GSL\_CONST\_CGSM\_ANGSTROM}}{cm}
\lineiii{hectare}{\constant{GSL\_CONST\_CGSM\_HECTARE}}{cm\^{}2}
\lineiii{acre}{\constant{GSL\_CONST\_CGSM\_ACRE}}{cm\^{}2}
\lineiii{barn}{\constant{GSL\_CONST\_CGSM\_BARN}}{cm\^{}2}
\lineiii{liter}{\constant{GSL\_CONST\_CGSM\_LITER}}{cm\^{}3}
\lineiii{us\_gallon}{\constant{GSL\_CONST\_CGSM\_US\_GALLON}}{cm\^{}3}
\lineiii{quart}{\constant{GSL\_CONST\_CGSM\_QUART}}{cm\^{}3}
\lineiii{pint}{\constant{GSL\_CONST\_CGSM\_PINT}}{cm\^{}3}
\lineiii{cup}{\constant{GSL\_CONST\_CGSM\_CUP}}{cm\^{}3}
\lineiii{fluid\_ounce}{\constant{GSL\_CONST\_CGSM\_FLUID\_OUNCE}}{cm\^{}3}
\lineiii{tablespoon}{\constant{GSL\_CONST\_CGSM\_TABLESPOON}}{cm\^{}3}
\lineiii{teaspoon}{\constant{GSL\_CONST\_CGSM\_TEASPOON}}{cm\^{}3}
\lineiii{canadian\_gallon}{\constant{GSL\_CONST\_CGSM\_CANADIAN\_GALLON}}{cm\^{}3}
\lineiii{uk\_gallon}{\constant{GSL\_CONST\_CGSM\_UK\_GALLON}}{cm\^{}3}
\lineiii{miles\_per\_hour}{\constant{GSL\_CONST\_CGSM\_MILES\_PER\_HOUR}}{cm / s}
\lineiii{kilometers\_per\_hour}{\constant{GSL\_CONST\_CGSM\_KILOMETERS\_PER\_HOUR}}{cm / s}
\lineiii{knot}{\constant{GSL\_CONST\_CGSM\_KNOT}}{cm / s}
\lineiii{pound\_mass}{\constant{GSL\_CONST\_CGSM\_POUND\_MASS}}{g}
\lineiii{ounce\_mass}{\constant{GSL\_CONST\_CGSM\_OUNCE\_MASS}}{g}
\lineiii{ton}{\constant{GSL\_CONST\_CGSM\_TON}}{g}
\lineiii{metric\_ton}{\constant{GSL\_CONST\_CGSM\_METRIC\_TON}}{g}
\lineiii{uk\_ton}{\constant{GSL\_CONST\_CGSM\_UK\_TON}}{g}
\lineiii{troy\_ounce}{\constant{GSL\_CONST\_CGSM\_TROY\_OUNCE}}{g}
\lineiii{carat}{\constant{GSL\_CONST\_CGSM\_CARAT}}{g}
\lineiii{unified\_atomic\_mass}{\constant{GSL\_CONST\_CGSM\_UNIFIED\_ATOMIC\_MASS}}{g}
\lineiii{gram\_force}{\constant{GSL\_CONST\_CGSM\_GRAM\_FORCE}}{cm g / s\^{}2}
\lineiii{pound\_force}{\constant{GSL\_CONST\_CGSM\_POUND\_FORCE}}{cm g / s\^{}2}
\lineiii{kilopound\_force}{\constant{GSL\_CONST\_CGSM\_KILOPOUND\_FORCE}}{cm g / s\^{}2}
\lineiii{poundal}{\constant{GSL\_CONST\_CGSM\_POUNDAL}}{cm g / s\^{}2}
\lineiii{calorie}{\constant{GSL\_CONST\_CGSM\_CALORIE}}{g cm\^{}2 / s\^{}2}
\lineiii{btu}{\constant{GSL\_CONST\_CGSM\_BTU}}{g cm\^{}2 / s\^{}2}
\lineiii{therm}{\constant{GSL\_CONST\_CGSM\_THERM}}{g cm\^{}2 / s\^{}2}
\lineiii{horsepower}{\constant{GSL\_CONST\_CGSM\_HORSEPOWER}}{g cm\^{}2 / s\^{}3}
\lineiii{bar}{\constant{GSL\_CONST\_CGSM\_BAR}}{g / cm s\^{}2}
\lineiii{std\_atmosphere}{\constant{GSL\_CONST\_CGSM\_STD\_ATMOSPHERE}}{g / cm s\^{}2}
\lineiii{torr}{\constant{GSL\_CONST\_CGSM\_TORR}}{g / cm s\^{}2}
\lineiii{meter\_of\_mercury}{\constant{GSL\_CONST\_CGSM\_METER\_OF\_MERCURY}}{g / cm s\^{}2}
\lineiii{inch\_of\_mercury}{\constant{GSL\_CONST\_CGSM\_INCH\_OF\_MERCURY}}{g / cm s\^{}2}
\lineiii{inch\_of\_water}{\constant{GSL\_CONST\_CGSM\_INCH\_OF\_WATER}}{g / cm s\^{}2}
\lineiii{psi}{\constant{GSL\_CONST\_CGSM\_PSI}}{g / cm s\^{}2}
\lineiii{poise}{\constant{GSL\_CONST\_CGSM\_POISE}}{g / cm s}
\lineiii{stokes}{\constant{GSL\_CONST\_CGSM\_STOKES}}{cm\^{}2 / s}
\lineiii{faraday}{\constant{GSL\_CONST\_CGSM\_FARADAY}}{A s / mol}
\lineiii{electron\_charge}{\constant{GSL\_CONST\_CGSM\_ELECTRON\_CHARGE}}{A s}
\lineiii{gauss}{\constant{GSL\_CONST\_CGSM\_GAUSS}}{g / A s\^{}2}
\lineiii{stilb}{\constant{GSL\_CONST\_CGSM\_STILB}}{cd / cm\^{}2}
\lineiii{lumen}{\constant{GSL\_CONST\_CGSM\_LUMEN}}{cd sr}
\lineiii{lux}{\constant{GSL\_CONST\_CGSM\_LUX}}{cd sr / cm\^{}2}
\lineiii{phot}{\constant{GSL\_CONST\_CGSM\_PHOT}}{cd sr / cm\^{}2}
\lineiii{footcandle}{\constant{GSL\_CONST\_CGSM\_FOOTCANDLE}}{cd sr / cm\^{}2}
\lineiii{lambert}{\constant{GSL\_CONST\_CGSM\_LAMBERT}}{cd sr / cm\^{}2}
\lineiii{footlambert}{\constant{GSL\_CONST\_CGSM\_FOOTLAMBERT}}{cd sr / cm\^{}2}
\lineiii{curie}{\constant{GSL\_CONST\_CGSM\_CURIE}}{1 / s}
\lineiii{roentgen}{\constant{GSL\_CONST\_CGSM\_ROENTGEN}}{A s / g}
\lineiii{rad}{\constant{GSL\_CONST\_CGSM\_RAD}}{cm\^{}2 / s\^{}2}
\lineiii{solar\_mass}{\constant{GSL\_CONST\_CGSM\_SOLAR\_MASS}}{g}
\lineiii{bohr\_radius}{\constant{GSL\_CONST\_CGSM\_BOHR\_RADIUS}}{cm}
\lineiii{vacuum\_permittivity}{\constant{GSL\_CONST\_CGSM\_VACUUM\_PERMITTIVITY}}{A\^{}2 s\^{}4 / g cm\^{}3}
\end{longtableiii}

\section[\protect\module{pygsl.const.num} --- Scientific number constants]
{\protect\module{pygsl.const.num} \\ Scientific number constants} 
\label{cha:const-num-module}

\begin{verbatim}
from pygsl.const import *
# an alternative to
# from pygsl.const.num import *
print "fine structure is 1/137 with an error of %g%%"%(abs(1.0/137.0/fine_structure-1.0)*100.0)
\end{verbatim}

Only two constants are available:\nopagebreak
\begin{longtableiii}{l|l|l}{constant}{Name}{gsl Name}{unit}
\lineiii{fine\_structure}{\constant{GSL\_CONST\_NUM\_FINE\_STRUCTURE}}{1}
\lineiii{avogadro}{\constant{GSL\_CONST\_NUM\_AVOGADRO}}{1 / mol}
\end{longtableiii}


\chapter[\protect\module{pygsl.chebyshev}]
{\protect\module{pygsl.chebyshev}}
\label{cha:statistics-module}

\declaremodule{standard}{pygsl.chebyshev}
\moduleauthor{Pierre Schnizer}{schnizer@users.sourceforge.net}

\begin{classdesc}{cheb_series}{}
  This base class can be instantiated by its name
\end{classdesc}
\begin{verbatim}
import pygsl.chebyshev
s=pygsl.chebyshev.cheb_series()
\end{verbatim}

\begin{methoddesc}{__init__}{n}\index{__init__}
            n ... number of coefficients        
\end{methoddesc}
\begin{methoddesc}{init}{f, a, b}\index{init}
        This function computes the Chebyshev approximation for the
        function F over the range (a,b) to the previously specified order.
        The computation of the Chebyshev approximation is an O($n^2$)
        process, and requires n function evaluations.

            f ... a gsl_function
            a ... lower limit
            b ... upper limit
        
\end{methoddesc}
\begin{methoddesc}{eval}{x}\index{eval}
        This function evaluates the Chebyshev series at a given point X.
\end{methoddesc}
\begin{methoddesc}{eval_err}{x}\index{eval_err}
         This function computes the Chebyshev series  at a given point X,
         estimating both the series RESULT and its absolute error ABSERR.
         The error estimate is made from the first neglected term in the
         series.
\end{methoddesc}
\begin{methoddesc}{eval_n}{n, x}\index{eval_n}
         This function evaluates the Chebyshev series at a given point
         x, to (at most) the given order n
\end{methoddesc}
\begin{methoddesc}{eval_n_err}{n, x}\index{eval_n_err}
        This function evaluates a Chebyshev series at a given point X,
        estimating both the series RESULT and its absolute error ABSERR,
        to (at most) the given order ORDER.  The error estimate is made
        from the first neglected term in the series.
\end{methoddesc}

\begin{methoddesc}{calc_deriv}{}\index{calc_deriv}
        This method computes the derivative of the series CS. It returns
        a new instance of the cheb_series class.
\end{methoddesc}
\begin{methoddesc}{calc_integ}{}\index{calc_integ}
        This method computes the integral of the series CS. It returns
        a new instance of the cheb_series class.
\end{methoddesc}
\begin{methoddesc}{get_a}{}\index{get_a}
        Get the lower boundary of the current representation       
\end{methoddesc}
\begin{methoddesc}{get_b}{}\index{get_b}
        Get the upper boundary of the current representation        
\end{methoddesc}
\begin{methoddesc}{get_coefficients}{}\index{get_coefficients}
        Get the chebyshev coefficients.         
\end{methoddesc}
\begin{methoddesc}{get_f}{}\index{get_f}
        Get the value f (what is it ?) The documentation does not tell anything
        about it.        
\end{methoddesc}
\begin{methoddesc}{get_order_sp}{}\index{get_order_sp}
        Get the value f (what is it ?) The documentation does not tell anything
        about it.        
\end{methoddesc}
\begin{methoddesc}{set_a}{}\index{set_a}
        Set the lower boundary of the current representation        
\end{methoddesc}
\begin{methoddesc}{set_b}{}\index{set_b}
        Set the upper boundary of the current         
\end{methoddesc}
\begin{methoddesc}{set_coefficients}{}\index{set_coefficients}
        Sets the chebyshev coefficients. 
\end{methoddesc}
\begin{methoddesc}{set_f}{f}\index{set_f}
        Set the value f (what is it ?)        
\end{methoddesc}
\begin{methoddesc}{set_order_sp}{...}\index{set_order_sp}
        Set the value f (what is it ?)        
\end{methoddesc}


\begin{funcdesc}{gsl_function}{f, params}\index{gsl_function}

    This class defines the callbacks known as gsl_function to
    gsl.

    e.g to supply the function f:
    
    def f(x, params):
        a = params[0]
        b = parmas[1]
        c = params[3]
        return a * x ** 2 + b * x + c

    to some solver, use

    function = gsl_function(f, params)
    
\end{funcdesc}

%%% Local Variables: 
%%% mode: latex
%%% TeX-master: "ref"
%%% End: 

\chapter[\protect\module{pygsl.deriv} --- NumericalDifferentiation]%
{\protect\module{pygsl.deriv} \\ Numerical Differentiation}
\label{cha:diff-module}

\declaremodule{extension}{pygsl.deriv}%
 \moduleauthor{Pierre  Schnizer}{schnizer@users.sourceforge.net}%
 \modulesynopsis{Numerical  Differentiation}%

\begin{quote}
  This chapter describes the available functions for numerical differentiation.
\end{quote}

The functions described in this chapter compute numerical derivatives by finite
differencing.  An adaptive algorithm is used to find the best choice of finite
difference and to estimate the error in the derivative. This module supersedes
the diff module which has been deprecated with the release of GSL-1. XXX


\begin{funcdesc}{central}{func, x, h}
  This function computes the numerical derivative of the function \var{func} at
  the point \var{x} using an adaptive central difference algorithm with a step
  size of h.  A tuple \code{(result, error)} is returned with the derivative
  and its estimated absolute error.
\end{funcdesc}

\begin{funcdesc}{backward}{func, x, h}
  This function computes the numerical derivative of the function \var{func} at
  the point \var{x} using an adaptive backward difference algorithm with a step
  size of h.  The function \var{func} is evaluated only at points smaller than
  \var{x} and at \var{x} itself.  A tuple \code{(result, error)} is returned
  with the derivative and its estimated absolute error.
\end{funcdesc}

\begin{funcdesc}{forward}{func, x, h}
  This function computes the numerical derivative of the function \var{func} at
  the point \var{x} using an adaptive forward difference algorithm with a step
  size of h.  The function \var{func} is evaluated only at points greater than
  \var{x} and at \var{x} itself.  A tuple \code{(result, error)} is returned
  with the derivative and its estimated absolute error.
\end{funcdesc}


\begin{seealso}
  The algorithms used by these functions are described in the following book:
  \seetext{S.D.\ Conte and Carl de Boor, \emph{Elementary Numerical Analysis:
      An Algorithmic Approach}, McGraw-Hill, 1972.}
\end{seealso}



%% Local Variables:
%% mode: LaTeX
%% mode: auto-fill
%% fill-column: 79
%% indent-tabs-mode: nil
%% ispell-dictionary: "british"
%% reftex-fref-is-default: nil
%% TeX-auto-save: t
%% TeX-command-default: "pdfeLaTeX"
%% TeX-master: "pygsl"
%% TeX-parse-self: t
%% End:


\chapter[\protect\module{pygsl.histogram} --- Histogram Types]
{\protect\module{pygsl.histogram} \\ Histogram Types}
\label{cha:histogram-module}
\declaremodule{extension}{pygsl.histogram}
\moduleauthor{Achim G\"adke}{achimgaedke@users.sourceforge.net}

This chapter is about the \class{histogram} and \class{histogram2d} type that
are contained in \module{pygsl.histogram}.

\section{\protect\class{histogram} --- 1-dimensional histograms}

\begin{classdesc}{histogram}{\texttt{long} size \code{|} \class{histogram} h}
This type implements all methods on \ctype{struct gsl_histogram}.
\end{classdesc}

\begin{methoddesc}{alloc}{\texttt{long} length}
allocate necessary space, \hfill returns \texttt{None}
\end{methoddesc}
\begin{methoddesc}{set_ranges_uniform}{\texttt{float} upper, \texttt{float} lower}
set the ranges to uniform distance, \hfill returns \texttt{None}
\end{methoddesc}
\begin{methoddesc}{reset}{}
sets all bin values to 0, \hfill returns \texttt{None}
\end{methoddesc}
\begin{methoddesc}{increment}{\texttt{float} x}
increments corresponding bin, \hfill returns \texttt{None}
\end{methoddesc}
\begin{methoddesc}{accumulate}{\texttt{float} x, \texttt{float} weight}
adds the weight to corresponding bin, \hfill returns \texttt{None}
\end{methoddesc}
\begin{methoddesc}{max}{}
returns upper range, \hfill as \texttt{float}
\end{methoddesc}
\begin{methoddesc}{min}{}
returns lower range, \hfill as \texttt{float}
\end{methoddesc}
\begin{methoddesc}{bins}{}
returns number of bins, \hfill as \texttt{long}
\end{methoddesc}
\begin{methoddesc}{get}{\texttt{long} n}
gets value of indexed bin, \hfill returns \texttt{float}
\end{methoddesc}
\begin{methoddesc}{get_range}{\texttt{long} n}
gets upper and lower range of indexed bin, \hfill returns \texttt{(float,float)}
\end{methoddesc}
\begin{methoddesc}{find}{\texttt{float} x}
finds index of corresponding bin, \hfill returns \texttt{long}
\end{methoddesc}
\begin{methoddesc}{max_val}{}
returns maximal bin value, \hfill as \texttt{float}
\end{methoddesc}
\begin{methoddesc}{max_bin}{}
returns bin index with maximal value, \hfill as \texttt{long}
\end{methoddesc}
\begin{methoddesc}{min_val}{}
returns minimal bin value, \hfill as \texttt{float}
\end{methoddesc}
\begin{methoddesc}{min_bin}{}
returns bin index with minimal value, \hfill as \texttt{long}
\end{methoddesc}
\begin{methoddesc}{mean}{}
returns mean of histogram, \hfill as \texttt{float}
\end{methoddesc}
\begin{methoddesc}{sigma}{}
returns std deviation of histogram, \hfill as \texttt{float}
\end{methoddesc}
\begin{methoddesc}{sum}{}
returns sum of bin values, \hfill as \texttt{float}
\end{methoddesc}
\begin{methoddesc}{set_ranges}{\texttt{sequence} ranges}
sets range according given sequence, \hfill returns \texttt{None}
\end{methoddesc}
\begin{methoddesc}{shift}{\texttt{float} offset}
shifts the contents of the bins by the given offset, \hfill returns
\texttt{None}
\end{methoddesc}
\begin{methoddesc}{scale}{\texttt{float} scale}
multiplies the contents of the bins by the given scale, \hfill returns \texttt{None}
\end{methoddesc}
\begin{methoddesc}{equal_bins_p}{}
true if the all of the individual bin ranges are identical, \hfill returns \texttt{int}
\end{methoddesc}
\begin{methoddesc}{add}{\texttt{histogram} h}
adds the contents of the bins, \hfill returns \texttt{None}
\end{methoddesc}
\begin{methoddesc}{sub}{\texttt{histogram} h}
substracts the contents of the bins, \hfill returns \texttt{None}
\end{methoddesc}
\begin{methoddesc}{mul}{\texttt{histogram} h}
multiplicates the contents of the bins, \hfill returns \texttt{None}
\end{methoddesc}
\begin{methoddesc}{div}{\texttt{histogram} h}
divides the contents of the bins, \hfill returns \texttt{None}
\end{methoddesc}
\begin{methoddesc}{clone}{\texttt{histogram} h}
returns a new copy of this histogram, \hfill returns new \texttt{histogram}
\end{methoddesc}
\begin{methoddesc}{copy}{\texttt{histogram} h}
copies the given histogram to myself, \hfill returns \texttt{None}
\end{methoddesc}
\begin{methoddesc}{read}{\texttt{file} input}
reads histogram binary data from file, \hfill returns \texttt{None}
\end{methoddesc}
\begin{methoddesc}{write}{\texttt{file} output}
writes histogram binary data to file, \hfill returns \texttt{None}
\end{methoddesc}
\begin{methoddesc}{scanf}{\texttt{file} input}
reads histogram data from file using scanf, \hfill returns \texttt{None}
\end{methoddesc}
\begin{methoddesc}{printf}{\texttt{file} output}
writes histogram data to file using printf, \hfill returns \texttt{None}
\end{methoddesc}


Some mapping operations are supported, too:\nopagebreak
\begin{tableii}{l|l}{texttt}{Mapping syntax}{Effect}
\lineii{histogram[index]}{returns the value of the indexed bin}
\lineii{histogram[index]=value}{sets the value of the indexed bin}
\lineii{len(histogram)}{returns the length of the histogram}
\end{tableii}

\begin{seealso}
For the special meaning and details please consult the GNU Scientific Library
reference.
\end{seealso}


\section{\protect\class{histogram2d} --- 2-dimensional histograms}

\begin{classdesc}{histogram2d}{\texttt{long} size x, \texttt{long} size y
                               \code{|} \class{histogram2d} h}
This class holds a 2d array and 2 sets of ranges for x and y coordinates for a
two paramter statistical event. It can be constructed by size parameters or
as a copy from another histogram. Most of the methods are the same as of
\class{histogram}.
\end{classdesc}

\begin{methoddesc}{set_ranges_uniform}{\texttt{float} xmin, \texttt{float} xmax,
                                       \texttt{float} ymin, \texttt{float} ymax}
set the ranges to uniform distance, \hfill returns \texttt{None}
\end{methoddesc}
\begin{methoddesc}{alloc}{\texttt{long} nx, \texttt{long} ny}
allocate necessary space, \hfill returns \texttt{None}
\end{methoddesc}
\begin{methoddesc}{reset}{}
sets all bin values to 0, \hfill returns \texttt{None}
\end{methoddesc}
\begin{methoddesc}{increment}{\texttt{float} x, \texttt{float} y}
increments corresponding bin, \hfill returns \texttt{None}
\end{methoddesc}
\begin{methoddesc}{accumulate}{\texttt{float} x, \texttt{float} y,
                               \texttt{float} weight}
adds the weight to corresponding bin, \hfill returns \texttt{None}
\end{methoddesc}
\begin{methoddesc}{xmax}{}
returns upper x range \hfill as \texttt{float}
\end{methoddesc}
\begin{methoddesc}{xmin}{}
returns lower x range \hfill as \texttt{float}
\end{methoddesc}
\begin{methoddesc}{ymax}{}
returns upper y range \hfill as \texttt{float}
\end{methoddesc}
\begin{methoddesc}{ymin}{}
returns lower y range \hfill as \texttt{float}
\end{methoddesc}
\begin{methoddesc}{nx}{}
returns number of x bins \hfill as \texttt{long}
\end{methoddesc}
\begin{methoddesc}{ny}{}
returns number of y bins \hfill as \texttt{long}
\end{methoddesc}
\begin{methoddesc}{get}{\texttt{long} i, \texttt{long} j}
gets value of indexed bin,\hfill returns \texttt{float}
\end{methoddesc}
\begin{methoddesc}{get_xrange}{\texttt{long} i}
gets upper and lower x range of indexed bin,
\hfill returns \texttt{(float \textrm{lower}, float \textrm{upper})}
\end{methoddesc}
\begin{methoddesc}{get_yrange}{\texttt{long} j}
gets upper and lower y range of indexed bin,
\hfill returns \texttt{(float \textrm{lower}, float \textrm{upper})}
\end{methoddesc}
\begin{methoddesc}{find}{\texttt{float} x, \texttt{float} y}
finds index pair of corresponding value pair,
\hfill returns (\texttt{long},\texttt{long})
\end{methoddesc}
\begin{methoddesc}{max_val}{}
returns maximal bin value \hfill as \texttt{float}
\end{methoddesc}
\begin{methoddesc}{max_bin}{}
returns bin index with maximal value \hfill as \texttt{long}
\end{methoddesc}
\begin{methoddesc}{min_val}{}
returns minimal bin value \hfill as \texttt{float}
\end{methoddesc}
\begin{methoddesc}{min_bin}{}
returns bin index with minimal value \hfill as \texttt{long}
\end{methoddesc}
\begin{methoddesc}{sum}{}
returns sum of bin values \hfill as \texttt{float}
\end{methoddesc}
\begin{methoddesc}{xmean}{}
returns x mean of histogram \hfill as \texttt{float}
\end{methoddesc}
\begin{methoddesc}{xsigma}{}
returns x std deviation of histogram \hfill as \texttt{float}
\end{methoddesc}
\begin{methoddesc}{ymean}{}
returns y mean of histogram \hfill as\texttt{float}
\end{methoddesc}
\begin{methoddesc}{ysigma}{}
returns y std deviation of histogram \hfill as \texttt{float}
\end{methoddesc}
\begin{methoddesc}{cov}{}
returns covariance of histogram \hfill as \texttt{float}
\end{methoddesc}
\begin{methoddesc}{set_ranges}{sequence xranges, sequence yranges}
set the ranges according to given sequences, \hfill returns \texttt{None}
\end{methoddesc}
\begin{methoddesc}{shift}{\texttt{float} offset}
shifts the contents of the bins by the given offset, \hfill returns \texttt{None}
\end{methoddesc}
\begin{methoddesc}{scale}{\texttt{float} scale}
multiplies the contents of the bins by the given scale, \hfill returns \texttt{None}
\end{methoddesc}
\begin{methoddesc}{equal_bins_p}{}
true if the all of the individual bin ranges are identical, \hfill returns \texttt{int}
\end{methoddesc}
\begin{methoddesc}{add}{\class{histogram2d} h}
adds the contents of the bins, \hfill returns \texttt{None}
\end{methoddesc}
\begin{methoddesc}{sub}{\class{histogram2d} h}
substracts the contents of the bins, \hfill returns \texttt{None}
\end{methoddesc}
\begin{methoddesc}{mul}{\class{histogram2d} h}
multiplicates the contents of the bins, \hfill returns \texttt{None}
\end{methoddesc}
\begin{methoddesc}{div}{\class{histogram2d} h}
divides the contents of the bins, \hfill returns \texttt{None}
\end{methoddesc}
\begin{methoddesc}{clone}{}
returns a copy instance of \hfill\class{histogram2d}
\end{methoddesc}
\begin{methoddesc}{copy}{\class{histogram2d} h}
copies the given histogram to myself, \hfill returns \texttt{None}
\end{methoddesc}
\begin{methoddesc}{read}{file input}
reads histogram binary data from file, \hfill returns \texttt{None}
\end{methoddesc}
\begin{methoddesc}{writew}{file output}
writes histogram binary data to file, \hfill returns \texttt{None}
\end{methoddesc}
\begin{methoddesc}{scanf}{file input}
reads histogram data from file using scanf, \hfill returns \texttt{None}
\end{methoddesc}
\begin{methoddesc}{printf}{file input}
writes histogram data to file using printf, \hfill returns \texttt{None}
\end{methoddesc}

Some mapping operations are supported, too:\nopagebreak
\begin{tableii}{l|l}{code}{Mapping syntax}{Effect}
\lineii{histogram[x\_index,y\_index]}{returns the value of the indexed bin}
\lineii{histogram[x\_index,y\_index]=value}{sets the value of the indexed bin}
\lineii{len(histogram)}{returns the size of the histogram, i.e nx$\times$ny}
\end{tableii}


\begin{seealso}
For the special meaning and details please consult the GNU Scientific Library
reference.
\end{seealso}

\section{\protect\class{histogram_pdf} and \protect\class{histogram2d_pdf}}

To be implemented\dots

%% Local Variables:
%% mode: LaTeX
%% mode: auto-fill
%% fill-column: 90
%% indent-tabs-mode: nil
%% ispell-dictionary: "american"
%% reftex-fref-is-default: nil
%% TeX-auto-save: t
%% TeX-command-default: "pdfeLaTeX"
%% TeX-master: "pygsl"
%% TeX-parse-self: t
%% End:


\chapter[\protect\module{pygsl.rng} --- Random Number Generators]
{\protect\module{pygsl.rng} \\ Random Number Generators}
\label{cha:rng-module}
% $Id$
% pygsl/doc/rng.tex

\declaremodule{standard}{pygsl.rng}%
\moduleauthor{Pierre Schnizer}{schnizer@users.sourceforge.net}%
\moduleauthor{Original Author: Achim G\"adke}{achimgaedke@users.sourceforge.net}%

This chapter introduces the random number generator type provided by \module{pygsl}.

\section{Random Number Generators}

All random number generatores are the same python type (PyGSL_rng), but using the
approbriate GSL random generator for generating the random numbers. Use the method
\code{name} to get the name of the rng used internally.

Methods of
this type \pytype{rng} provide the transformation to different probability
distributions and give access to basic properties of random number generators. 
All methods allow to pass one optional integer. Then the method will be evaluated n times and the result
will be returned as an array.

\begin{pytypedesc}{rng}{\texttt{string} typenamme \code{|} \class{rng} r}
  This base class can be instantiated by its name
\begin{verbatim}
import pygsl.rng
my_ran0=pygsl.rng.ran0()
\end{verbatim}
.
\end{pytypedesc}
The type of the allocated generator is given by the method
\begin{methoddesc}{name}{}
  which returns its name as string.
\end{methoddesc}
All generators can be seeded with
\begin{methoddesc}{set}{seed}
  which sets the internal seed according to the positive integer {\tt seed}. Zero as seed
  has a special meaning, please read details in the gsl reference.
\end{methoddesc}
The basic returned number type is integer, these are generated by
\begin{methoddesc}{get}{}
  which returns the next number of the pseudo random sequence.
\end{methoddesc}
All methods support internal sampling; i.e each method has an optional integer. 
If given it will return a sample of the approbriate size.
\begin{methoddesc}{get}{|n}
  will return the next n numbers of the pseudo random sequence.
\end{methoddesc}

Basic information about these numbers can be obtained by
\begin{methoddesc}{max}{}
  maximum number of this sequence and
\end{methoddesc}
\begin{methoddesc}{min}{}
  minimum number of this sequence.
\end{methoddesc}
Implemented uniform probability densities are:
\begin{methoddesc}{uniform}{}
  returns a real number between $[0,1)$.
\end{methoddesc}
\begin{methoddesc}{uniform_pos}{}
  returns a real number between $(0,1)$ --- this excludes 0.
\end{methoddesc}
\begin{methoddesc}{uniform_int}{upper limit}
  returns an integer from 0 to the upper limit (exclusive). If this limit is larger than
  the number of return values of the underlying generator, \exception{pygsl.gsl_Error} is
  raised.
\end{methoddesc}
Furthermore a lot of derived probability densities can be used:
\begin{methoddesc}{gaussian}{sigma}
  gaussian distribution with mean 0 and given sigma \hfill returns {\tt float}
\end{methoddesc}
\begin{methoddesc}{gaussian\_ratio\_method}{sigma}
  gaussian distribution with mean 0 and given sigma.  This variate uses the
  Kinderman-Monahan ratio method.  \hfill returns {\tt float}
\end{methoddesc}
\begin{methoddesc}{ugaussian}{}
  gaussian distribution with unit sigma and mean 0.  \hfill returns {\tt float}
\end{methoddesc}
\begin{methoddesc}{ugaussian\_ratio\_method}{}
  gaussian distribution with unit sigma and mean 0.  This variate uses the
  Kinderman-Monahan ratio method.  \hfill returns {\tt float}
\end{methoddesc}
\begin{methoddesc}{gaussian\_tail}{sigma, a}
  upper tail of a Gaussian distribution with standard deviation sigma>0.  \hfill returns
  {\tt float}
\end{methoddesc}
\begin{methoddesc}{ugaussian\_tail}{a}
  upper tail of a Gaussian distribution with unit standard deviation.  \hfill returns {\tt
    float}
\end{methoddesc}
\begin{methoddesc}{bivariate\_gaussian}{sigma\_x, sigma\_y, rho}
  pair of correlated gaussian variates, with mean zero, correlation coefficient rho and
  standard deviations sigma\_x and sigma\_y in the x and y directions \hfill returns~{\tt
    (float,float)}
\end{methoddesc}
\begin{methoddesc}{exponential}{mu}
  \hfill returns {\tt float}
\end{methoddesc}
\begin{methoddesc}{laplace}{mu}
  \hfill returns {\tt float}
\end{methoddesc}
\begin{methoddesc}{exppow}{mu, a}
  \hfill returns {\tt float}
\end{methoddesc}
\begin{methoddesc}{cauchy}{mu}
  \hfill returns {\tt float}
\end{methoddesc}
\begin{methoddesc}{rayleigh}{sigma}
  \hfill returns {\tt float}
\end{methoddesc}
\begin{methoddesc}{rayleigh\_tail}{a, sigma}
  \hfill returns {\tt float}
\end{methoddesc}
\begin{methoddesc}{levy}{mu,a}
  \hfill returns {\tt float}
\end{methoddesc}
\begin{methoddesc}{levy_skew}{mu,a,beta}
  \hfill returns {\tt float}
\end{methoddesc}
\begin{methoddesc}{gamma}{a, b}
  \hfill returns {\tt float}
\end{methoddesc}
\begin{methoddesc}{gamma\_int}{long a}
  \hfill returns {\tt float}
\end{methoddesc}
\begin{methoddesc}{flat}{a, b}
  \hfill returns {\tt float}
\end{methoddesc}
\begin{methoddesc}{lognormal}{zeta, sigma}
  \hfill returns {\tt float}
\end{methoddesc}
\begin{methoddesc}{chisq}{nu}
  \hfill returns {\tt float}
\end{methoddesc}
\begin{methoddesc}{fdist}{nu1, nu2}
  \hfill returns {\tt float}
\end{methoddesc}
\begin{methoddesc}{tdist}{nu}
  \hfill returns {\tt float}
\end{methoddesc}
\begin{methoddesc}{beta}{a, b}
  \hfill returns {\tt float}
\end{methoddesc}
\begin{methoddesc}{logistic}{mu}
  \hfill returns {\tt float}
\end{methoddesc}
\begin{methoddesc}{pareto}{a, b}
  \hfill returns {\tt float}
\end{methoddesc}
\begin{methoddesc}{dir\_2d}{}
  \hfill returns {\tt (float, float)}
\end{methoddesc}
\begin{methoddesc}{dir\_2d\_trig\_method}{}
  \hfill returns {\tt (float, float)}
\end{methoddesc}
\begin{methoddesc}{dir\_3d}{}
  \hfill returns {\tt (float, float, float)}
\end{methoddesc}
\begin{methoddesc}{dir\_nd}{int n}
  \hfill returns {\tt (float, \dots, float)}
\end{methoddesc}
\begin{methoddesc}{weibull}{mu, a}
  \hfill returns {\tt float}
\end{methoddesc}
\begin{methoddesc}{gumbel1}{a, b}
  \hfill returns {\tt float}
\end{methoddesc}
\begin{methoddesc}{gumbel2}{}
\end{methoddesc}
\begin{methoddesc}{poisson}{}
\end{methoddesc}
\begin{methoddesc}{bernoulli}{}
\end{methoddesc}
\begin{methoddesc}{binomial}{}
\end{methoddesc}
\begin{methoddesc}{negative\_binomial}{}
\end{methoddesc}
\begin{methoddesc}{pascal}{}
\end{methoddesc}
\begin{methoddesc}{geometric}{}
\end{methoddesc}
\begin{methoddesc}{hypergeometric}{}
\end{methoddesc}
\begin{methoddesc}{logarithmic}{}
\end{methoddesc}
\begin{methoddesc}{landau}{}
\end{methoddesc}
\begin{methoddesc}{erlang}{}
\end{methoddesc}


The different generator classes are created according to the output of
\code{gsl_rng_types_setup()} when the \module{pygsl.rng} is loaded. Here is the list of
children from \class{rng} for gsl-1.2: \newline \class{rng_borosh13}, \class{rng_coveyou},
\class{rng_cmrg}, \class{rng_fishman18}, \class{rng_fishman20}, \class{rng_fishman2x},
\class{rng_gfsr4}, \class{rng_knuthran}, \class{rng_knuthran2}, \class{rng_lecuyer21},
\class{rng_minstd}, \class{rng_mrg}, \class{rng_mt19937}, \class{rng_mt19937_1999},
\class{rng_mt19937_1998}, \class{rng_r250}, \class{rng_ran0}, \class{rng_ran1},
\class{rng_ran2}, \class{rng_ran3}, \class{rng_rand}, \class{rng_rand48},
\class{rng_random128_bsd}, \class{rng_random128_glibc2}, \class{rng_random128_libc5},
\class{rng_random256_bsd}, \class{rng_random256_glibc2}, \class{rng_random256_libc5},
\class{rng_random32_bsd}, \class{rng_random32_glibc2}, \class{rng_random32_libc5},
\class{rng_random64_bsd}, \class{rng_random64_glibc2}, \class{rng_random64_libc5},
\class{rng_random8_bsd}, \class{rng_random8_glibc2}, \class{rng_random8_libc5},
\class{rng_random_bsd}, \class{rng_random_glibc2}, \class{rng_random_libc5},
\class{rng_randu}, \class{rng_ranf}, \class{rng_ranlux}, \class{rng_ranlux389},
\class{rng_ranlxd1}, \class{rng_ranlxd2}, \class{rng_ranlxs0}, \class{rng_ranlxs1},
\class{rng_ranlxs2}, \class{rng_ranmar}, \class{rng_slatec}, \class{rng_taus},
\class{rng_taus2}, \class{rng_taus113}, \class{rng_transputer}, \class{rng_tt800},
\class{rng_uni}, \class{rng_uni32}, \class{rng_vax}, \class{rng_waterman14}, and
\class{rng_zuf}.  
\newline 

The default generator of the \class{rng} defaults to {\tt rng_mt19937} but can be set from the
environment variable \envvar{GSL_RNG_TYPE} using the function \function{rng.env_setup()}.

\section{Probability Density Functions}


\section{Using probability densities with random number generators}


%% Local Variables:
%% mode: LaTeX
%% mode: auto-fill
%% fill-column: 90
%% indent-tabs-mode: nil
%% ispell-dictionary: "british"
%% reftex-fref-is-default: nil
%% TeX-auto-save: t
%% TeX-command-default: "pdfeLaTeX"
%% TeX-master: "pygsl"
%% TeX-parse-self: t
%% End:


%\chapter[\protect\module{pygsl.sf} --- Special Functions]
%{\protect\module{pygsl.sf} \\ Special Functions}
%\label{cha:sf-module}
%\declaremodule{extension}{pygsl.sf}
\moduleauthor{Achim G\"adke}{achimgaedke@users.sourceforge.net}

This chapter shows you the list of implemented special function and explains
details of error handling and return values.

\section{Function list}

\begin{longtableii}{l|l}{texttt}{Function}{Description}
\lineii{}{ToDo}
\end{longtableii}

\section{Return values}

\section{Error handling}


\chapter[\protect\module{pygsl.sum} --- Series acceleration]{
  \protect\module{pygsl.sum} \\ Series acceleration}
\label{cha:sum-module}

\declaremodule{extension}{pygsl.sum}
\modulesynopsis{Series acceleration.}

This chapter describes the use of the series acceleration tools based
on the Levin $u$-transform.  This method takes a small number of terms
from the start of a series and uses a systematic approximation to
compute an extrapolated value and an estimate of its error. The
$u$-transform works for both convergent and divergent series,
including asymptotic series.

\begin{equation}
  \label{eq:levin}
  \function{levin_sum}\code{(a)} = (A, \epsilon)
  \qquad\text{where}
  \qquad
  A \approx \sum_{n=0}^{\infty} a_{n} \pm \epsilon, 
\end{equation}
$\code{a} = [a_{0}, a_{1}, \ldots, a_{n}]$, and $\epsilon$ is an
estimate of the absolute error.

Note: This function is intended for summing analytic series where each
term is known to high accuracy, and the rounding errors are assumed to
originate from finite precision. They are taken to be relative errors
of order \constant{GSL_DBL_EPSILON} for each term (as defined in the
\GSL{} source code).

\section{Function list}
\begin{funcdesc}{levin_sum}{a, truncate=False, info_dict=None}
  Return ($A, \epsilon$) where $A$ is the approximated sum of the
  series~(\ref{eq:levin}) and $\epsilon$ is its absolute error
  estimate.

  The calculation of the error in the extrapolated value is an
  O$(N^2)$ process, which is expensive in time and memory.  A full
  table of intermediate values and derivatives through to O$(N)$ must
  be computed and stored, but this does give a reliable error
  estimate.

  A faster but less reliable method which estimates the error from the
  convergence of the extrapolated value is employed if \var{truncate}
  is \code{True}.  This attempts to estimate the error from the
  ``truncation error'' in the extrapolation, the difference between
  the final two approximations. Using this method avoids the need to
  compute an intermediate table of derivatives because the error is
  estimated from the behavior of the extrapolated value
  itself. Consequently this algorithm is an O$(N)$ process and only
  requires O$(N)$ terms of storage. If the series converges sufficiently
  fast then this procedure can be acceptable. It is appropriate to use
  this method when there is a need to compute many extrapolations of
  series with similar convergence properties at high-speed. For
  example, when numerically integrating a function defined by a
  parameterized series where the parameter varies only slightly. A
  reliable error estimate should be computed first using the full
  algorithm described above in order to verify the consistency of the
  results.

  If a dictionary is passed as \var{info_dict}, then two entries will
  be added: \var{info_dict}\code{['terms_used']} will be the number of
  terms used\footnote{Note that it appears that this is the number of
    terms \emph{beyond} the first term that are used.  I.e.\ there are
    a total of $\var{terms_used}+1$ terms:
    \begin{equation}
      \var{sum_plain} = 
      \sum_{n=0}^{\var{terms_used}}
      a_{n}
    \end{equation}}
  and \var{info_dict}\code{['sum_plain']} will be the sum of these terms without
  acceleration.
\end{funcdesc}

\section{Further Reading}
For details on the underlying implementation of these functions please
consult the \GSL{} reference manual.  The algorithms used by these
functions are described Fessler \textit{et al.} (1983).  The theory of
the $u$-transform was presented Levin in 1973, and a review paper from
2000 by Homeier is available online.

\begin{seealso}
  \seetext{T.~Fessler, W.~F.~Ford, D.~A.~Smith, \textit{hurry: An
      acceleration algorithm for scalar sequences and series}. ACM
    Transactions on Mathematical Software, \textbf{9}(3):346--354,
    (1983), and Algorithm 602 9(3):355--357, 1983.}
  \seetext{D.~Levin, \textit{Development of Non-Linear Transformations
      for Improving Convergence of Sequences,} Intern.~J.~Computer
    Math. \textbf{B3}:371--388, (1973).}
  \seetitle[http://arXiv.org/abs/math/0005209]{Herbert H.~H.~Homeier,
    \textit{Scalar Levin-Type Sequence Transformations.}}{}
\end{seealso}
%%% Local Variables: 
%%% mode: latex
%%% TeX-master: "ref"
%%% End: 

\chapter[\protect\module{pygsl.statistics} --- Statistics
functions]{\protect\module{pygsl.statistics} \\ Statistics functions}
\label{cha:statistics-module}

\declaremodule{extension}{pygsl.statistics}
\moduleauthor{Pierre Schnizer}{schnizer@users.sourceforge.net}
\moduleauthor{Original Author: Jochen K\"upper}{jochen@jochen-kuepper.de}
\modulesynopsis{Statistical functions.}

\index{mean}
\index{standard deviation}
\index{variance}
\index{estimated standard deviation}
\index{estimated variance}
\index{t-test}
\index{range}
\index{min}
\index{max}
\index{kurtosis}
\index{skewness}
\index{autocorrelation}
\index{covariance}

\begin{quote}
   This chapter describes the statistical functions in the library.  The basic
   statistical functions include routines to compute the mean, variance and
   standard deviation. More advanced functions allow you to calculate absolute
   deviations, skewness, and kurtosis as well as the median and arbitrary
   percentiles.
\end{quote}

The algorithms provided here use recurrence relations to compute average
quantities in a stable way, without large intermediate values that might
overflow.  All functions work on any Python sequence (of appropriate
data-type), but see section \ref{sec:stat:speed-considerations} for advantages
and drawbacks of different kinds of input data.

\begin{seealso}
   For details on the underlying implementation of these functions please
   consult the \GSL{} reference manual.
\end{seealso}



\section{Organization of the module}
\label{sec:stat:organization}

Individual parts of the \gsl{} functions names, providing artificial namespaces
in C, are mapped to modules and submodules in \pygsl{}.  That is,
\cfunction{gsl_stats_mean} can be found as \function{pygsl.statistics.mean} and
\cfunction{gsl_stats_long_mean} as \function{pygsl.statistics.long.mean}.

The functions in the module are available in versions for datasets in the
standard and \numpy{} floating-point and integer types. The generic versions
available in the \module{pygsl.statistics} module are using the generic \gsl{}
\ctype{double} versions.  The submodules use \gsl{} functions according to the
submodule name, e.g. long for \module{pygsl.statistics.long}.

Implemented submodules are \module{char}, \module{uchar}, \module{short},
\module{int}, \module{long}, \module{float}, and \module{double}. The latter
one also serves as default and is used whenever you don't expclicitely state a
different datatype. In most cases it is appropriate to simply use the default
implementation as it covers the widest range of the real space, offers high
precision, and as such is simple to use. If you have a sequence of all integer
values it is straightforward to use \module{pygsl.statistics.long} functions as
these use an implementation corresponding to Pythons \class{Float}-type. These
implemented submodules represent all numeric datatypes available in Python
(\class{Int}, \class{Float}) besides \class{Long Int} which has no
representation in standard C, as well as all numeric datatypes available in
\numpy{} that have corresponding implementations in \gsl{} (on 32 bit systems
these are: Character, UnsigendInt8, Int16, Int32, Int, Float32, Float).



\section{Available functions}
\label{sec:stat:available-functions}

\subsection{Mean, Standard Deviation, and Variance}
\label{sec:stat:mean-stddev-var}

\begin{funcdesc}{mean}{x}\index{mean}
   Arithmetic mean (\emph{sample mean}) of \var{x}:
   \begin{equation}
      \hat\mu = \frac{1}{N} \sum x_i
   \end{equation}
\end{funcdesc}

\begin{funcdesc}{variance}{x}\index{variance}
   Estimated (\emph{sample}) variance of \var{x}:
   \begin{equation}
      \hat\sigma^2 = \frac{1}{N-1} \sum (x_i - \hat\mu)^2
   \end{equation}
   This function computes the mean via a call to \function{mean}.  If you have
   already computed the mean then you can pass it directly to
   \function{variance_m}.
\end{funcdesc}

\begin{funcdesc}{variance_m}{x, mean}\index{variance}
   Estimated (\emph{sample}) variance of \var{x} relative to \var{mean}:
   \begin{equation}
      \hat\sigma^2 = \frac{1}{N-1} \sum (x_i - mean)^2
   \end{equation}
\end{funcdesc}

\begin{funcdesc}{sd}{x}
\end{funcdesc}
\begin{funcdesc}{sd_m}{x, mean}\index{sd}\index{mean}
   The standard deviation is defined as the square root of the variance of
   \var{x}.  These functions returns the square root of the respective
   variance-functions above.
\end{funcdesc}

\begin{funcdesc}{variance_with_fixed_mean}{x, mean}\index{variance}\index{mean}
   Compute an unbiased estimate of the variance of \var{x} when the population
   mean \var{mean} of the underlying distribution is known \emph{a priori}.  In
   this case the estimator for the variance uses the factor $1/N$ and the
   sample mean $\hat\mu$ is replaced by the known population mean $\mu$:
   \begin{equation}
      \hat\sigma^2 = \frac{1}{N} \sum (x_i - \mu)^2
   \end{equation}
\end{funcdesc}


\subsection{Absolute deviation}
\label{sec:stat:absolute-deviation}

\begin{funcdesc}{absdev}{data}
   Compute the absolute deviation from the mean of \var{data} The absolute
   deviation from the mean is defined as
   \begin{equation}
      absdev  = (1/N) \sum |x_i - \hat\mu|
   \end{equation}
   where $x_i$ are the elements of the dataset \var{data}.  The absolute
   deviation from the mean provides a more robust measure of the width of a
   distribution than the variance.  This function computes the mean of
   \var{data} via a call to \function{mean}.
\end{funcdesc}

\begin{funcdesc}{absdev_m}{data, mean}
   Compute the absolute deviation of the dataset \var{data} relative to the
   given value of \var{mean}
   \begin{equation}
      absdev  = (1/N) \sum |x_i - mean|
   \end{equation}
   This function is useful if you have already computed the mean of \var{data}
   (and want to avoid recomputing it), or wish to calculate the absolute
   deviation relative to another value (such as zero, or the median).
\end{funcdesc}


\subsection{Higher moments (skewness and kurtosis)}
\label{sec:stat:higher-moments}

\begin{funcdesc}{skew}{data}
   Compute the skewness of \var{data}.  The skewness is defined as
   \begin{equation}
      skew = (1/N) \sum ((x_i - \hat\mu)/\hat\sigma)^3
   \end{equation}
   where $x_i$ are the elements of the dataset \var{data}.  The skewness
   measures the asymmetry of the tails of a distribution.
   
   The function computes the mean and estimated standard deviation of
   \var{data} via calls to \function{mean} and \function{sd}.
\end{funcdesc}


\begin{funcdesc}{skew_m_sd}{data, mean, sd}
   Compute the skewness of the dataset \var{data} using the given values of the
   mean \var{mean} and standard deviation var{sd}
   \begin{equation}
      skew = (1/N) \sum ((x_i - mean)/sd)^3
   \end{equation}
   These functions are useful if you have already computed the mean and
   standard deviation of \var{data} and want to avoid recomputing them.
\end{funcdesc}


\begin{funcdesc}{kurtosis}{data}
   Compute the kurtosis of \var{data}.  The kurtosis is defined as
   \begin{equation}
      kurtosis = ((1/N) \sum ((x_i - \hat\mu)/\hat\sigma)^4) - 3
   \end{equation}
   The kurtosis measures how sharply peaked a distribution is, relative to its
   width.  The kurtosis is normalized to zero for a gaussian distribution.
\end{funcdesc}


\begin{funcdesc}{kurtosis_m_sd}{data, mean, sd}
   This function computes the kurtosis of the dataset \var{data} using the
   given values of the mean \var{mean} and standard deviation \var{sd}
   \begin{equation}
      kurtosis = ((1/N) \sum ((x_i - mean)/sd)^4) - 3
   \end{equation}
   This function is useful if you have already computed the mean and standard
   deviation of \var{data} and want to avoid recomputing them.
\end{funcdesc}



\subsection{Autocorrelation}
\label{sec:stat:autocorrelation}

\begin{funcdesc}{lag1_autocorrelation}{x}
   Computes the lag-1 autocorrelation of the dataset \var{x}
   \begin{equation}
      a_1 = \frac{\sum^{n}_{i = 1} (x_{i} - \hat\mu) (x_{i-1} - \hat\mu)}{
         \sum^{n}_{i = 1} (x_{i} - \hat\mu) (x_{i} - \hat\mu)}
   \end{equation}
 \end{funcdesc}

\begin{funcdesc}{lag1_autocorrelation_m}{x, mean}
   Computes the lag-1 autocorrelation of the dataset \var{x} using the given
   value of the mean \var{mean}.
   \begin{equation}
      a_1 = \frac{\sum_{i = 1}^{n} (x_{i} - \var{mean}) (x_{i-1} - \var{mean})}{
         \sum^{n}_{i = 1} (x_{i} - \var{mean}) (x_{i} - \var{mean})}
   \end{equation}
\end{funcdesc}



\subsection{Covariance}
\label{sec:stat:covariance}

\begin{funcdesc}{covariance}{x, y}
   Computes the covariance of the datasets \var{x} and \var{y} which must be of
   same length.
   \begin{equation}
      c = \frac{1}{n-1} \sum^{n}_{i=1} (x_i - \hat x) (y_i - \hat y)
   \end{equation}
\end{funcdesc}

\begin{funcdesc}{lag1_autocorrelation_m}{x, y, mean\_x, mean\_y}
   Computes the covariance of the datasets \var{x} and \var{y} using the given
   values of the means \var{mean\_x} and \var{mean\_y}. The datasets \var{x}
   and \var{y} must be of equal length.
   \begin{equation}
      c = \frac{1}{n-1} \sum^{n}_{i=1} (x_i - \var{mean\_x}) (y_i -
      \var{mean\_y})
   \end{equation}
\end{funcdesc}




\subsection{Maximum and Minimum values}
\label{sec:stat:max-min-value}


\begin{funcdesc}{max}{data}
   This function returns the maximum value in \var{data}.  The maximum value is
   defined as the value of the element $x_i$ which satisfies $x_i \ge x_j$ for
   all $j$.
   
   If you want instead to find the element with the largest absolute magnitude
   you will need to apply `fabs' or `abs' to your data before calling this
   function.
\end{funcdesc}

\begin{funcdesc}{min}{data}
   This function returns the minimum value in \var{data}. The maximum value is
   defined as the value of the element $x_i$ which satisfies $x_i \le x_j$ for
   all $j$.
   
   If you want instead to find the element with the smallest absolute magnitude
   you will need to apply `fabs' or `abs' to your data before calling this
   function.
\end{funcdesc}

\begin{funcdesc}{minmax}{data}
   This function returns both the minimum and maximum values of \var{data},
   determined in a single pass.
\end{funcdesc}

\begin{funcdesc}{max_index}{data}
   This function returns the index of the maximum value in \var{data}.  The
   maximum value is defined as the value of the element $x_i$ which satisfies
   $x_i \ge x_j$ for all $j$.  When there are several equal maximum elements
   then the first one is chosen.
\end{funcdesc}

\begin{funcdesc}{min_index}{data}
   This function returns the index of the minimum value in \var{data}.  The
   minimum value is defined as the value of the element $x_i$ which satisfies
   $x_i \le x_j$ for all $j$.  When there are several equal minimum elements
   then the first one is chosen.
\end{funcdesc}

\begin{funcdesc}{minmax_index}{data}
   This function returns the indexes of the minimum and maximum values of
   \var{data}, determined in a single pass.
\end{funcdesc}



\subsection{Median and Percentiles}
\label{sec:stat:median-percentiles}

The median and percentile functions described in this section operate on sorted
data.  For convenience we use "quantiles", measured on a scale of 0 to 1,
instead of percentiles (which use a scale of 0 to 100).

\begin{funcdesc}{median_from_sorted_data}{data}
   This function returns the median value of \var{data}.  The elements of the
   array must be in ascending numerical order.  There are no checks to see
   whether the data are sorted, so the function \function{sort} should always
   be used first.
   
   When the dataset has an odd number of elements the median is the value of
   element (n-1)/2.  When the dataset has an even number of elements the median
   is the mean of the two nearest middle values, elements (n-1)/2 and n/2.
   Since the algorithm for computing the median involves interpolation this
   function always returns a floating-point number, even for integer data
   types.
\end{funcdesc}

\begin{funcdesc}{quantile_from_sorted_data}{data, F}
   This function returns a quantile value of \var{data}.  The elements of the
   array must be in ascending numerical order.  The quantile is determined by
   the \var{F}, a fraction between 0 and 1.  For example, to compute the value
   of the 75th percentile \var{F} should have the value 0.75.
   
   There are no checks to see whether the data are sorted, so the function
   \function{sort} should always be used first.
   
   The quantile is found by interpolation, using the formula
   \begin{equation}
      quantile = (1 - \delta) x_i + \delta x_{i+1}
   \end{equation}
   where $i$ is $floor((n - 1)f)$ and $\delta$ is $(n-1)f - i$.
   
   Thus the minimum value of the array (\var{data[0]}) is given by \var{F}
   equal to zero, the maximum value (\var{data[-1]}) is given by \var{F} equal
   to one and the median value is given by \var{F} equal to 0.5.  Since the
   algorithm for computing quantiles involves interpolation this function
   always returns a floating-point number, even for integer data types.
\end{funcdesc}


\subsection{Weighted Samples}
\label{sec:weighted-samples}

The functions described in this section allow the computation of statistics for
weighted samples.  The functions accept an array of samples, $x_i$, with
associated weights, $w_i$.  Each sample $x_i$ is considered as having been
drawn from a Gaussian distribution with variance $\sigma_i^2$.  The sample
weight $w_i$ is defined as the reciprocal of this variance, $w_i =
1/\sigma_i^2$.  Setting a weight to zero corresponds to removing a sample from
a dataset.

\begin{funcdesc}{wmean}{w, data}
   This function returns the weighted mean of the dataset \var{data} using the
   set of weights \var{w}.  The weighted mean is defined as
   \begin{equation}
      \hat\mu = (\sum w_i x_i) / (\sum w_i)
   \end{equation}
\end{funcdesc}

\begin{funcdesc}{wvariance }{w, data}
   This function returns the estimated variance of the dataset \var{data},
   using the set of weights \var{w}.  The estimated variance of a weighted
   dataset is defined as
   \begin{equation}
      \hat\sigma^2 = ((\sum w_i)/((\sum w_i)^2 - \sum (w_i^2))) \sum w_i (x_i - \hat\mu)^2
   \end{equation}
   Note that this expression reduces to an unweighted variance with the
   familiar $1/(N-1)$ factor when there are $N$ equal non-zero weights.
\end{funcdesc}

\begin{funcdesc}{wvariance_m}{w, data, wmean}
   This function returns the estimated variance of the weighted dataset
   \var{data} using the given weighted mean \var{wmean}.
\end{funcdesc}

\begin{funcdesc}{wsd}{w, data}
   The standard deviation is defined as the square root of the variance.  This
   function returns the square root of the corresponding variance function
   \function{wvariance} above.
\end{funcdesc}

\begin{funcdesc}{wsd_m}{w, data, wmean}
   This function returns the square root of the corresponding variance function
   \function{wvariance_m} above.
\end{funcdesc}

\begin{funcdesc}{wvariance_with_fixed_mean}{w, data, mean}
   This function computes an unbiased estimate of the variance of weighted
   dataset \var{data} when the population mean \var{mean} of the underlying
   distribution is known _a priori_.  In this case the estimator for the
   variance replaces the sample mean $\hat\mu$ by the known population mean
   $\mu$,
   \begin{equation}
      \hat\sigma^2 = (\sum w_i (x_i - \mu)^2) / (\sum w_i)
   \end{equation}
\end{funcdesc}

\begin{funcdesc}{wsd_with_fixed_mean}{w, data, mean}
   The standard deviation is defined as the square root of the variance.  This
   function returns the square root of the corresponding variance function
   above.
\end{funcdesc}

\begin{funcdesc}{wabsdev}{w, data}
   This function computes the weighted absolute deviation from the weighted
   mean of \var{data}.  The absolute deviation from the mean is defined as
   \begin{equation}
      absdev = (\sum w_i |x_i - \hat\mu|) / (\sum w_i)
   \end{equation}
\end{funcdesc}

\begin{funcdesc}{wabsdev_m}{w, data, wmean}
   This function computes the absolute deviation of the weighted dataset DATA
   about the given weighted mean WMEAN.
\end{funcdesc}

\begin{funcdesc}{wskew}{w, data}
   This function computes the weighted skewness of the dataset DATA.
   \begin{equation}
      skew = (\sum w_i ((x_i - xbar)/\sigma)^3) / (\sum w_i)
   \end{equation}
\end{funcdesc}

\begin{funcdesc}{wskew_m_sd}{w, data, mean, wsd}
   This function computes the weighted skewness of the dataset \var{data} using
   the given values of the weighted mean and weighted standard deviation,
   \var{wmean} and \var{wsd}.
\end{funcdesc}

\begin{funcdesc}{wkurtosis}{w, data}
   This function computes the weighted kurtosis of the dataset \var{data}. The
   kurtosis is defined as 
   \begin{equation}
      kurtosis = ((\sum w_i ((x_i - xbar)/sigma)^4) / (\sum w_i)) - 3
   \end{equation}
\end{funcdesc}

\begin{funcdesc}{wkurtosis_m_sd}{w, data, mean, wsd}
   This function computes the weighted kurtosis of the dataset \var{data} using
   the given values of the weighted mean and weighted standard deviation,
   \var{wmean} and \var{wsd}.
\end{funcdesc}





\section{Further Reading}
\label{sec:stat:further-reading}

See the \gsl{} reference manual for a description of all available functions
and the calculations they perform.

The standard reference for almost any topic in statistics is the multi-volume
\emph{Advanced Theory of Statistics} by Kendall and Stuart.  Many statistical
concepts can be more easily understood by a Bayesian approach.  The book by
Gelman, Carlin, Stern and Rubin gives a comprehensive coverage of the subject.
For physicists the Particle Data Group provides useful reviews of Probability
and Statistics in the "Mathematical Tools" section of its Annual Review of
Particle Physics.
   
\begin{seealso}
   \seetext{Maurice Kendall, Alan Stuart, and J.\ Keith Ord: \emph{The Advanced
         Theory of Statistics} (multiple volumes) reprinted as \emph{Kendall's
         Advanced Theory of Statistics}.  Wiley, ISBN 047023380X.}
   
   \seetext{Andrew Gelman, John B.\ Carlin, Hal S.\ Stern, Donald B.\ Rubin:
      \emph{Bayesian Data Analysis}.  Chapman \& Hall, ISBN 0412039915.}
   
   \seetext{R.M.\ Barnett et al: Review of Particle Properties. \emph{Physical
         Review} \textbf{D54}, 1 (1996).}
   
   \seetext{D.E.\ Groom et al., \emph{The European Physical Journal}
      \textbf{C15}, 1 (2000) and \emph{2001 off-year partial update for the
         2002 edition} available on the PDG WWW pages (URL:
      \url{http://pdg.lbl.gov/}).}
   
   \seetext{Siegmund Brandt: \emph{Datenanalyse}, 4th ed. 1999, Spektrum,
      Heidelberg, ISBN 3827401585.}  
   
   \seetext{Siegmund Brandt: \emph{Data Analysis}. 3rd ed. 1998, Springer,
      Berlin, ISBN 0387984984.}
\end{seealso}


%% Local Variables:
%% mode: LaTeX
%% mode: auto-fill
%% fill-column: 79
%% indent-tabs-mode: nil
%% ispell-dictionary: "british"
%% reftex-fref-is-default: nil
%% TeX-auto-save: t
%% TeX-command-default: "pdfeLaTeX"
%% TeX-master: "pygsl"
%% TeX-parse-self: t
%% End:


[common]
sensorid = default

[virtual_file_system]
data_fs_url = default
fs_url = default

[session]
timeout = 30

[daemon]
;user = conpot
;group = conpot

[json]
enabled = False
filename = /var/log/conpot.json

[sqlite]
enabled = False

[mysql]
enabled = False
device = /tmp/mysql.sock
host = localhost
port = 3306
db = conpot
username = conpot
passphrase = conpot
socket = tcp        ; tcp (sends to host:port), dev (sends to mysql device/socket file)

[syslog]
enabled = False
device = /dev/log
host = localhost
port = 514
facility = local0
socket = dev        ; udp (sends to host:port), dev (sends to device)

[hpfriends]
enabled = False
host = hpfriends.honeycloud.net
port = 20000
ident = 3Ykf9Znv
secret = 4nFRhpm44QkG9cvD
channels = ["conpot.events", ]

[taxii]
enabled = False
host = taxiitest.mitre.org
port = 80
inbox_path = /services/inbox/default/
use_https = False

[fetch_public_ip]
enabled = True
urls = ["http://whatismyip.akamai.com/", "http://wgetip.com/"]

[change_mac_addr]
enabled = False
iface = eth0
addr = 00:de:ad:be:ef:00


\appendix

\chapter[\protect\module{pygsl.ieee} --- Floating Point Unit Support]
{\protect\module{pygsl.ieee} \\ Floating Point Unit Support}
\label{cha:ieee-module}
\declaremodule{extension}{pygsl.ieee}
\moduleauthor{Achim G\"adke}{achimgaedke@users.sourceforge.net}

This chapter lists features to configure the ``Floating Point Unit'' of your machine.
The exact behaviour of your Floating Point Unit can't be discussed here in general --- its just machine type dependent.

\begin{funcdesc} {set_mode}{int precision, int rounding, int exception\_mask}
the mode has effect on the behaviour during calcualtion, e.g. division by zero or rounding.

The following constants are used as precision argument:
\begin{tableii}{l|l}{constant}{mode value}{definition via gsl}
\lineii{single\_precision}{\code{GSL\_IEEE\_SINGLE\_PRECISION}}
\lineii{double\_precision}{\code{GSL\_IEEE\_DOUBLE\_PRECISION}}
\lineii{extended\_precision}{\code{GSL\_IEEE\_EXTENDED\_PRECISION}}
\end{tableii}
Possible round arguments are:
\begin{tableii}{l|l}{constant}{mode value}{definition via gsl}
\lineii{round\_to\_nearest}{\code{GSL\_IEEE\_ROUND\_TO\_NEAREST}}
\lineii{round\_down}{\code{GSL\_IEEE\_ROUND\_DOWN}}
\lineii{round\_up}{\code{GSL\_IEEE\_ROUND\_UP}}
\lineii{round\_to\_zero}{\code{GSL\_IEEE\_ROUND\_TO\_ZERO}}
\end{tableii}
These exception arguments can be added.
\constant{mask\_all} is the sum of all 5 \constant{mask\_*} constants.
\begin{tableii}{l|l}{constant}{mode value}{definition via gsl}
\lineii{mask\_invalid}{\code{GSL\_IEEE\_MASK\_INVALID}}
\lineii{mask\_denormalized}{\code{GSL\_IEEE\_MASK\_DENORMALIZED}}
\lineii{mask\_division\_by\_zero}{\code{GSL\_IEEE\_MASK\_DIVISION\_BY\_ZERO}}
\lineii{mask\_overflow}{\code{GSL\_IEEE\_MASK\_OVERFLOW}}
\lineii{mask\_underflow}{\code{GSL\_IEEE\_MASK\_UNDERFLOW}}
\lineii{mask\_all}{\code{GSL\_IEEE\_MASK\_ALL}}
\lineii{trap\_inexact}{\code{GSL\_IEEE\_TRAP\_INEXACT}}
\end{tableii}
\end{funcdesc}

\begin{funcdesc} {env\_setup}{}
sets the ieee mode from \envvar{GSL\_IEEE\_MODE}. This is not called any more
automatically  when importing the  \module{pygsl}.
\end{funcdesc}

\begin{funcdesc} {bin\_repr}{float value}
%\cfunction{gsl_ieee_double_to_rep}
returns the binary representation as tuple with the following contents:
\code{(int sign, string mantissa, int exponent, int type)}
These values are used as \constant{type} in \function{bin\_repr}:
\begin{tableii}{l|l}{constant}{type value}{definition via gsl}
\lineii{type\_nan}{\code{GSL\_IEEE\_TYPE\_NAN}}
\lineii{type\_inf}{\code{GSL\_IEEE\_TYPE\_INF}}
\lineii{type\_normal}{\code{GSL\_IEEE\_TYPE\_NORMAL}}
\lineii{type\_denormal}{\code{GSL\_IEEE\_TYPE\_DENORMAL}}
\lineii{type\_zero}{\code{GSL\_IEEE\_TYPE\_ZERO}}
\end{tableii}
\end{funcdesc}

\begin{funcdesc}{isnan}{float value}
determines if the argument is not a valid number
\end{funcdesc}

\begin{funcdesc}{nan}{}
generates a ``not-a-number'' value. This is implemented as function, because of the potential exception generation by your floating-point unit.
\end{funcdesc}

\begin{funcdesc}{isinf}{float value}
returns -1 if the argument represents a negative infinite value and +1 if positive, 0 otherwise
\end{funcdesc}

\begin{funcdesc}{posinf}{}
gives you the representation of ``positive infinity''
\end{funcdesc}

\begin{funcdesc}{neginf}{}
the same as posinf, but negative
\end{funcdesc}

\begin{funcdesc}{finite}{float value}
results in 1 if the value is finite, 0 if it is not a number or infinite
\end{funcdesc}


%\chapter[\protect\module{pygsl.init} --- Library initialisation]
%{\protect\module{pygsl.init} \\ Library initialisation}
%\label{cha:library-initialisation}
%\declaremodule{extension}{pygsl.init}
\moduleauthor{Pierre Schnizer}{schnizer@users.sourceforge.net}
\moduleauthor{Achim G\"adke}{achimgaedke@users.sourceforge.net}

This module is called the first time when loading \module{pygsl}.
All following procedures are called once and before everything other.

\section{Exception handling}
\index{exception handling!initialisation} GSL provides a selectable error
handler, that is called for occuring errors (like domain errors, division by
zero, etc. ).  This is switched off. Instead each wrapper function will check
the error return value and in case of error an python exception is created. 

Here is a python level example:
\begin{verbatim}
import pygsl.histogram
import pygsl.errors
hist=pygsl.histogram.histogram2d(100,100)
try:
   hist[-1,-1]=0
except pygsl.errors.gsl_Error,err:
   print err
\end{verbatim}
Will result
\begin{verbatim}
input domain error: index i lies outside valid range of 0 .. nx - 1
\end{verbatim}


An exception are ufuncs in the testings.sf module (see section\ref{sec:ufuncs}).

%\module{pygsl.init} installs a handler by calling
%\cfunction{gsl_set_error_handler} to set an appropiate exception from
%\module{pygsl.errors}.  A \module{pygsl} interface function should return
%\code{NULL} in case of an error, so the exception is raised.  If this handler
%is called more than once before returning to python, only the first set
%exception is raised.
%
%
% 
% \section{IEEE-mode}
% \index{ieee-mode!initialisation}
% The IEEE mode is set from the environment variable
%  \envvar{GSL_IEEE_MODE} via \cfunction{gsl_ieee_env_setup()}.
% After the initialisation use \module{pygsl.ieee} for manipulation.
% 
% \section{random number generators}
% \index{random number generator!initialisation}
% Also the random number generator can be initialised from the environment variables
%  \envvar{GSL_RNG_TYPE}
% and \envvar{GSL_RNG_SEED} using the gsl function \cfunction{gsl_rng_env_setup()}.
% Each random number generators are initialised with \envvar{GSL_RNG_SEED}.
% 
% The default generator can be created by:\nopagebreak
% \begin{verbatim}
% import pygsl.rng
% my_rng=pygsl.rng.rng()
% print my_rng.name()
% \end{verbatim}




\chapter{GNU Free Documentation License}
\label{cha:free-documentation-license}

Version 1.1, March 2000\\

 Copyright \copyright\ 2000  Free Software Foundation, Inc.\\
     59 Temple Place, Suite 330, Boston, MA  02111-1307  USA\\
 Everyone is permitted to copy and distribute verbatim copies
 of this license document, but changing it is not allowed.

\section*{Preamble}

The purpose of this License is to make a manual, textbook, or other
written document ``free'' in the sense of freedom: to assure everyone
the effective freedom to copy and redistribute it, with or without
modifying it, either commercially or noncommercially.  Secondarily,
this License preserves for the author and publisher a way to get
credit for their work, while not being considered responsible for
modifications made by others.

This License is a kind of ``copyleft'', which means that derivative
works of the document must themselves be free in the same sense.  It
complements the GNU General Public License, which is a copyleft
license designed for free software.

We have designed this License in order to use it for manuals for free
software, because free software needs free documentation: a free
program should come with manuals providing the same freedoms that the
software does.  But this License is not limited to software manuals;
it can be used for any textual work, regardless of subject matter or
whether it is published as a printed book.  We recommend this License
principally for works whose purpose is instruction or reference.

\section{Applicability and Definitions}

This License applies to any manual or other work that contains a
notice placed by the copyright holder saying it can be distributed
under the terms of this License.  The ``Document'', below, refers to any
such manual or work.  Any member of the public is a licensee, and is
addressed as ``you''.

A ``Modified Version'' of the Document means any work containing the
Document or a portion of it, either copied verbatim, or with
modifications and/or translated into another language.

A ``Secondary Section'' is a named appendix or a front-matter section of
the Document that deals exclusively with the relationship of the
publishers or authors of the Document to the Document's overall subject
(or to related matters) and contains nothing that could fall directly
within that overall subject.  (For example, if the Document is in part a
textbook of mathematics, a Secondary Section may not explain any
mathematics.)  The relationship could be a matter of historical
connection with the subject or with related matters, or of legal,
commercial, philosophical, ethical or political position regarding
them.

The ``Invariant Sections'' are certain Secondary Sections whose titles
are designated, as being those of Invariant Sections, in the notice
that says that the Document is released under this License.

The ``Cover Texts'' are certain short passages of text that are listed,
as Front-Cover Texts or Back-Cover Texts, in the notice that says that
the Document is released under this License.

A ``Transparent'' copy of the Document means a machine-readable copy,
represented in a format whose specification is available to the
general public, whose contents can be viewed and edited directly and
straightforwardly with generic text editors or (for images composed of
pixels) generic paint programs or (for drawings) some widely available
drawing editor, and that is suitable for input to text formatters or
for automatic translation to a variety of formats suitable for input
to text formatters.  A copy made in an otherwise Transparent file
format whose markup has been designed to thwart or discourage
subsequent modification by readers is not Transparent.  A copy that is
not ``Transparent'' is called ``Opaque''.

Examples of suitable formats for Transparent copies include plain
ASCII without markup, Texinfo input format, \LaTeX~input format, SGML
or XML using a publicly available DTD, and standard-conforming simple
HTML designed for human modification.  Opaque formats include
PostScript, PDF, proprietary formats that can be read and edited only
by proprietary word processors, SGML or XML for which the DTD and/or
processing tools are not generally available, and the
machine-generated HTML produced by some word processors for output
purposes only.

The ``Title Page'' means, for a printed book, the title page itself,
plus such following pages as are needed to hold, legibly, the material
this License requires to appear in the title page.  For works in
formats which do not have any title page as such, ``Title Page'' means
the text near the most prominent appearance of the work's title,
preceding the beginning of the body of the text.


\section{Verbatim Copying}

You may copy and distribute the Document in any medium, either
commercially or noncommercially, provided that this License, the
copyright notices, and the license notice saying this License applies
to the Document are reproduced in all copies, and that you add no other
conditions whatsoever to those of this License.  You may not use
technical measures to obstruct or control the reading or further
copying of the copies you make or distribute.  However, you may accept
compensation in exchange for copies.  If you distribute a large enough
number of copies you must also follow the conditions in section 3.

You may also lend copies, under the same conditions stated above, and
you may publicly display copies.


\section{Copying in Quantity}

If you publish printed copies of the Document numbering more than 100,
and the Document's license notice requires Cover Texts, you must enclose
the copies in covers that carry, clearly and legibly, all these Cover
Texts: Front-Cover Texts on the front cover, and Back-Cover Texts on
the back cover.  Both covers must also clearly and legibly identify
you as the publisher of these copies.  The front cover must present
the full title with all words of the title equally prominent and
visible.  You may add other material on the covers in addition.
Copying with changes limited to the covers, as long as they preserve
the title of the Document and satisfy these conditions, can be treated
as verbatim copying in other respects.

If the required texts for either cover are too voluminous to fit
legibly, you should put the first ones listed (as many as fit
reasonably) on the actual cover, and continue the rest onto adjacent
pages.

If you publish or distribute Opaque copies of the Document numbering
more than 100, you must either include a machine-readable Transparent
copy along with each Opaque copy, or state in or with each Opaque copy
a publicly-accessible computer-network location containing a complete
Transparent copy of the Document, free of added material, which the
general network-using public has access to download anonymously at no
charge using public-standard network protocols.  If you use the latter
option, you must take reasonably prudent steps, when you begin
distribution of Opaque copies in quantity, to ensure that this
Transparent copy will remain thus accessible at the stated location
until at least one year after the last time you distribute an Opaque
copy (directly or through your agents or retailers) of that edition to
the public.

It is requested, but not required, that you contact the authors of the
Document well before redistributing any large number of copies, to give
them a chance to provide you with an updated version of the Document.


\section{Modifications}

You may copy and distribute a Modified Version of the Document under
the conditions of sections 2 and 3 above, provided that you release
the Modified Version under precisely this License, with the Modified
Version filling the role of the Document, thus licensing distribution
and modification of the Modified Version to whoever possesses a copy
of it.  In addition, you must do these things in the Modified Version:

\begin{itemize}

\item Use in the Title Page (and on the covers, if any) a title distinct
   from that of the Document, and from those of previous versions
   (which should, if there were any, be listed in the History section
   of the Document).  You may use the same title as a previous version
   if the original publisher of that version gives permission.
\item List on the Title Page, as authors, one or more persons or entities
   responsible for authorship of the modifications in the Modified
   Version, together with at least five of the principal authors of the
   Document (all of its principal authors, if it has less than five).
\item State on the Title page the name of the publisher of the
   Modified Version, as the publisher.
\item Preserve all the copyright notices of the Document.
\item Add an appropriate copyright notice for your modifications
   adjacent to the other copyright notices.
\item Include, immediately after the copyright notices, a license notice
   giving the public permission to use the Modified Version under the
   terms of this License, in the form shown in the Addendum below.
\item Preserve in that license notice the full lists of Invariant Sections
   and required Cover Texts given in the Document's license notice.
\item Include an unaltered copy of this License.
\item Preserve the section entitled ``History'', and its title, and add to
   it an item stating at least the title, year, new authors, and
   publisher of the Modified Version as given on the Title Page.  If
   there is no section entitled ``History'' in the Document, create one
   stating the title, year, authors, and publisher of the Document as
   given on its Title Page, then add an item describing the Modified
   Version as stated in the previous sentence.
\item Preserve the network location, if any, given in the Document for
   public access to a Transparent copy of the Document, and likewise
   the network locations given in the Document for previous versions
   it was based on.  These may be placed in the ``History'' section.
   You may omit a network location for a work that was published at
   least four years before the Document itself, or if the original
   publisher of the version it refers to gives permission.
\item In any section entitled ``Acknowledgements'' or ``Dedications'',
   preserve the section's title, and preserve in the section all the
   substance and tone of each of the contributor acknowledgements
   and/or dedications given therein.
\item Preserve all the Invariant Sections of the Document,
   unaltered in their text and in their titles.  Section numbers
   or the equivalent are not considered part of the section titles.
\item Delete any section entitled ``Endorsements''.  Such a section
   may not be included in the Modified Version.
\item Do not retitle any existing section as ``Endorsements''
   or to conflict in title with any Invariant Section.

\end{itemize}

If the Modified Version includes new front-matter sections or
appendices that qualify as Secondary Sections and contain no material
copied from the Document, you may at your option designate some or all
of these sections as invariant.  To do this, add their titles to the
list of Invariant Sections in the Modified Version's license notice.
These titles must be distinct from any other section titles.

You may add a section entitled ``Endorsements'', provided it contains
nothing but endorsements of your Modified Version by various
parties -- for example, statements of peer review or that the text has
been approved by an organization as the authoritative definition of a
standard.

You may add a passage of up to five words as a Front-Cover Text, and a
passage of up to 25 words as a Back-Cover Text, to the end of the list
of Cover Texts in the Modified Version.  Only one passage of
Front-Cover Text and one of Back-Cover Text may be added by (or
through arrangements made by) any one entity.  If the Document already
includes a cover text for the same cover, previously added by you or
by arrangement made by the same entity you are acting on behalf of,
you may not add another; but you may replace the old one, on explicit
permission from the previous publisher that added the old one.

The author(s) and publisher(s) of the Document do not by this License
give permission to use their names for publicity for or to assert or
imply endorsement of any Modified Version.


\section{Combining Documents}

You may combine the Document with other documents released under this
License, under the terms defined in section 4 above for modified
versions, provided that you include in the combination all of the
Invariant Sections of all of the original documents, unmodified, and
list them all as Invariant Sections of your combined work in its
license notice.

The combined work need only contain one copy of this License, and
multiple identical Invariant Sections may be replaced with a single
copy.  If there are multiple Invariant Sections with the same name but
different contents, make the title of each such section unique by
adding at the end of it, in parentheses, the name of the original
author or publisher of that section if known, or else a unique number.
Make the same adjustment to the section titles in the list of
Invariant Sections in the license notice of the combined work.

In the combination, you must combine any sections entitled ``History''
in the various original documents, forming one section entitled
``History''; likewise combine any sections entitled ``Acknowledgements'',
and any sections entitled ``Dedications''.  You must delete all sections
entitled ``Endorsements.''


\section{Collections of Documents}

You may make a collection consisting of the Document and other documents
released under this License, and replace the individual copies of this
License in the various documents with a single copy that is included in
the collection, provided that you follow the rules of this License for
verbatim copying of each of the documents in all other respects.

You may extract a single document from such a collection, and distribute
it individually under this License, provided you insert a copy of this
License into the extracted document, and follow this License in all
other respects regarding verbatim copying of that document.



\section{Aggregation With Independent Works}

A compilation of the Document or its derivatives with other separate
and independent documents or works, in or on a volume of a storage or
distribution medium, does not as a whole count as a Modified Version
of the Document, provided no compilation copyright is claimed for the
compilation.  Such a compilation is called an ``aggregate'', and this
License does not apply to the other self-contained works thus compiled
with the Document, on account of their being thus compiled, if they
are not themselves derivative works of the Document.

If the Cover Text requirement of section 3 is applicable to these
copies of the Document, then if the Document is less than one quarter
of the entire aggregate, the Document's Cover Texts may be placed on
covers that surround only the Document within the aggregate.
Otherwise they must appear on covers around the whole aggregate.


\section{Translation}

Translation is considered a kind of modification, so you may
distribute translations of the Document under the terms of section 4.
Replacing Invariant Sections with translations requires special
permission from their copyright holders, but you may include
translations of some or all Invariant Sections in addition to the
original versions of these Invariant Sections.  You may include a
translation of this License provided that you also include the
original English version of this License.  In case of a disagreement
between the translation and the original English version of this
License, the original English version will prevail.


\section{Termination}

You may not copy, modify, sublicense, or distribute the Document except
as expressly provided for under this License.  Any other attempt to
copy, modify, sublicense or distribute the Document is void, and will
automatically terminate your rights under this License.  However,
parties who have received copies, or rights, from you under this
License will not have their licenses terminated so long as such
parties remain in full compliance.


\section{Future Revisions of This License}

The Free Software Foundation may publish new, revised versions
of the GNU Free Documentation License from time to time.  Such new
versions will be similar in spirit to the present version, but may
differ in detail to address new problems or concerns. See
http://www.gnu.org/copyleft/.

Each version of the License is given a distinguishing version number.
If the Document specifies that a particular numbered version of this
License "or any later version" applies to it, you have the option of
following the terms and conditions either of that specified version or
of any later version that has been published (not as a draft) by the
Free Software Foundation.  If the Document does not specify a version
number of this License, you may choose any version ever published (not
as a draft) by the Free Software Foundation.


% Complete documentation on the extended LaTeX markup used for Python
% documentation is available in ``Documenting Python'', which is part
% of the standard documentation for Python.  It may be found online
% at:
%
%     http://www.python.org/doc/current/doc/doc.html

\documentclass[hyperref]{manual}

% latex2html doesn't know [T1]{fontenc}, so we cannot use that:(
\usepackage{amsmath}
\usepackage[latin1]{inputenc}
\usepackage{textcomp}
\usepackage{hyperref}

% this version does not reset module names at section level
%begin{latexonly}
\makeatletter
\let\py@OldOldChapter=\chapter
\renewcommand{\chapter}{\py@reset%
                        \py@OldOldChapter}
\renewcommand{\section}{\@startsection{section}{1}{\z@}%
   {-3.5ex \@plus -1ex \@minus -.2ex}%
   {2.3ex \@plus.2ex}%
   {\reset@font\Large\py@HeaderFamily}}
\makeatother
%end{latexonly}


% some convenience declarations
\newcommand{\gsl}{GSL}
\newcommand{\GSL}{GNU Scientific Library}
\newcommand{\numpy}{NumPy}
\newcommand{\NUMPY}{Numerical Python}
\newcommand{\pygsl}{PyGSL}
\newcommand{\PYGSL}{PyGSL: Python wrapper of the GNU Scientific Library}

\makeatletter
\newenvironment{pytypedesc}[2]{
  % Using \renewcommand doesn't work for this, for unknown reasons:
  \global\def\py@thisclass{#1}
  \begin{fulllineitems}
    \py@sigline{\strong{pytype }\bfcode{#1}}{#2}%
    \index{#1@{\py@idxcode{#1}} (pytype in \py@thismodule)}
}{\end{fulllineitems}}
\makeatother


\title{PyGSL Reference Manual}

\ifhtml
\author{
  \ulink{Achim G\"adke}{mailto:achimgaedke@users.sourceforge.net}\\
  Technische Universit\"at Darmstadt, Darmstadt, Germany
}
\author{
  \ulink{Pierre Schnizer}{mailto:schnizer@users.sourceforge.net}\\
  Gesellschaft f\"ur Schwerionenforschung, Darmstadt, Germany
}
%\author{
%  \ulink{Jochen K\"upper}{mailto:jochen@jochen-kuepper.de}\\
%  Fritz-Haber-Institut der MPG, Berlin, Germany
%}
%\author{
%  \ulink{S\'ebastien Maret}{mailto:schnizer@users.sourceforge.net}\\
%  Department of Astronomy, University of Michigan, Ann Arbor, USA
%}
\else
%begin{latexonly}
%% pdfelatex (TeXLive 7) doesn't handle \footnotemark in here...
\author{Achim G\"adke \\ 
          Jochen K\"upper \\ 
         %S\'ebastien Maret \\
        Pierre Schnizer}
% Please at least include a long-lived email address!
\authoraddress{
   Technische Universit\"at Darmstadt, Darmstadt, Germany \\
   \email{achimgaedke@users.sourceforge.net} \\
   Gesellschaft f\"ur Schwerionenforschung, Darmstadt, Germany \\
   \email{schnizer@users.sourceforge.net} \\
%   Fritz-Haber-Institut der MPG, Berlin, Germany \\
%   \email{jochen@jochen-kuepper.de} \\
%   Department of Astronomy, University of Michigan, Ann Arbor, USA \\
%   \email{bmaret@users.sourceforge.et} \\
}
%end{latexonly}
\fi

\date{October, 2008}            % update before release!
                                % Use an explicit date so that reformatting
                                % doesn't cause a new date to be used.  Setting
                                % the date to \today can be used during draft
                                % stages to make it easier to handle versions.
\release{0.9}                   % release version; this is used to define the
\setshortversion{0.9}           % \version macro
\makeindex                      % tell \index to actually write the .idx file


\begin{document}

\maketitle

% This makes the contents more accessible from the front page of the HTML.
\ifhtml
\chapter*{Front Matter}
\label{front}
\fi

Copyright \copyright{} 2002,2005 The pygsl Team.

Permission is granted to copy, distribute and/or modify this document under the
terms of the GNU Free Documentation License, Version 1.1 or any later version
published by the Free Software Foundation; with no Invariant Sections, no
Front-Cover Texts, and no Back-Cover Texts.  A copy of the license is included
in section \ref{cha:free-documentation-license} entitled ``GNU Free
Documentation License''.


%% Local Variables:
%% mode: LaTeX
%% mode: auto-fill
%% fill-column: 79
%% indent-tabs-mode: nil
%% ispell-dictionary: "american"
%% reftex-fref-is-default: nil
%% TeX-auto-save: t
%% TeX-command-default: "pdfeLaTeX"
%% TeX-master: "pygsl"
%% TeX-parse-self: t
%% End:


\begin{abstract}
   \noindent
   pygsl grants python users access to the GNU scientific library.  The latest
   version can be found at the project homepage, \url{http://pygsl.sf.net}.

   \textbf{Implemented features:} \\
   \begin{tabular}{ll}
     \module{pygsl.blas}                & basic linear algebra system\\
     \module{pygsl.chebyshev}           & chebyshev approximations\\
     \module{pygsl.combination}         & combinations  \\
     \module{pygsl.const}               & $>200$ often used mathematical and
                                          scientific constants. \\
     \module{pygsl.diff}                & (Deprecated. Use pygsl.deriv instead). \\
     \module{pygsl.deriv}               & Numerical differentiation. \\
     \module{pygsl.eigen}               &\\
     \module{pygsl.fit}                 &\\
     \module{pygsl.histogram}          & 1d and 2d histograms and operations
                                          on histograms. \\
     \module{pygsl.ieee}                & Access to the ieee-arithmetics layer
                                          of gsl. \\ 
     \module{pygsl.integrate}           &\\
     \module{pygsl.interpolation}       &\\ 
     \module{pygsl.linalg}              &\\
     \module{pygsl.math}                &\\
     \module{pygsl.monte}               &\\
     \module{pygsl.minimize}            &\\
     \module{pygsl.multifit}            &\\
     \module{pygsl.multifit_nlin}       &\\    
     \module{pygsl.multimin}            &\\
     \module{pygsl.multiroots}          &\\ 
     \module{pygsl.odeiv}               &\\
     \module{pygsl.permutation}         &\\  
     \module{pygsl.poly}                &\\
     \module{pygsl.qrng}                &\\     
     \module{pygsl.rng}                 & random number generators and probability densities. \\
     \module{pygsl.roots}               &\\
     \module{pygsl.siman}               &Simulated anealing\\
     \module{pygsl.sum}                 & \\
     \module{pygsl.sf}                  & $>200$ special functions. \\
     \module{pygsl.statistics}          & Statistical functions. \\
   \end{tabular}
\end{abstract}


\tableofcontents


\chapter{System Requirements, Installation}
\label{cha:system-req-installation}
\section{Status}

\paragraph*{Status of GSL-Library}
The gsl-library is since version 1.0 stable and for general use.
More information about it at \url{http://www.gnu.org/software/gsl/}.

\paragraph*{Status of this interface}
Nearly all modules are wrapped. A lot of tests are
covering various functionality. Please report to the mailing list
\url{pygsl-discuss@lists.sourceforge.net} if you find a bug.

The hankel modules have been
wrapped. Please write to the mailing list
\url{pygsl-discuss@lists.sourceforge.net} 
if you require one of the modules
and are willing to help with a simple example. 
If any other function is missing or some other module (e.g. ntuple) or
function, do not hesitate to write to the list.

\paragraph*{Retriving the Interface}
You can download it here: \url{http://sourceforge.net/projects/pygsl}

\section{Requirements}

To build the interface, you will need
\begin{itemize}
\item \ulink{gsl-1.x}{http://sources.redhat.com/gsl},
\item \ulink{python2.6}{http://www.python.org} or better,
\item \ulink{NumPy}{http://numpy.sf.net}, and
\item a c compiler (like \ulink{gcc}{http://gcc.gnu.org}).
\end{itemize}

Supported Platforms are:
\begin{itemize}
\item Linux (Redhat/Debian/SuSE) with python2.* and gsl-1.*
\item Win32
\end{itemize}
It was tested and is tested on an irregular basis on the following platforms
\begin{itemize}
\item SUN
\item Cygwin
\item MacOS X
\end{itemize}
but is supposed to build on any POSIX platforms.

\section{Installing the pygsl interface}

\program{gsl-config} must be on your path:\nopagebreak
\begin{verbatim}
# unpack the source distribution
gzip -d -c pygsl-x.y.z.tar.gz|tar xvf-
cd pygsl-x.y.z
# do this with your prefered python version
# to set the gsl location explicitly use setup.py --gsl-prefix=/path/to/gsl
python setup.py build
# change to an user id, that is allowed to do installation
python setup.py install
\end{verbatim}
Ready....

{\bf Do not test the interface in the distribution root or in the directories
 \file{src} or \file{pygsl}.}

If you find unresolved symbols later on, delete the C source in the
swig_src files. Check that swig can be called from the command line. 
Then start the build process again. 

In this case swig will rebuild the C files. The swig_src files
distributed with pygsl are to an up to date version of GSL (1.16 as of
this writing). Swig parses partly some header header files and builds
the appropriate interface functions. If you have an older GSL version 
locally installed, the sources in the swig_src directory can contain 
links to symbols which are not defined by the locally installed GSL
version.

\subsection{Building on win32}

Windows by default does not allow to run a posix shell. Here a different path
is required. First change into the directory \file{gsl_dist}. Copy the file 
\file{gsl_site_example.py}
and edit it to reflect your installation of GSL and SWIG if you want to run it
yourself. The pygsl windows binaries distributed over 
\url{http://sourceforge.net/projects/pygsl/} are built using the mingw32 
compiler. 

\paragraph*{Uninstall GSL interface}
\code{rm -r }"python install path"\code{/lib/python}"version"\code{/site-packages/pygsl}

\paragraph*{Testing}
the directory \file{tests} contains several testsuites, based on python
\module{unittest}.
The script \file{run_test.py} in this directory will run one after the other.

\paragraph*{Support}
Please send mails to our mailinglist at
\email{pygsl-discuss@lists.sourceforge.net}.

\paragraph*{Developement}
You can browse our cvs tree at
\url{http://cvs.sourceforge.net/cgi-bin/viewcvs.cgi/pygsl/pygsl/}.
\\
Type this to check out the actual version:
\begin{verbatim}
cvs -d:pserver:anonymous@cvs.pygsl.sourceforge.net:/cvsroot/pygsl login
#Hit return for no password.
cvs -z3 -d:pserver:anonymous@cvs.pygsl.sourceforge.net:/cvsroot/pygsl co pygsl
\end{verbatim}
The script \program{tools/extract_tool.py} generates most of the special 
function code.

%\input{install_advanced.tex}
\paragraph*{ToDo}
Implement other parts:


\paragraph*{History}
\begin{itemize}
\item a gsl-interface for python was needed for a project at
\ulink{Center for Applied Informatics Cologne}{http://www.zaik.uni-koeln.de/AFS}.
\item \file{gsl-0.0.3} was released at May 23, 2001
\item \file{gsl-0.0.4} was released at January 8, 2002
\item \file{gsl-0.0.5} is growing since January, 2002
\item \file{gsl-0.2.0} was released at 
\item \file{gsl-0.3.0} was released at 
\item \file{gsl-0.3.1} was released at 
\item \file{gsl-0.3.2} was released at 
\item \file{gsl-0.9.4} was released at 25. October 2008
\end{itemize}

\paragraph*{Thanks}
Jochen K\"upper (\email{jochen@jochen-kuepper.de}) for 
\module{pygsl.statistics} part\\
Fabian Jakobs for \module{pygsl.blas}, \module{pygsl.eigen}
\module{pygsl.linalg}, \module{pygsl.permutation}\\ 
Leonardo Milano for rpm build\\
Eric Gurrola and  Peter Stoltz for testing and supporting the port of pygsl to
the MAC\\
Sebastien Maret for supporting the Fink \url{http://fink.sourceforge.net}
port of pygsl.


\paragraph*{Maintainers}
Achim G\"adke (\email{AchimGaedke@users.sourceforge.net}),\\
Pierre Schnizer (\email{schnizer@users.sourceforge.net})

\input{installadvanced.tex}
\chapter{Design of the \pygsl{} interface}

The GSL library was implemented using the C language. This implies that 
each function uses a certain type for the different variables and are fixed
 to one specific type. The wrapper will try to convert each argument to the approbriate
C type. 
The \pygsl{} interface
tries to follow it as much as possible but only as far as resonable. 
For example the definition of the poly_eval function in C is given by
\begin{funcdesc}{\texttt{double} gsl_poly_eval}
                {\texttt{const double} C[], \texttt{const int} LEN, \texttt{const double} X}
\end{funcdesc}

The corresponding python wrapper was implemented by
\begin{funcdesc}{poly.poly_eval}{C, x}
\end{funcdesc}
as the wrapper can get the length of any python object and then fill the len variable. 
The mathematical calculation is performed by the GSL library. Thus the calculation is limited 
to the precision provided by the underlying hardware.

Default arguments are used to allocate workspaces on the fly if not provided by the user. 
Consider for example the fft module. The function for the real forward transform is
named 

\begin{funcdesc}{\texttt{int} gsl_fft_real_transform}
{\texttt{double DATA[]}, 
 \texttt{size_t STRIDE},
 \texttt{size_t N}, 
 \texttt{const gsl_fft_real_wavetable * WAVETABLE},
 \texttt{gsl_fft_real_workspace * WORK}
}
\end{funcdesc}

The corresponding python wrapper is found in the fft module called
real_transform
\begin{funcdesc}{real_transform}
{data, \optional{space, 
    table, 
    output}}
\end{funcdesc}
The wrapper will get the stride and size information from the data object provided
by the user. If space or table are not provided, these objects will be generated on 
the fly. As the GSL function applies the transformation in space, an internal copy is 
made of the data and only then the object is passed to the \gsl{} function. If an output
object is provided the data will be copied there instead. \pygsl{} will always make copies
of objects which would be otherwise modified in place.

\section{Callbacks}

Solvers require as one argument a user function to work on which have to be provided by the
user. These callbacks typically are of the form
\begin{funcdesc}{f}{x, params}
  \dots\\
  return result
\end{funcdesc}
Please note that this function must return the exact number of arguments
as given in the example. The wrappers around callbacks go a long way to try to provide
meaningfull error messages. If a solver fails, please check that the number of input and 
output arguments it takes are correct

\section{Error handling}
\label{sec:interface-error-handling}
As GSL is a C library error handling is implemented using an error handler and return values.
\pygsl{} generates python exceptions out of these values. See \module{pygsl.errors} 
(chapter~\ref{cha:error-module}) for a list of the exceptions.

\section{Exception handling}
\index{exception handling!initialisation} GSL provides a selectable error
handler, that is called for occuring errors (like domain errors, division by
zero, etc. ).  This is switched off. Instead each wrapper function will check
the error return value and in case of error an python exception is created. 

Here is a python level example:
\begin{verbatim}
import pygsl.histogram
import pygsl.errors
hist=pygsl.histogram.histogram2d(100,100)
try:
   hist[-1,-1]=0
except pygsl.errors.gsl_Error,err:
   print err
\end{verbatim}
Will result
\begin{verbatim}
input domain error: index i lies outside valid range of 0 .. nx - 1
\end{verbatim}


An exception are ufuncs in the testings.sf module (see section\ref{sec:ufuncs}).


\subsection{Change of internal error handling.}
Before a error handler was installed by init_pygsl into gsl which translated
the error code (and the message) to a python exception.
This required that the GIL was available, which numpy ufuncs dispose. Thus
now this gsl error handler is deactivated and instead the C error code
returned by the C function is translated to an error code by the wrapper
called from python.

UFuncs do not call this handler now at all.


\section{The documentation gap}

\pygsl{} does still lack an approbriate documentation. Most documentation is accessible over
the internal documentation strings. These are accessible as \code{__doc__} attributes (the help
function does not always show them).  It can be sometimes necessary to create an 
object to see its methods as well as the documentation of the methods
 (e.g.a random number generator in the rng module to see its methods). 
The \file{example} directory contains examples for (nearly each) module.

Please feel welcome to add to the documentation!


\paragraph*{Acknowledgment}
\label{sec:acknowledgment}
Parts of this this manual are based on the \GSL{} reference manual.
The authors want to thank all for contribution of code,
support material for generating distribution packages, bug reports
and example scripts.


\chapter[\protect\module{pygsl.errors} --- Error and warning classes]
{\protect\module{pygsl.errors} \\ Error and warning classes} 
\label{cha:error-module}
\declaremodule{standard}{pygsl.errors}
\moduleauthor{Pierre Schnizer}{schnizer@users.sourceforge.net}
\moduleauthor{Original Author: Achim G\"adke}{achimgaedke@users.sourceforge.net}

This chapter provides information about the \exception{gsl_Error} exception class that comes with this module.

\section{Exception Classes}


\begin{excclassdesc} {gsl_Error}{}
derived from \exception{Exception}, can be constructed with any object as parameter.
It is baseclass to all other \gsl{} Exceptions
\end{excclassdesc}
These classes are translations of the \file{<gsl/gsl_errno.h>} to python
exceptions.


\begin{excclassdesc}{gsl_ArithmeticError}{}
derived from \exception{gsl_Error} and \exception{exceptions.ArithmeticError},
base of all common arithmetic exceptions
\end{excclassdesc}

\begin{excclassdesc}{gsl_OverflowError}{}
derived from \exception{gsl_Error} and \exception{exceptions.OverflowError}
\end{excclassdesc}

\begin{excclassdesc}{gsl_ZeroDivisionError}{}
derived from \exception{gsl_Error} and \exception{exceptions.ZeroDivisionError}
\end{excclassdesc}

\begin{excclassdesc}{gsl_FloatingPointError}{}
derived from \exception{gsl_Error} and \exception{exceptions.FloatingPointError}
\end{excclassdesc}

\begin{excclassdesc}{gsl_ArithmeticError}{}
is derived from  \exception{gsl_Error} and from  \exception{ArithmeticError} .
This exception is the    base of all common arithmetic exceptions.
\end{excclassdesc}

\begin{excclassdesc}{gsl_AccuracyLossError}{}
is derived from  \exception{gsl_ArithmeticError} .
This exception is raised if the failed to reach the specified tolerance.
\end{excclassdesc}
\begin{excclassdesc}{gsl_BadFuncError}{}
is derived from  \exception{gsl_Error} .
This exception is raised if problem with a user-supplied function occur.
\end{excclassdesc}
\begin{excclassdesc}{gsl_BadLength}{}
is derived from  \exception{gsl_Error} .
This exception is raised if  matrix or  vector lengths are not conformant.
\end{excclassdesc}
\begin{excclassdesc}{gsl_BadToleranceError}{}
is derived from  \exception{gsl_Error} .
This exception is raised if user specified an tolerance which can not be reached.
\end{excclassdesc}
\begin{excclassdesc}{gsl_CacheLimitError}{}
is derived from  \exception{gsl_Error} .
This exception is raised if the    cache limit is exceeded.
\end{excclassdesc}
\begin{excclassdesc}{gsl_DivergeError}{}
is derived from  \exception{gsl_ArithmeticError} .
This exception is raised if an   integral or series is divergent.
\end{excclassdesc}
\begin{excclassdesc}{gsl_DomainError}{}
is derived from  \exception{gsl_Error} .
This exception is raised if    domain errors occure. e.g. sqrt(-1).
\end{excclassdesc}
\begin{excclassdesc}{gsl_EOFError}{}
is derived from  \exception{gsl_Error} and from  \exception{EOFError} .
This exception is raised if 
    end of file
     .
\end{excclassdesc}
\begin{excclassdesc}{gsl_FactorizationError}{}
is derived from  \exception{gsl_Error} .
This exception is raised if     factorization failed.
\end{excclassdesc}
\begin{excclassdesc}{gsl_FloatingPointError}{}
is derived from  \exception{gsl_Error} and from  \exception{FloatingPointError} .
\end{excclassdesc}
\begin{excclassdesc}{gsl_GenericError}{}
is derived from  \exception{gsl_Error} .
\end{excclassdesc}
\begin{excclassdesc}{gsl_InvalidArgumentError}{}
is derived from  \exception{gsl_Error} .
This exception is raised if an invalid argument is supplied by the user.
\end{excclassdesc}
\begin{excclassdesc}{gsl_JacobianEvaluationError}{}
is derived from  \exception{gsl_ArithmeticError} .
This exception is raised if jacobian evaluations are not improving the solution.
\end{excclassdesc}
\begin{excclassdesc}{gsl_MatrixNotSquare}{}
is derived from  \exception{gsl_Error} .
This exception is raised if the given matrix is not square.
\end{excclassdesc}
\begin{excclassdesc}{gsl_MaximumIterationError}{}
is derived from  \exception{gsl_ArithmeticError} .
This exception is raised if    the maximum number  of iterations is exceeded.
\end{excclassdesc}
\begin{excclassdesc}{gsl_NoHardwareSupportError}{}
is derived from  \exception{gsl_Error} .
This exception is raised if the requested feature is not supported by the hardware.
\end{excclassdesc}
\begin{excclassdesc}{gsl_NoProgressError}{}
is derived from  \exception{gsl_ArithmeticError} .
This exception is raised if the  iteration is not making progress towards solution.
\end{excclassdesc}
\begin{excclassdesc}{gsl_NotImplementedError}{}
is derived from  \exception{gsl_Error} and from  \exception{NotImplementedError} .
This exception is raised if  a requested feature is not (yet) implemented .
\end{excclassdesc}
\begin{excclassdesc}{gsl_OverflowError}{}
is derived from  \exception{gsl_Error} and from  \exception{OverflowError} .
\end{excclassdesc}
\begin{excclassdesc}{gsl_PointerError}{}
is derived from  \exception{gsl_Error} .
This exception is raised if an invalid pointer is found by the C wrapper code
or by the GSL library.
\end{excclassdesc}
\begin{excclassdesc}{gsl_RangeError}{}
is derived from  \exception{gsl_ArithmeticError} .
This exception is raised if     output would be out or range, e.g. exp(1e100)
     .
\end{excclassdesc}
\begin{excclassdesc}{gsl_RoundOffError}{}
is derived from  \exception{gsl_ArithmeticError} .
This exception is raised if  arithmetic failed because of roundoff error.
\end{excclassdesc}
\begin{excclassdesc}{gsl_RunAwayError}{}
is derived from  \exception{gsl_ArithmeticError} .
This exception is raised if   iterative process is out of control.
\end{excclassdesc}
\begin{excclassdesc}{gsl_SanityCheckError}{}
is derived from  \exception{gsl_Error} .
This exception is raised if a sanity check failed - shouldn't happen.
\end{excclassdesc}
\begin{excclassdesc}{gsl_SingularityError}{}
is derived from  \exception{gsl_ArithmeticError} .
This exception is raised if  an   apparent singularity is detected.
\end{excclassdesc}
\begin{excclassdesc}{gsl_TableLimitError}{}
is derived from  \exception{gsl_Error} .
This exception is raised if the table limit is exceeded.
\end{excclassdesc}
\begin{excclassdesc}{gsl_ToleranceError}{}
is derived from  \exception{gsl_ArithmeticError} .
This exception is raised if  the alghorithm failed to reach the specified tolerance.
\end{excclassdesc}
\begin{excclassdesc}{gsl_ToleranceFError}{}
is derived from  \exception{gsl_ArithmeticError} .
This exception is raised if  the alghorithm cannot reach the specified
tolerance in F (typically the variation of the evaluated function).
\end{excclassdesc}
\begin{excclassdesc}{gsl_ToleranceGradientError}{}
is derived from  \exception{gsl_ArithmeticError} .
This exception is raised if  cannot reach the specified tolerance for the gradient.
\end{excclassdesc}
\begin{excclassdesc}{gsl_ToleranceXError}{}
is derived from  \exception{gsl_ArithmeticError} .
This exception is raised if cannot reach the specified tolerance in X
(typically a search result).
\end{excclassdesc}
\begin{excclassdesc}{gsl_UnderflowError}{}
is derived from  \exception{gsl_Error} and from  \exception{OverflowError} .
\end{excclassdesc}
\begin{excclassdesc}{gsl_ZeroDivisionError}{}
is derived from  \exception{gsl_Error} and from  \exception{ZeroDivisionError} .
\end{excclassdesc}

All the above errors are just translations of the errno to python exceptions.

The following two are specific to pygsl:
\begin{excclassdesc}{pygsl.errors.pygsl_NotImplementedError}{}
is derived from  \exception{gsl_Error} and from  \exception{NotImplementedError} .
This exception is raised if a feature is requested but not
implemented. Currently only used if a module requests the debugging enviroment
of the init module, but the init module was not compiled with \code{\#define DEBUG=1}
\end{excclassdesc}
\begin{excclassdesc}{pygsl.errors.pygsl_StrideError}{}
is derived from  \exception{gsl_SanityCheckError} .
GSL uses as strides multiples of the basis type; for a vector or doubles, one
means from one double to the next. Numpy or numarray count the stride in
multiples of the size of a char. Therefore the stride has to be recalculated
before the approbriate \gsl{} function can be called. If that fails this
exception is raised.
\end{excclassdesc}

\section{Warning Classes}

\begin{excclassdesc} {gsl_Warning}{}
The dedicated warning class for \gsl{} has \exception{Warning} as base class.
\end{excclassdesc}

\begin{excclassdesc}{gsl_DomainWarning}{}
derived from \exception{gsl_Warning}, used by some \module{pygsl.histogram} functions
\end{excclassdesc}


\chapter[\protect\module{pygsl.const} --- Mathematical and scientific
constants]{\protect\module{pygsl.const} \\ Mathematical and scientific
constants} 
\label{cha:const-module}
\declaremodule{extension}{pygsl.const} 
\moduleauthor{Achim  G\"adke}{achimgaedke@users.sourceforge.net}

In this module some usefull constants are defined.  There are four groups of
constants:

\begin{itemize}
\item mathematical,
\item physical in mks unit system,
\item physical in cgs unit system and
\item physical number constants (e.g. fine structure)
\end{itemize}

The other modules are created during the initialisation of
\module{pygsl.const}.  For convenience the mathematical, physical mks
constants and number constants also are available in the namespace of
\module{pygsl.const}.  If the used GSL version is before gsl1.4, see
\begin{verbatim}
pygsl.compiled_gsl_version
\end{verbatim}
than the module names are cgs and mks. Form gsl1.5 these modules have been
renamed to cgsm and mksa. So to use cgs constants one has to write
\begin{verbatim}
import pygsl.const
import pygsl.const.cgs
print pygsl.const.cgs.speed_of_light/pygsl.const.speed_of_light
\end{verbatim}
for gsl $<$ 1.5 and
\begin{verbatim}
import pygsl.const
import pygsl.const.cgsm
print pygsl.const.cgsm.speed_of_light/pygsl.const.speed_of_light
\end{verbatim}.
Of course the result is \constant{100.0}.
Short examples are given at top of each section.

\begin{seealso}
  The actual values are taken form the \gsl{} headers.  The \GSL{} reference
  provides a more detailed description of these constants.
\end{seealso}

\section[\protect\module{pygsl.const.m} --- Mathematical constants]
{\protect\module{pygsl.const.m} \\ Mathematical constants} 
\label{cha:const-math-module}

\begin{verbatim}
from pygsl.const.m import *
print sqrt2*sqrt2
\end{verbatim}\\
Prints \constant{2.0}.\\
 Here comes the list:\nopagebreak
\begin{longtableiii}{l|l|l}{constant}{Name}{\gsl{} Name}{value}
\lineiii{e}{\protect\constant{M\_E}}{e}
\lineiii{log2e}{\constant{M\_LOG2E}}{$\log_2 e$}
\lineiii{log10e}{\constant{M\_LOG10E}}{$\log_{10} e$}
\lineiii{sqrt2}{\constant{M\_SQRT2}}{$\sqrt{2}$}
\lineiii{sqrt1\_2}{\constant{M\_SQRT1\_2}}{$\sqrt{1/2}$}
\lineiii{sqrt3}{\constant{M\_SQRT3}}{$\sqrt{3}$}
\lineiii{pi}{\constant{M\_PI}}{$\pi$}
\lineiii{pi\_2}{\constant{M\_PI\_2}}{$\pi/2$}
\lineiii{pi\_4}{\constant{M\_PI\_4}}{$\pi/4$}
\lineiii{sqrtpi}{\constant{M\_SQRTPI}}{$\sqrt{\pi}$}
\lineiii{2\_sqrtpi}{\constant{M\_2\_SQRTPI}}{$2/\sqrt{\pi}$}
\lineiii{1\_pi}{\constant{M\_1\_PI}}{$1/\pi$}
\lineiii{2\_pi}{\constant{M\_2\_PI}}{$2/\pi$}
\lineiii{ln10}{\constant{M\_LN10}}{$\ln 10$}
\lineiii{ln2}{\constant{M\_LN2}}{$\ln 2$}
\lineiii{lnpi}{\constant{M\_LNPI}}{$\ln{\pi}$}
\lineiii{euler}{\constant{M\_EULER}}{Euler constant}
\end{longtableiii}

\section[\protect\module{pygsl.const.mksa} --- Scientific constants in mksa units]
{\protect\module{pygsl.const.mksa} \\ Scientific constants in mksa units} 
\label{cha:const-mks-module}

\begin{verbatim}
from pygsl.const import cgsm
print "a teaspoon contains %g m^3"%mks.teaspoon
\end{verbatim}

These are the provided constants:\nopagebreak
\begin{longtableiii}{l|l|l}{constant}{Name}{gsl Name}{unit}
\lineiii{speed\_of\_light}{\constant{GSL\_CONST\_MKSA\_SPEED\_OF\_LIGHT}}{m / s}
\lineiii{gravitational\_constant}{\constant{GSL\_CONST\_MKSA\_GRAVITATIONAL\_CONSTANT}}{m\^{}3 / kg s\^{}2}
\lineiii{plancks\_constant\_h}{\constant{GSL\_CONST\_MKSA\_PLANCKS\_CONSTANT\_H}}{kg m\^{}2 / s}
\lineiii{plancks\_constant\_hbar}{\constant{GSL\_CONST\_MKSA\_PLANCKS\_CONSTANT\_HBAR}}{kg m\^{}2 / s}
\lineiii{vacuum\_permeability}{\constant{GSL\_CONST\_MKSA\_VACUUM\_PERMEABILITY}}{kg m / A\^{}2 s\^{}2}
\lineiii{astronomical\_unit}{\constant{GSL\_CONST\_MKSA\_ASTRONOMICAL\_UNIT}}{m}
\lineiii{light\_year}{\constant{GSL\_CONST\_MKSA\_LIGHT\_YEAR}}{m}
\lineiii{parsec}{\constant{GSL\_CONST\_MKSA\_PARSEC}}{m}
\lineiii{grav\_accel}{\constant{GSL\_CONST\_MKSA\_GRAV\_ACCEL}}{m / s\^{}2}
\lineiii{electron\_volt}{\constant{GSL\_CONST\_MKSA\_ELECTRON\_VOLT}}{kg m\^{}2 / s\^{}2}
\lineiii{mass\_electron}{\constant{GSL\_CONST\_MKSA\_MASS\_ELECTRON}}{kg}
\lineiii{mass\_muon}{\constant{GSL\_CONST\_MKSA\_MASS\_MUON}}{kg}
\lineiii{mass\_proton}{\constant{GSL\_CONST\_MKSA\_MASS\_PROTON}}{kg}
\lineiii{mass\_neutron}{\constant{GSL\_CONST\_MKSA\_MASS\_NEUTRON}}{kg}
\lineiii{rydberg}{\constant{GSL\_CONST\_MKSA\_RYDBERG}}{kg m\^{}2 / s\^{}2}
\lineiii{boltzmann}{\constant{GSL\_CONST\_MKSA\_BOLTZMANN}}{kg m\^{}2 / K s\^{}2}
\lineiii{bohr\_magneton}{\constant{GSL\_CONST\_MKSA\_BOHR\_MAGNETON}}{A m\^{}2}
\lineiii{nuclear\_magneton}{\constant{GSL\_CONST\_MKSA\_NUCLEAR\_MAGNETON}}{A m\^{}2}
\lineiii{electron\_magnetic\_moment}{\constant{GSL\_CONST\_MKSA\_ELECTRON\_MAGNETIC\_MOMENT}}{A m\^{}2}
\lineiii{proton\_magnetic\_moment}{\constant{GSL\_CONST\_MKSA\_PROTON\_MAGNETIC\_MOMENT}}{A m\^{}2}
\lineiii{molar\_gas}{\constant{GSL\_CONST\_MKSA\_MOLAR\_GAS}}{kg m\^{}2 / K mol s\^{}2}
\lineiii{standard\_gas\_volume}{\constant{GSL\_CONST\_MKSA\_STANDARD\_GAS\_VOLUME}}{m\^{}3 / mol}
\lineiii{minute}{\constant{GSL\_CONST\_MKSA\_MINUTE}}{s}
\lineiii{hour}{\constant{GSL\_CONST\_MKSA\_HOUR}}{s}
\lineiii{day}{\constant{GSL\_CONST\_MKSA\_DAY}}{s}
\lineiii{week}{\constant{GSL\_CONST\_MKSA\_WEEK}}{s}
\lineiii{inch}{\constant{GSL\_CONST\_MKSA\_INCH}}{m}
\lineiii{foot}{\constant{GSL\_CONST\_MKSA\_FOOT}}{m}
\lineiii{yard}{\constant{GSL\_CONST\_MKSA\_YARD}}{m}
\lineiii{mile}{\constant{GSL\_CONST\_MKSA\_MILE}}{m}
\lineiii{nautical\_mile}{\constant{GSL\_CONST\_MKSA\_NAUTICAL\_MILE}}{m}
\lineiii{fathom}{\constant{GSL\_CONST\_MKSA\_FATHOM}}{m}
\lineiii{mil}{\constant{GSL\_CONST\_MKSA\_MIL}}{m}
\lineiii{point}{\constant{GSL\_CONST\_MKSA\_POINT}}{m}
\lineiii{texpoint}{\constant{GSL\_CONST\_MKSA\_TEXPOINT}}{m}
\lineiii{micron}{\constant{GSL\_CONST\_MKSA\_MICRON}}{m}
\lineiii{angstrom}{\constant{GSL\_CONST\_MKSA\_ANGSTROM}}{m}
\lineiii{hectare}{\constant{GSL\_CONST\_MKSA\_HECTARE}}{m\^{}2}
\lineiii{acre}{\constant{GSL\_CONST\_MKSA\_ACRE}}{m\^{}2}
\lineiii{barn}{\constant{GSL\_CONST\_MKSA\_BARN}}{m\^{}2}
\lineiii{liter}{\constant{GSL\_CONST\_MKSA\_LITER}}{m\^{}3}
\lineiii{us\_gallon}{\constant{GSL\_CONST\_MKSA\_US\_GALLON}}{m\^{}3}
\lineiii{quart}{\constant{GSL\_CONST\_MKSA\_QUART}}{m\^{}3}
\lineiii{pint}{\constant{GSL\_CONST\_MKSA\_PINT}}{m\^{}3}
\lineiii{cup}{\constant{GSL\_CONST\_MKSA\_CUP}}{m\^{}3}
\lineiii{fluid\_ounce}{\constant{GSL\_CONST\_MKSA\_FLUID\_OUNCE}}{m\^{}3}
\lineiii{tablespoon}{\constant{GSL\_CONST\_MKSA\_TABLESPOON}}{m\^{}3}
\lineiii{teaspoon}{\constant{GSL\_CONST\_MKSA\_TEASPOON}}{m\^{}3}
\lineiii{canadian\_gallon}{\constant{GSL\_CONST\_MKSA\_CANADIAN\_GALLON}}{m\^{}3}
\lineiii{uk\_gallon}{\constant{GSL\_CONST\_MKSA\_UK\_GALLON}}{m\^{}3}
\lineiii{miles\_per\_hour}{\constant{GSL\_CONST\_MKSA\_MILES\_PER\_HOUR}}{m / s}
\lineiii{kilometers\_per\_hour}{\constant{GSL\_CONST\_MKSA\_KILOMETERS\_PER\_HOUR}}{m / s}
\lineiii{knot}{\constant{GSL\_CONST\_MKSA\_KNOT}}{m / s}
\lineiii{pound\_mass}{\constant{GSL\_CONST\_MKSA\_POUND\_MASS}}{kg}
\lineiii{ounce\_mass}{\constant{GSL\_CONST\_MKSA\_OUNCE\_MASS}}{kg}
\lineiii{ton}{\constant{GSL\_CONST\_MKSA\_TON}}{kg}
\lineiii{metric\_ton}{\constant{GSL\_CONST\_MKSA\_METRIC\_TON}}{kg}
\lineiii{uk\_ton}{\constant{GSL\_CONST\_MKSA\_UK\_TON}}{kg}
\lineiii{troy\_ounce}{\constant{GSL\_CONST\_MKSA\_TROY\_OUNCE}}{kg}
\lineiii{carat}{\constant{GSL\_CONST\_MKSA\_CARAT}}{kg}
\lineiii{unified\_atomic\_mass}{\constant{GSL\_CONST\_MKSA\_UNIFIED\_ATOMIC\_MASS}}{kg}
\lineiii{gram\_force}{\constant{GSL\_CONST\_MKSA\_GRAM\_FORCE}}{kg m / s\^{}2}
\lineiii{pound\_force}{\constant{GSL\_CONST\_MKSA\_POUND\_FORCE}}{kg m / s\^{}2}
\lineiii{kilopound\_force}{\constant{GSL\_CONST\_MKSA\_KILOPOUND\_FORCE}}{kg m / s\^{}2}
\lineiii{poundal}{\constant{GSL\_CONST\_MKSA\_POUNDAL}}{kg m / s\^{}2}
\lineiii{calorie}{\constant{GSL\_CONST\_MKSA\_CALORIE}}{kg m\^{}2 / s\^{}2}
\lineiii{btu}{\constant{GSL\_CONST\_MKSA\_BTU}}{kg m\^{}2 / s\^{}2}
\lineiii{therm}{\constant{GSL\_CONST\_MKSA\_THERM}}{kg m\^{}2 / s\^{}2}
\lineiii{horsepower}{\constant{GSL\_CONST\_MKSA\_HORSEPOWER}}{kg m\^{}2 / s\^{}3}
\lineiii{bar}{\constant{GSL\_CONST\_MKSA\_BAR}}{kg / m s\^{}2}
\lineiii{std\_atmosphere}{\constant{GSL\_CONST\_MKSA\_STD\_ATMOSPHERE}}{kg / m s\^{}2}
\lineiii{torr}{\constant{GSL\_CONST\_MKSA\_TORR}}{kg / m s\^{}2}
\lineiii{meter\_of\_mercury}{\constant{GSL\_CONST\_MKSA\_METER\_OF\_MERCURY}}{kg / m s\^{}2}
\lineiii{inch\_of\_mercury}{\constant{GSL\_CONST\_MKSA\_INCH\_OF\_MERCURY}}{kg / m s\^{}2}
\lineiii{inch\_of\_water}{\constant{GSL\_CONST\_MKSA\_INCH\_OF\_WATER}}{kg / m s\^{}2}
\lineiii{psi}{\constant{GSL\_CONST\_MKSA\_PSI}}{kg / m s\^{}2}
\lineiii{poise}{\constant{GSL\_CONST\_MKSA\_POISE}}{kg / m / s}
\lineiii{stokes}{\constant{GSL\_CONST\_MKSA\_STOKES}}{m\^{}2 / s}
\lineiii{faraday}{\constant{GSL\_CONST\_MKSA\_FARADAY}}{A s / mol}
\lineiii{electron\_charge}{\constant{GSL\_CONST\_MKSA\_ELECTRON\_CHARGE}}{A s}
\lineiii{gauss}{\constant{GSL\_CONST\_MKSA\_GAUSS}}{kg / A s\^{}2}
\lineiii{stilb}{\constant{GSL\_CONST\_MKSA\_STILB}}{cd / m\^{}2}
\lineiii{lumen}{\constant{GSL\_CONST\_MKSA\_LUMEN}}{cd sr}
\lineiii{lux}{\constant{GSL\_CONST\_MKSA\_LUX}}{cd sr / m\^{}2}
\lineiii{phot}{\constant{GSL\_CONST\_MKSA\_PHOT}}{cd sr / m\^{}2}
\lineiii{footcandle}{\constant{GSL\_CONST\_MKSA\_FOOTCANDLE}}{cd sr / m\^{}2}
\lineiii{lambert}{\constant{GSL\_CONST\_MKSA\_LAMBERT}}{cd sr / m\^{}2}
\lineiii{footlambert}{\constant{GSL\_CONST\_MKSA\_FOOTLAMBERT}}{cd sr / m\^{}2}
\lineiii{curie}{\constant{GSL\_CONST\_MKSA\_CURIE}}{1 / s}
\lineiii{roentgen}{\constant{GSL\_CONST\_MKSA\_ROENTGEN}}{A s / kg}
\lineiii{rad}{\constant{GSL\_CONST\_MKSA\_RAD}}{m\^{}2 / s\^{}2}
\lineiii{solar\_mass}{\constant{GSL\_CONST\_MKSA\_SOLAR\_MASS}}{kg}
\lineiii{bohr\_radius}{\constant{GSL\_CONST\_MKSA\_BOHR\_RADIUS}}{m}
\lineiii{vacuum\_permittivity}{\constant{GSL\_CONST\_MKSA\_VACUUM\_PERMITTIVITY}}{A\^{}2 s\^{}4 / kg m\^{}3}
\end{longtableiii}

\section[\protect\module{pygsl.const.cgsm} --- Scientific constants in cgsm units]
{\protect\module{pygsl.const.cgsm} \\ Scientific constants in cgsm units} 
\label{cha:const-cgs-module}

\begin{verbatim}
from pygsl.const import cgsm
print "a teaspoon contains %g ml"%cgs.teaspoon
\end{verbatim}

You can access the following constants:\nopagebreak
\begin{longtableiii}{l|l|l}{constant}{Name}{gsl Name}{unit or value}
\lineiii{speed\_of\_light}{\constant{GSL\_CONST\_CGSM\_SPEED\_OF\_LIGHT}}{cm / s}
\lineiii{gravitational\_constant}{\constant{GSL\_CONST\_CGSM\_GRAVITATIONAL\_CONSTANT}}{cm\^{}3 / g s\^{} 2}
\lineiii{plancks\_constant\_h}{\constant{GSL\_CONST\_CGSM\_PLANCKS\_CONSTANT\_H}}{g cm\^{}2 / s}
\lineiii{plancks\_constant\_hbar}{\constant{GSL\_CONST\_CGSM\_PLANCKS\_CONSTANT\_HBAR}}{g cm\^{}2 / s}
\lineiii{vacuum\_permeability}{\constant{GSL\_CONST\_CGSM\_VACUUM\_PERMEABILITY}}{cm g / A\^{}2 s\^{}2}
\lineiii{astronomical\_unit}{\constant{GSL\_CONST\_CGSM\_ASTRONOMICAL\_UNIT}}{cm}
\lineiii{light\_year}{\constant{GSL\_CONST\_CGSM\_LIGHT\_YEAR}}{cm}
\lineiii{parsec}{\constant{GSL\_CONST\_CGSM\_PARSEC}}{cm}
\lineiii{grav\_accel}{\constant{GSL\_CONST\_CGSM\_GRAV\_ACCEL}}{cm / s\^{}2}
\lineiii{electron\_volt}{\constant{GSL\_CONST\_CGSM\_ELECTRON\_VOLT}}{g cm\^{}2 / s\^{}2}
\lineiii{mass\_electron}{\constant{GSL\_CONST\_CGSM\_MASS\_ELECTRON}}{g}
\lineiii{mass\_muon}{\constant{GSL\_CONST\_CGSM\_MASS\_MUON}}{g}
\lineiii{mass\_proton}{\constant{GSL\_CONST\_CGSM\_MASS\_PROTON}}{g}
\lineiii{mass\_neutron}{\constant{GSL\_CONST\_CGSM\_MASS\_NEUTRON}}{g}
\lineiii{rydberg}{\constant{GSL\_CONST\_CGSM\_RYDBERG}}{g cm\^{}2 / s\^{}2}
\lineiii{boltzmann}{\constant{GSL\_CONST\_CGSM\_BOLTZMANN}}{g cm\^{}2 / K s\^{}2}
\lineiii{bohr\_magneton}{\constant{GSL\_CONST\_CGSM\_BOHR\_MAGNETON}}{A cm\^{}2}
\lineiii{nuclear\_magneton}{\constant{GSL\_CONST\_CGSM\_NUCLEAR\_MAGNETON}}{A cm\^{}2}
\lineiii{electron\_magnetic\_moment}{\constant{GSL\_CONST\_CGSM\_ELECTRON\_MAGNETIC\_MOMENT}}{A cm\^{}2}
\lineiii{proton\_magnetic\_moment}{\constant{GSL\_CONST\_CGSM\_PROTON\_MAGNETIC\_MOMENT}}{A cm\^{}2}
\lineiii{molar\_gas}{\constant{GSL\_CONST\_CGSM\_MOLAR\_GAS}}{g cm\^{}2 / K mol s\^{}2}
\lineiii{standard\_gas\_volume}{\constant{GSL\_CONST\_CGSM\_STANDARD\_GAS\_VOLUME}}{cm\^{}3 / mol}
\lineiii{minute}{\constant{GSL\_CONST\_CGSM\_MINUTE}}{s}
\lineiii{hour}{\constant{GSL\_CONST\_CGSM\_HOUR}}{s}
\lineiii{day}{\constant{GSL\_CONST\_CGSM\_DAY}}{s}
\lineiii{week}{\constant{GSL\_CONST\_CGSM\_WEEK}}{s}
\lineiii{inch}{\constant{GSL\_CONST\_CGSM\_INCH}}{cm}
\lineiii{foot}{\constant{GSL\_CONST\_CGSM\_FOOT}}{cm}
\lineiii{yard}{\constant{GSL\_CONST\_CGSM\_YARD}}{cm}
\lineiii{mile}{\constant{GSL\_CONST\_CGSM\_MILE}}{cm}
\lineiii{nautical\_mile}{\constant{GSL\_CONST\_CGSM\_NAUTICAL\_MILE}}{cm}
\lineiii{fathom}{\constant{GSL\_CONST\_CGSM\_FATHOM}}{cm}
\lineiii{mil}{\constant{GSL\_CONST\_CGSM\_MIL}}{cm}
\lineiii{point}{\constant{GSL\_CONST\_CGSM\_POINT}}{cm}
\lineiii{texpoint}{\constant{GSL\_CONST\_CGSM\_TEXPOINT}}{cm}
\lineiii{micron}{\constant{GSL\_CONST\_CGSM\_MICRON}}{cm}
\lineiii{angstrom}{\constant{GSL\_CONST\_CGSM\_ANGSTROM}}{cm}
\lineiii{hectare}{\constant{GSL\_CONST\_CGSM\_HECTARE}}{cm\^{}2}
\lineiii{acre}{\constant{GSL\_CONST\_CGSM\_ACRE}}{cm\^{}2}
\lineiii{barn}{\constant{GSL\_CONST\_CGSM\_BARN}}{cm\^{}2}
\lineiii{liter}{\constant{GSL\_CONST\_CGSM\_LITER}}{cm\^{}3}
\lineiii{us\_gallon}{\constant{GSL\_CONST\_CGSM\_US\_GALLON}}{cm\^{}3}
\lineiii{quart}{\constant{GSL\_CONST\_CGSM\_QUART}}{cm\^{}3}
\lineiii{pint}{\constant{GSL\_CONST\_CGSM\_PINT}}{cm\^{}3}
\lineiii{cup}{\constant{GSL\_CONST\_CGSM\_CUP}}{cm\^{}3}
\lineiii{fluid\_ounce}{\constant{GSL\_CONST\_CGSM\_FLUID\_OUNCE}}{cm\^{}3}
\lineiii{tablespoon}{\constant{GSL\_CONST\_CGSM\_TABLESPOON}}{cm\^{}3}
\lineiii{teaspoon}{\constant{GSL\_CONST\_CGSM\_TEASPOON}}{cm\^{}3}
\lineiii{canadian\_gallon}{\constant{GSL\_CONST\_CGSM\_CANADIAN\_GALLON}}{cm\^{}3}
\lineiii{uk\_gallon}{\constant{GSL\_CONST\_CGSM\_UK\_GALLON}}{cm\^{}3}
\lineiii{miles\_per\_hour}{\constant{GSL\_CONST\_CGSM\_MILES\_PER\_HOUR}}{cm / s}
\lineiii{kilometers\_per\_hour}{\constant{GSL\_CONST\_CGSM\_KILOMETERS\_PER\_HOUR}}{cm / s}
\lineiii{knot}{\constant{GSL\_CONST\_CGSM\_KNOT}}{cm / s}
\lineiii{pound\_mass}{\constant{GSL\_CONST\_CGSM\_POUND\_MASS}}{g}
\lineiii{ounce\_mass}{\constant{GSL\_CONST\_CGSM\_OUNCE\_MASS}}{g}
\lineiii{ton}{\constant{GSL\_CONST\_CGSM\_TON}}{g}
\lineiii{metric\_ton}{\constant{GSL\_CONST\_CGSM\_METRIC\_TON}}{g}
\lineiii{uk\_ton}{\constant{GSL\_CONST\_CGSM\_UK\_TON}}{g}
\lineiii{troy\_ounce}{\constant{GSL\_CONST\_CGSM\_TROY\_OUNCE}}{g}
\lineiii{carat}{\constant{GSL\_CONST\_CGSM\_CARAT}}{g}
\lineiii{unified\_atomic\_mass}{\constant{GSL\_CONST\_CGSM\_UNIFIED\_ATOMIC\_MASS}}{g}
\lineiii{gram\_force}{\constant{GSL\_CONST\_CGSM\_GRAM\_FORCE}}{cm g / s\^{}2}
\lineiii{pound\_force}{\constant{GSL\_CONST\_CGSM\_POUND\_FORCE}}{cm g / s\^{}2}
\lineiii{kilopound\_force}{\constant{GSL\_CONST\_CGSM\_KILOPOUND\_FORCE}}{cm g / s\^{}2}
\lineiii{poundal}{\constant{GSL\_CONST\_CGSM\_POUNDAL}}{cm g / s\^{}2}
\lineiii{calorie}{\constant{GSL\_CONST\_CGSM\_CALORIE}}{g cm\^{}2 / s\^{}2}
\lineiii{btu}{\constant{GSL\_CONST\_CGSM\_BTU}}{g cm\^{}2 / s\^{}2}
\lineiii{therm}{\constant{GSL\_CONST\_CGSM\_THERM}}{g cm\^{}2 / s\^{}2}
\lineiii{horsepower}{\constant{GSL\_CONST\_CGSM\_HORSEPOWER}}{g cm\^{}2 / s\^{}3}
\lineiii{bar}{\constant{GSL\_CONST\_CGSM\_BAR}}{g / cm s\^{}2}
\lineiii{std\_atmosphere}{\constant{GSL\_CONST\_CGSM\_STD\_ATMOSPHERE}}{g / cm s\^{}2}
\lineiii{torr}{\constant{GSL\_CONST\_CGSM\_TORR}}{g / cm s\^{}2}
\lineiii{meter\_of\_mercury}{\constant{GSL\_CONST\_CGSM\_METER\_OF\_MERCURY}}{g / cm s\^{}2}
\lineiii{inch\_of\_mercury}{\constant{GSL\_CONST\_CGSM\_INCH\_OF\_MERCURY}}{g / cm s\^{}2}
\lineiii{inch\_of\_water}{\constant{GSL\_CONST\_CGSM\_INCH\_OF\_WATER}}{g / cm s\^{}2}
\lineiii{psi}{\constant{GSL\_CONST\_CGSM\_PSI}}{g / cm s\^{}2}
\lineiii{poise}{\constant{GSL\_CONST\_CGSM\_POISE}}{g / cm s}
\lineiii{stokes}{\constant{GSL\_CONST\_CGSM\_STOKES}}{cm\^{}2 / s}
\lineiii{faraday}{\constant{GSL\_CONST\_CGSM\_FARADAY}}{A s / mol}
\lineiii{electron\_charge}{\constant{GSL\_CONST\_CGSM\_ELECTRON\_CHARGE}}{A s}
\lineiii{gauss}{\constant{GSL\_CONST\_CGSM\_GAUSS}}{g / A s\^{}2}
\lineiii{stilb}{\constant{GSL\_CONST\_CGSM\_STILB}}{cd / cm\^{}2}
\lineiii{lumen}{\constant{GSL\_CONST\_CGSM\_LUMEN}}{cd sr}
\lineiii{lux}{\constant{GSL\_CONST\_CGSM\_LUX}}{cd sr / cm\^{}2}
\lineiii{phot}{\constant{GSL\_CONST\_CGSM\_PHOT}}{cd sr / cm\^{}2}
\lineiii{footcandle}{\constant{GSL\_CONST\_CGSM\_FOOTCANDLE}}{cd sr / cm\^{}2}
\lineiii{lambert}{\constant{GSL\_CONST\_CGSM\_LAMBERT}}{cd sr / cm\^{}2}
\lineiii{footlambert}{\constant{GSL\_CONST\_CGSM\_FOOTLAMBERT}}{cd sr / cm\^{}2}
\lineiii{curie}{\constant{GSL\_CONST\_CGSM\_CURIE}}{1 / s}
\lineiii{roentgen}{\constant{GSL\_CONST\_CGSM\_ROENTGEN}}{A s / g}
\lineiii{rad}{\constant{GSL\_CONST\_CGSM\_RAD}}{cm\^{}2 / s\^{}2}
\lineiii{solar\_mass}{\constant{GSL\_CONST\_CGSM\_SOLAR\_MASS}}{g}
\lineiii{bohr\_radius}{\constant{GSL\_CONST\_CGSM\_BOHR\_RADIUS}}{cm}
\lineiii{vacuum\_permittivity}{\constant{GSL\_CONST\_CGSM\_VACUUM\_PERMITTIVITY}}{A\^{}2 s\^{}4 / g cm\^{}3}
\end{longtableiii}

\section[\protect\module{pygsl.const.num} --- Scientific number constants]
{\protect\module{pygsl.const.num} \\ Scientific number constants} 
\label{cha:const-num-module}

\begin{verbatim}
from pygsl.const import *
# an alternative to
# from pygsl.const.num import *
print "fine structure is 1/137 with an error of %g%%"%(abs(1.0/137.0/fine_structure-1.0)*100.0)
\end{verbatim}

Only two constants are available:\nopagebreak
\begin{longtableiii}{l|l|l}{constant}{Name}{gsl Name}{unit}
\lineiii{fine\_structure}{\constant{GSL\_CONST\_NUM\_FINE\_STRUCTURE}}{1}
\lineiii{avogadro}{\constant{GSL\_CONST\_NUM\_AVOGADRO}}{1 / mol}
\end{longtableiii}


\chapter[\protect\module{pygsl.chebyshev}]
{\protect\module{pygsl.chebyshev}}
\label{cha:statistics-module}

\declaremodule{standard}{pygsl.chebyshev}
\moduleauthor{Pierre Schnizer}{schnizer@users.sourceforge.net}

\begin{classdesc}{cheb_series}{}
  This base class can be instantiated by its name
\end{classdesc}
\begin{verbatim}
import pygsl.chebyshev
s=pygsl.chebyshev.cheb_series()
\end{verbatim}

\begin{methoddesc}{__init__}{n}\index{__init__}
            n ... number of coefficients        
\end{methoddesc}
\begin{methoddesc}{init}{f, a, b}\index{init}
        This function computes the Chebyshev approximation for the
        function F over the range (a,b) to the previously specified order.
        The computation of the Chebyshev approximation is an O($n^2$)
        process, and requires n function evaluations.

            f ... a gsl_function
            a ... lower limit
            b ... upper limit
        
\end{methoddesc}
\begin{methoddesc}{eval}{x}\index{eval}
        This function evaluates the Chebyshev series at a given point X.
\end{methoddesc}
\begin{methoddesc}{eval_err}{x}\index{eval_err}
         This function computes the Chebyshev series  at a given point X,
         estimating both the series RESULT and its absolute error ABSERR.
         The error estimate is made from the first neglected term in the
         series.
\end{methoddesc}
\begin{methoddesc}{eval_n}{n, x}\index{eval_n}
         This function evaluates the Chebyshev series at a given point
         x, to (at most) the given order n
\end{methoddesc}
\begin{methoddesc}{eval_n_err}{n, x}\index{eval_n_err}
        This function evaluates a Chebyshev series at a given point X,
        estimating both the series RESULT and its absolute error ABSERR,
        to (at most) the given order ORDER.  The error estimate is made
        from the first neglected term in the series.
\end{methoddesc}

\begin{methoddesc}{calc_deriv}{}\index{calc_deriv}
        This method computes the derivative of the series CS. It returns
        a new instance of the cheb_series class.
\end{methoddesc}
\begin{methoddesc}{calc_integ}{}\index{calc_integ}
        This method computes the integral of the series CS. It returns
        a new instance of the cheb_series class.
\end{methoddesc}
\begin{methoddesc}{get_a}{}\index{get_a}
        Get the lower boundary of the current representation       
\end{methoddesc}
\begin{methoddesc}{get_b}{}\index{get_b}
        Get the upper boundary of the current representation        
\end{methoddesc}
\begin{methoddesc}{get_coefficients}{}\index{get_coefficients}
        Get the chebyshev coefficients.         
\end{methoddesc}
\begin{methoddesc}{get_f}{}\index{get_f}
        Get the value f (what is it ?) The documentation does not tell anything
        about it.        
\end{methoddesc}
\begin{methoddesc}{get_order_sp}{}\index{get_order_sp}
        Get the value f (what is it ?) The documentation does not tell anything
        about it.        
\end{methoddesc}
\begin{methoddesc}{set_a}{}\index{set_a}
        Set the lower boundary of the current representation        
\end{methoddesc}
\begin{methoddesc}{set_b}{}\index{set_b}
        Set the upper boundary of the current         
\end{methoddesc}
\begin{methoddesc}{set_coefficients}{}\index{set_coefficients}
        Sets the chebyshev coefficients. 
\end{methoddesc}
\begin{methoddesc}{set_f}{f}\index{set_f}
        Set the value f (what is it ?)        
\end{methoddesc}
\begin{methoddesc}{set_order_sp}{...}\index{set_order_sp}
        Set the value f (what is it ?)        
\end{methoddesc}


\begin{funcdesc}{gsl_function}{f, params}\index{gsl_function}

    This class defines the callbacks known as gsl_function to
    gsl.

    e.g to supply the function f:
    
    def f(x, params):
        a = params[0]
        b = parmas[1]
        c = params[3]
        return a * x ** 2 + b * x + c

    to some solver, use

    function = gsl_function(f, params)
    
\end{funcdesc}

%%% Local Variables: 
%%% mode: latex
%%% TeX-master: "ref"
%%% End: 

\chapter[\protect\module{pygsl.deriv} --- NumericalDifferentiation]%
{\protect\module{pygsl.deriv} \\ Numerical Differentiation}
\label{cha:diff-module}

\declaremodule{extension}{pygsl.deriv}%
 \moduleauthor{Pierre  Schnizer}{schnizer@users.sourceforge.net}%
 \modulesynopsis{Numerical  Differentiation}%

\begin{quote}
  This chapter describes the available functions for numerical differentiation.
\end{quote}

The functions described in this chapter compute numerical derivatives by finite
differencing.  An adaptive algorithm is used to find the best choice of finite
difference and to estimate the error in the derivative. This module supersedes
the diff module which has been deprecated with the release of GSL-1. XXX


\begin{funcdesc}{central}{func, x, h}
  This function computes the numerical derivative of the function \var{func} at
  the point \var{x} using an adaptive central difference algorithm with a step
  size of h.  A tuple \code{(result, error)} is returned with the derivative
  and its estimated absolute error.
\end{funcdesc}

\begin{funcdesc}{backward}{func, x, h}
  This function computes the numerical derivative of the function \var{func} at
  the point \var{x} using an adaptive backward difference algorithm with a step
  size of h.  The function \var{func} is evaluated only at points smaller than
  \var{x} and at \var{x} itself.  A tuple \code{(result, error)} is returned
  with the derivative and its estimated absolute error.
\end{funcdesc}

\begin{funcdesc}{forward}{func, x, h}
  This function computes the numerical derivative of the function \var{func} at
  the point \var{x} using an adaptive forward difference algorithm with a step
  size of h.  The function \var{func} is evaluated only at points greater than
  \var{x} and at \var{x} itself.  A tuple \code{(result, error)} is returned
  with the derivative and its estimated absolute error.
\end{funcdesc}


\begin{seealso}
  The algorithms used by these functions are described in the following book:
  \seetext{S.D.\ Conte and Carl de Boor, \emph{Elementary Numerical Analysis:
      An Algorithmic Approach}, McGraw-Hill, 1972.}
\end{seealso}



%% Local Variables:
%% mode: LaTeX
%% mode: auto-fill
%% fill-column: 79
%% indent-tabs-mode: nil
%% ispell-dictionary: "british"
%% reftex-fref-is-default: nil
%% TeX-auto-save: t
%% TeX-command-default: "pdfeLaTeX"
%% TeX-master: "pygsl"
%% TeX-parse-self: t
%% End:


\chapter[\protect\module{pygsl.histogram} --- Histogram Types]
{\protect\module{pygsl.histogram} \\ Histogram Types}
\label{cha:histogram-module}
\declaremodule{extension}{pygsl.histogram}
\moduleauthor{Achim G\"adke}{achimgaedke@users.sourceforge.net}

This chapter is about the \class{histogram} and \class{histogram2d} type that
are contained in \module{pygsl.histogram}.

\section{\protect\class{histogram} --- 1-dimensional histograms}

\begin{classdesc}{histogram}{\texttt{long} size \code{|} \class{histogram} h}
This type implements all methods on \ctype{struct gsl_histogram}.
\end{classdesc}

\begin{methoddesc}{alloc}{\texttt{long} length}
allocate necessary space, \hfill returns \texttt{None}
\end{methoddesc}
\begin{methoddesc}{set_ranges_uniform}{\texttt{float} upper, \texttt{float} lower}
set the ranges to uniform distance, \hfill returns \texttt{None}
\end{methoddesc}
\begin{methoddesc}{reset}{}
sets all bin values to 0, \hfill returns \texttt{None}
\end{methoddesc}
\begin{methoddesc}{increment}{\texttt{float} x}
increments corresponding bin, \hfill returns \texttt{None}
\end{methoddesc}
\begin{methoddesc}{accumulate}{\texttt{float} x, \texttt{float} weight}
adds the weight to corresponding bin, \hfill returns \texttt{None}
\end{methoddesc}
\begin{methoddesc}{max}{}
returns upper range, \hfill as \texttt{float}
\end{methoddesc}
\begin{methoddesc}{min}{}
returns lower range, \hfill as \texttt{float}
\end{methoddesc}
\begin{methoddesc}{bins}{}
returns number of bins, \hfill as \texttt{long}
\end{methoddesc}
\begin{methoddesc}{get}{\texttt{long} n}
gets value of indexed bin, \hfill returns \texttt{float}
\end{methoddesc}
\begin{methoddesc}{get_range}{\texttt{long} n}
gets upper and lower range of indexed bin, \hfill returns \texttt{(float,float)}
\end{methoddesc}
\begin{methoddesc}{find}{\texttt{float} x}
finds index of corresponding bin, \hfill returns \texttt{long}
\end{methoddesc}
\begin{methoddesc}{max_val}{}
returns maximal bin value, \hfill as \texttt{float}
\end{methoddesc}
\begin{methoddesc}{max_bin}{}
returns bin index with maximal value, \hfill as \texttt{long}
\end{methoddesc}
\begin{methoddesc}{min_val}{}
returns minimal bin value, \hfill as \texttt{float}
\end{methoddesc}
\begin{methoddesc}{min_bin}{}
returns bin index with minimal value, \hfill as \texttt{long}
\end{methoddesc}
\begin{methoddesc}{mean}{}
returns mean of histogram, \hfill as \texttt{float}
\end{methoddesc}
\begin{methoddesc}{sigma}{}
returns std deviation of histogram, \hfill as \texttt{float}
\end{methoddesc}
\begin{methoddesc}{sum}{}
returns sum of bin values, \hfill as \texttt{float}
\end{methoddesc}
\begin{methoddesc}{set_ranges}{\texttt{sequence} ranges}
sets range according given sequence, \hfill returns \texttt{None}
\end{methoddesc}
\begin{methoddesc}{shift}{\texttt{float} offset}
shifts the contents of the bins by the given offset, \hfill returns
\texttt{None}
\end{methoddesc}
\begin{methoddesc}{scale}{\texttt{float} scale}
multiplies the contents of the bins by the given scale, \hfill returns \texttt{None}
\end{methoddesc}
\begin{methoddesc}{equal_bins_p}{}
true if the all of the individual bin ranges are identical, \hfill returns \texttt{int}
\end{methoddesc}
\begin{methoddesc}{add}{\texttt{histogram} h}
adds the contents of the bins, \hfill returns \texttt{None}
\end{methoddesc}
\begin{methoddesc}{sub}{\texttt{histogram} h}
substracts the contents of the bins, \hfill returns \texttt{None}
\end{methoddesc}
\begin{methoddesc}{mul}{\texttt{histogram} h}
multiplicates the contents of the bins, \hfill returns \texttt{None}
\end{methoddesc}
\begin{methoddesc}{div}{\texttt{histogram} h}
divides the contents of the bins, \hfill returns \texttt{None}
\end{methoddesc}
\begin{methoddesc}{clone}{\texttt{histogram} h}
returns a new copy of this histogram, \hfill returns new \texttt{histogram}
\end{methoddesc}
\begin{methoddesc}{copy}{\texttt{histogram} h}
copies the given histogram to myself, \hfill returns \texttt{None}
\end{methoddesc}
\begin{methoddesc}{read}{\texttt{file} input}
reads histogram binary data from file, \hfill returns \texttt{None}
\end{methoddesc}
\begin{methoddesc}{write}{\texttt{file} output}
writes histogram binary data to file, \hfill returns \texttt{None}
\end{methoddesc}
\begin{methoddesc}{scanf}{\texttt{file} input}
reads histogram data from file using scanf, \hfill returns \texttt{None}
\end{methoddesc}
\begin{methoddesc}{printf}{\texttt{file} output}
writes histogram data to file using printf, \hfill returns \texttt{None}
\end{methoddesc}


Some mapping operations are supported, too:\nopagebreak
\begin{tableii}{l|l}{texttt}{Mapping syntax}{Effect}
\lineii{histogram[index]}{returns the value of the indexed bin}
\lineii{histogram[index]=value}{sets the value of the indexed bin}
\lineii{len(histogram)}{returns the length of the histogram}
\end{tableii}

\begin{seealso}
For the special meaning and details please consult the GNU Scientific Library
reference.
\end{seealso}


\section{\protect\class{histogram2d} --- 2-dimensional histograms}

\begin{classdesc}{histogram2d}{\texttt{long} size x, \texttt{long} size y
                               \code{|} \class{histogram2d} h}
This class holds a 2d array and 2 sets of ranges for x and y coordinates for a
two paramter statistical event. It can be constructed by size parameters or
as a copy from another histogram. Most of the methods are the same as of
\class{histogram}.
\end{classdesc}

\begin{methoddesc}{set_ranges_uniform}{\texttt{float} xmin, \texttt{float} xmax,
                                       \texttt{float} ymin, \texttt{float} ymax}
set the ranges to uniform distance, \hfill returns \texttt{None}
\end{methoddesc}
\begin{methoddesc}{alloc}{\texttt{long} nx, \texttt{long} ny}
allocate necessary space, \hfill returns \texttt{None}
\end{methoddesc}
\begin{methoddesc}{reset}{}
sets all bin values to 0, \hfill returns \texttt{None}
\end{methoddesc}
\begin{methoddesc}{increment}{\texttt{float} x, \texttt{float} y}
increments corresponding bin, \hfill returns \texttt{None}
\end{methoddesc}
\begin{methoddesc}{accumulate}{\texttt{float} x, \texttt{float} y,
                               \texttt{float} weight}
adds the weight to corresponding bin, \hfill returns \texttt{None}
\end{methoddesc}
\begin{methoddesc}{xmax}{}
returns upper x range \hfill as \texttt{float}
\end{methoddesc}
\begin{methoddesc}{xmin}{}
returns lower x range \hfill as \texttt{float}
\end{methoddesc}
\begin{methoddesc}{ymax}{}
returns upper y range \hfill as \texttt{float}
\end{methoddesc}
\begin{methoddesc}{ymin}{}
returns lower y range \hfill as \texttt{float}
\end{methoddesc}
\begin{methoddesc}{nx}{}
returns number of x bins \hfill as \texttt{long}
\end{methoddesc}
\begin{methoddesc}{ny}{}
returns number of y bins \hfill as \texttt{long}
\end{methoddesc}
\begin{methoddesc}{get}{\texttt{long} i, \texttt{long} j}
gets value of indexed bin,\hfill returns \texttt{float}
\end{methoddesc}
\begin{methoddesc}{get_xrange}{\texttt{long} i}
gets upper and lower x range of indexed bin,
\hfill returns \texttt{(float \textrm{lower}, float \textrm{upper})}
\end{methoddesc}
\begin{methoddesc}{get_yrange}{\texttt{long} j}
gets upper and lower y range of indexed bin,
\hfill returns \texttt{(float \textrm{lower}, float \textrm{upper})}
\end{methoddesc}
\begin{methoddesc}{find}{\texttt{float} x, \texttt{float} y}
finds index pair of corresponding value pair,
\hfill returns (\texttt{long},\texttt{long})
\end{methoddesc}
\begin{methoddesc}{max_val}{}
returns maximal bin value \hfill as \texttt{float}
\end{methoddesc}
\begin{methoddesc}{max_bin}{}
returns bin index with maximal value \hfill as \texttt{long}
\end{methoddesc}
\begin{methoddesc}{min_val}{}
returns minimal bin value \hfill as \texttt{float}
\end{methoddesc}
\begin{methoddesc}{min_bin}{}
returns bin index with minimal value \hfill as \texttt{long}
\end{methoddesc}
\begin{methoddesc}{sum}{}
returns sum of bin values \hfill as \texttt{float}
\end{methoddesc}
\begin{methoddesc}{xmean}{}
returns x mean of histogram \hfill as \texttt{float}
\end{methoddesc}
\begin{methoddesc}{xsigma}{}
returns x std deviation of histogram \hfill as \texttt{float}
\end{methoddesc}
\begin{methoddesc}{ymean}{}
returns y mean of histogram \hfill as\texttt{float}
\end{methoddesc}
\begin{methoddesc}{ysigma}{}
returns y std deviation of histogram \hfill as \texttt{float}
\end{methoddesc}
\begin{methoddesc}{cov}{}
returns covariance of histogram \hfill as \texttt{float}
\end{methoddesc}
\begin{methoddesc}{set_ranges}{sequence xranges, sequence yranges}
set the ranges according to given sequences, \hfill returns \texttt{None}
\end{methoddesc}
\begin{methoddesc}{shift}{\texttt{float} offset}
shifts the contents of the bins by the given offset, \hfill returns \texttt{None}
\end{methoddesc}
\begin{methoddesc}{scale}{\texttt{float} scale}
multiplies the contents of the bins by the given scale, \hfill returns \texttt{None}
\end{methoddesc}
\begin{methoddesc}{equal_bins_p}{}
true if the all of the individual bin ranges are identical, \hfill returns \texttt{int}
\end{methoddesc}
\begin{methoddesc}{add}{\class{histogram2d} h}
adds the contents of the bins, \hfill returns \texttt{None}
\end{methoddesc}
\begin{methoddesc}{sub}{\class{histogram2d} h}
substracts the contents of the bins, \hfill returns \texttt{None}
\end{methoddesc}
\begin{methoddesc}{mul}{\class{histogram2d} h}
multiplicates the contents of the bins, \hfill returns \texttt{None}
\end{methoddesc}
\begin{methoddesc}{div}{\class{histogram2d} h}
divides the contents of the bins, \hfill returns \texttt{None}
\end{methoddesc}
\begin{methoddesc}{clone}{}
returns a copy instance of \hfill\class{histogram2d}
\end{methoddesc}
\begin{methoddesc}{copy}{\class{histogram2d} h}
copies the given histogram to myself, \hfill returns \texttt{None}
\end{methoddesc}
\begin{methoddesc}{read}{file input}
reads histogram binary data from file, \hfill returns \texttt{None}
\end{methoddesc}
\begin{methoddesc}{writew}{file output}
writes histogram binary data to file, \hfill returns \texttt{None}
\end{methoddesc}
\begin{methoddesc}{scanf}{file input}
reads histogram data from file using scanf, \hfill returns \texttt{None}
\end{methoddesc}
\begin{methoddesc}{printf}{file input}
writes histogram data to file using printf, \hfill returns \texttt{None}
\end{methoddesc}

Some mapping operations are supported, too:\nopagebreak
\begin{tableii}{l|l}{code}{Mapping syntax}{Effect}
\lineii{histogram[x\_index,y\_index]}{returns the value of the indexed bin}
\lineii{histogram[x\_index,y\_index]=value}{sets the value of the indexed bin}
\lineii{len(histogram)}{returns the size of the histogram, i.e nx$\times$ny}
\end{tableii}


\begin{seealso}
For the special meaning and details please consult the GNU Scientific Library
reference.
\end{seealso}

\section{\protect\class{histogram_pdf} and \protect\class{histogram2d_pdf}}

To be implemented\dots

%% Local Variables:
%% mode: LaTeX
%% mode: auto-fill
%% fill-column: 90
%% indent-tabs-mode: nil
%% ispell-dictionary: "american"
%% reftex-fref-is-default: nil
%% TeX-auto-save: t
%% TeX-command-default: "pdfeLaTeX"
%% TeX-master: "pygsl"
%% TeX-parse-self: t
%% End:


\chapter[\protect\module{pygsl.rng} --- Random Number Generators]
{\protect\module{pygsl.rng} \\ Random Number Generators}
\label{cha:rng-module}
% $Id$
% pygsl/doc/rng.tex

\declaremodule{standard}{pygsl.rng}%
\moduleauthor{Pierre Schnizer}{schnizer@users.sourceforge.net}%
\moduleauthor{Original Author: Achim G\"adke}{achimgaedke@users.sourceforge.net}%

This chapter introduces the random number generator type provided by \module{pygsl}.

\section{Random Number Generators}

All random number generatores are the same python type (PyGSL_rng), but using the
approbriate GSL random generator for generating the random numbers. Use the method
\code{name} to get the name of the rng used internally.

Methods of
this type \pytype{rng} provide the transformation to different probability
distributions and give access to basic properties of random number generators. 
All methods allow to pass one optional integer. Then the method will be evaluated n times and the result
will be returned as an array.

\begin{pytypedesc}{rng}{\texttt{string} typenamme \code{|} \class{rng} r}
  This base class can be instantiated by its name
\begin{verbatim}
import pygsl.rng
my_ran0=pygsl.rng.ran0()
\end{verbatim}
.
\end{pytypedesc}
The type of the allocated generator is given by the method
\begin{methoddesc}{name}{}
  which returns its name as string.
\end{methoddesc}
All generators can be seeded with
\begin{methoddesc}{set}{seed}
  which sets the internal seed according to the positive integer {\tt seed}. Zero as seed
  has a special meaning, please read details in the gsl reference.
\end{methoddesc}
The basic returned number type is integer, these are generated by
\begin{methoddesc}{get}{}
  which returns the next number of the pseudo random sequence.
\end{methoddesc}
All methods support internal sampling; i.e each method has an optional integer. 
If given it will return a sample of the approbriate size.
\begin{methoddesc}{get}{|n}
  will return the next n numbers of the pseudo random sequence.
\end{methoddesc}

Basic information about these numbers can be obtained by
\begin{methoddesc}{max}{}
  maximum number of this sequence and
\end{methoddesc}
\begin{methoddesc}{min}{}
  minimum number of this sequence.
\end{methoddesc}
Implemented uniform probability densities are:
\begin{methoddesc}{uniform}{}
  returns a real number between $[0,1)$.
\end{methoddesc}
\begin{methoddesc}{uniform_pos}{}
  returns a real number between $(0,1)$ --- this excludes 0.
\end{methoddesc}
\begin{methoddesc}{uniform_int}{upper limit}
  returns an integer from 0 to the upper limit (exclusive). If this limit is larger than
  the number of return values of the underlying generator, \exception{pygsl.gsl_Error} is
  raised.
\end{methoddesc}
Furthermore a lot of derived probability densities can be used:
\begin{methoddesc}{gaussian}{sigma}
  gaussian distribution with mean 0 and given sigma \hfill returns {\tt float}
\end{methoddesc}
\begin{methoddesc}{gaussian\_ratio\_method}{sigma}
  gaussian distribution with mean 0 and given sigma.  This variate uses the
  Kinderman-Monahan ratio method.  \hfill returns {\tt float}
\end{methoddesc}
\begin{methoddesc}{ugaussian}{}
  gaussian distribution with unit sigma and mean 0.  \hfill returns {\tt float}
\end{methoddesc}
\begin{methoddesc}{ugaussian\_ratio\_method}{}
  gaussian distribution with unit sigma and mean 0.  This variate uses the
  Kinderman-Monahan ratio method.  \hfill returns {\tt float}
\end{methoddesc}
\begin{methoddesc}{gaussian\_tail}{sigma, a}
  upper tail of a Gaussian distribution with standard deviation sigma>0.  \hfill returns
  {\tt float}
\end{methoddesc}
\begin{methoddesc}{ugaussian\_tail}{a}
  upper tail of a Gaussian distribution with unit standard deviation.  \hfill returns {\tt
    float}
\end{methoddesc}
\begin{methoddesc}{bivariate\_gaussian}{sigma\_x, sigma\_y, rho}
  pair of correlated gaussian variates, with mean zero, correlation coefficient rho and
  standard deviations sigma\_x and sigma\_y in the x and y directions \hfill returns~{\tt
    (float,float)}
\end{methoddesc}
\begin{methoddesc}{exponential}{mu}
  \hfill returns {\tt float}
\end{methoddesc}
\begin{methoddesc}{laplace}{mu}
  \hfill returns {\tt float}
\end{methoddesc}
\begin{methoddesc}{exppow}{mu, a}
  \hfill returns {\tt float}
\end{methoddesc}
\begin{methoddesc}{cauchy}{mu}
  \hfill returns {\tt float}
\end{methoddesc}
\begin{methoddesc}{rayleigh}{sigma}
  \hfill returns {\tt float}
\end{methoddesc}
\begin{methoddesc}{rayleigh\_tail}{a, sigma}
  \hfill returns {\tt float}
\end{methoddesc}
\begin{methoddesc}{levy}{mu,a}
  \hfill returns {\tt float}
\end{methoddesc}
\begin{methoddesc}{levy_skew}{mu,a,beta}
  \hfill returns {\tt float}
\end{methoddesc}
\begin{methoddesc}{gamma}{a, b}
  \hfill returns {\tt float}
\end{methoddesc}
\begin{methoddesc}{gamma\_int}{long a}
  \hfill returns {\tt float}
\end{methoddesc}
\begin{methoddesc}{flat}{a, b}
  \hfill returns {\tt float}
\end{methoddesc}
\begin{methoddesc}{lognormal}{zeta, sigma}
  \hfill returns {\tt float}
\end{methoddesc}
\begin{methoddesc}{chisq}{nu}
  \hfill returns {\tt float}
\end{methoddesc}
\begin{methoddesc}{fdist}{nu1, nu2}
  \hfill returns {\tt float}
\end{methoddesc}
\begin{methoddesc}{tdist}{nu}
  \hfill returns {\tt float}
\end{methoddesc}
\begin{methoddesc}{beta}{a, b}
  \hfill returns {\tt float}
\end{methoddesc}
\begin{methoddesc}{logistic}{mu}
  \hfill returns {\tt float}
\end{methoddesc}
\begin{methoddesc}{pareto}{a, b}
  \hfill returns {\tt float}
\end{methoddesc}
\begin{methoddesc}{dir\_2d}{}
  \hfill returns {\tt (float, float)}
\end{methoddesc}
\begin{methoddesc}{dir\_2d\_trig\_method}{}
  \hfill returns {\tt (float, float)}
\end{methoddesc}
\begin{methoddesc}{dir\_3d}{}
  \hfill returns {\tt (float, float, float)}
\end{methoddesc}
\begin{methoddesc}{dir\_nd}{int n}
  \hfill returns {\tt (float, \dots, float)}
\end{methoddesc}
\begin{methoddesc}{weibull}{mu, a}
  \hfill returns {\tt float}
\end{methoddesc}
\begin{methoddesc}{gumbel1}{a, b}
  \hfill returns {\tt float}
\end{methoddesc}
\begin{methoddesc}{gumbel2}{}
\end{methoddesc}
\begin{methoddesc}{poisson}{}
\end{methoddesc}
\begin{methoddesc}{bernoulli}{}
\end{methoddesc}
\begin{methoddesc}{binomial}{}
\end{methoddesc}
\begin{methoddesc}{negative\_binomial}{}
\end{methoddesc}
\begin{methoddesc}{pascal}{}
\end{methoddesc}
\begin{methoddesc}{geometric}{}
\end{methoddesc}
\begin{methoddesc}{hypergeometric}{}
\end{methoddesc}
\begin{methoddesc}{logarithmic}{}
\end{methoddesc}
\begin{methoddesc}{landau}{}
\end{methoddesc}
\begin{methoddesc}{erlang}{}
\end{methoddesc}


The different generator classes are created according to the output of
\code{gsl_rng_types_setup()} when the \module{pygsl.rng} is loaded. Here is the list of
children from \class{rng} for gsl-1.2: \newline \class{rng_borosh13}, \class{rng_coveyou},
\class{rng_cmrg}, \class{rng_fishman18}, \class{rng_fishman20}, \class{rng_fishman2x},
\class{rng_gfsr4}, \class{rng_knuthran}, \class{rng_knuthran2}, \class{rng_lecuyer21},
\class{rng_minstd}, \class{rng_mrg}, \class{rng_mt19937}, \class{rng_mt19937_1999},
\class{rng_mt19937_1998}, \class{rng_r250}, \class{rng_ran0}, \class{rng_ran1},
\class{rng_ran2}, \class{rng_ran3}, \class{rng_rand}, \class{rng_rand48},
\class{rng_random128_bsd}, \class{rng_random128_glibc2}, \class{rng_random128_libc5},
\class{rng_random256_bsd}, \class{rng_random256_glibc2}, \class{rng_random256_libc5},
\class{rng_random32_bsd}, \class{rng_random32_glibc2}, \class{rng_random32_libc5},
\class{rng_random64_bsd}, \class{rng_random64_glibc2}, \class{rng_random64_libc5},
\class{rng_random8_bsd}, \class{rng_random8_glibc2}, \class{rng_random8_libc5},
\class{rng_random_bsd}, \class{rng_random_glibc2}, \class{rng_random_libc5},
\class{rng_randu}, \class{rng_ranf}, \class{rng_ranlux}, \class{rng_ranlux389},
\class{rng_ranlxd1}, \class{rng_ranlxd2}, \class{rng_ranlxs0}, \class{rng_ranlxs1},
\class{rng_ranlxs2}, \class{rng_ranmar}, \class{rng_slatec}, \class{rng_taus},
\class{rng_taus2}, \class{rng_taus113}, \class{rng_transputer}, \class{rng_tt800},
\class{rng_uni}, \class{rng_uni32}, \class{rng_vax}, \class{rng_waterman14}, and
\class{rng_zuf}.  
\newline 

The default generator of the \class{rng} defaults to {\tt rng_mt19937} but can be set from the
environment variable \envvar{GSL_RNG_TYPE} using the function \function{rng.env_setup()}.

\section{Probability Density Functions}


\section{Using probability densities with random number generators}


%% Local Variables:
%% mode: LaTeX
%% mode: auto-fill
%% fill-column: 90
%% indent-tabs-mode: nil
%% ispell-dictionary: "british"
%% reftex-fref-is-default: nil
%% TeX-auto-save: t
%% TeX-command-default: "pdfeLaTeX"
%% TeX-master: "pygsl"
%% TeX-parse-self: t
%% End:


%\chapter[\protect\module{pygsl.sf} --- Special Functions]
%{\protect\module{pygsl.sf} \\ Special Functions}
%\label{cha:sf-module}
%\declaremodule{extension}{pygsl.sf}
\moduleauthor{Achim G\"adke}{achimgaedke@users.sourceforge.net}

This chapter shows you the list of implemented special function and explains
details of error handling and return values.

\section{Function list}

\begin{longtableii}{l|l}{texttt}{Function}{Description}
\lineii{}{ToDo}
\end{longtableii}

\section{Return values}

\section{Error handling}


\chapter[\protect\module{pygsl.sum} --- Series acceleration]{
  \protect\module{pygsl.sum} \\ Series acceleration}
\label{cha:sum-module}

\declaremodule{extension}{pygsl.sum}
\modulesynopsis{Series acceleration.}

This chapter describes the use of the series acceleration tools based
on the Levin $u$-transform.  This method takes a small number of terms
from the start of a series and uses a systematic approximation to
compute an extrapolated value and an estimate of its error. The
$u$-transform works for both convergent and divergent series,
including asymptotic series.

\begin{equation}
  \label{eq:levin}
  \function{levin_sum}\code{(a)} = (A, \epsilon)
  \qquad\text{where}
  \qquad
  A \approx \sum_{n=0}^{\infty} a_{n} \pm \epsilon, 
\end{equation}
$\code{a} = [a_{0}, a_{1}, \ldots, a_{n}]$, and $\epsilon$ is an
estimate of the absolute error.

Note: This function is intended for summing analytic series where each
term is known to high accuracy, and the rounding errors are assumed to
originate from finite precision. They are taken to be relative errors
of order \constant{GSL_DBL_EPSILON} for each term (as defined in the
\GSL{} source code).

\section{Function list}
\begin{funcdesc}{levin_sum}{a, truncate=False, info_dict=None}
  Return ($A, \epsilon$) where $A$ is the approximated sum of the
  series~(\ref{eq:levin}) and $\epsilon$ is its absolute error
  estimate.

  The calculation of the error in the extrapolated value is an
  O$(N^2)$ process, which is expensive in time and memory.  A full
  table of intermediate values and derivatives through to O$(N)$ must
  be computed and stored, but this does give a reliable error
  estimate.

  A faster but less reliable method which estimates the error from the
  convergence of the extrapolated value is employed if \var{truncate}
  is \code{True}.  This attempts to estimate the error from the
  ``truncation error'' in the extrapolation, the difference between
  the final two approximations. Using this method avoids the need to
  compute an intermediate table of derivatives because the error is
  estimated from the behavior of the extrapolated value
  itself. Consequently this algorithm is an O$(N)$ process and only
  requires O$(N)$ terms of storage. If the series converges sufficiently
  fast then this procedure can be acceptable. It is appropriate to use
  this method when there is a need to compute many extrapolations of
  series with similar convergence properties at high-speed. For
  example, when numerically integrating a function defined by a
  parameterized series where the parameter varies only slightly. A
  reliable error estimate should be computed first using the full
  algorithm described above in order to verify the consistency of the
  results.

  If a dictionary is passed as \var{info_dict}, then two entries will
  be added: \var{info_dict}\code{['terms_used']} will be the number of
  terms used\footnote{Note that it appears that this is the number of
    terms \emph{beyond} the first term that are used.  I.e.\ there are
    a total of $\var{terms_used}+1$ terms:
    \begin{equation}
      \var{sum_plain} = 
      \sum_{n=0}^{\var{terms_used}}
      a_{n}
    \end{equation}}
  and \var{info_dict}\code{['sum_plain']} will be the sum of these terms without
  acceleration.
\end{funcdesc}

\section{Further Reading}
For details on the underlying implementation of these functions please
consult the \GSL{} reference manual.  The algorithms used by these
functions are described Fessler \textit{et al.} (1983).  The theory of
the $u$-transform was presented Levin in 1973, and a review paper from
2000 by Homeier is available online.

\begin{seealso}
  \seetext{T.~Fessler, W.~F.~Ford, D.~A.~Smith, \textit{hurry: An
      acceleration algorithm for scalar sequences and series}. ACM
    Transactions on Mathematical Software, \textbf{9}(3):346--354,
    (1983), and Algorithm 602 9(3):355--357, 1983.}
  \seetext{D.~Levin, \textit{Development of Non-Linear Transformations
      for Improving Convergence of Sequences,} Intern.~J.~Computer
    Math. \textbf{B3}:371--388, (1973).}
  \seetitle[http://arXiv.org/abs/math/0005209]{Herbert H.~H.~Homeier,
    \textit{Scalar Levin-Type Sequence Transformations.}}{}
\end{seealso}
%%% Local Variables: 
%%% mode: latex
%%% TeX-master: "ref"
%%% End: 

\chapter[\protect\module{pygsl.statistics} --- Statistics
functions]{\protect\module{pygsl.statistics} \\ Statistics functions}
\label{cha:statistics-module}

\declaremodule{extension}{pygsl.statistics}
\moduleauthor{Pierre Schnizer}{schnizer@users.sourceforge.net}
\moduleauthor{Original Author: Jochen K\"upper}{jochen@jochen-kuepper.de}
\modulesynopsis{Statistical functions.}

\index{mean}
\index{standard deviation}
\index{variance}
\index{estimated standard deviation}
\index{estimated variance}
\index{t-test}
\index{range}
\index{min}
\index{max}
\index{kurtosis}
\index{skewness}
\index{autocorrelation}
\index{covariance}

\begin{quote}
   This chapter describes the statistical functions in the library.  The basic
   statistical functions include routines to compute the mean, variance and
   standard deviation. More advanced functions allow you to calculate absolute
   deviations, skewness, and kurtosis as well as the median and arbitrary
   percentiles.
\end{quote}

The algorithms provided here use recurrence relations to compute average
quantities in a stable way, without large intermediate values that might
overflow.  All functions work on any Python sequence (of appropriate
data-type), but see section \ref{sec:stat:speed-considerations} for advantages
and drawbacks of different kinds of input data.

\begin{seealso}
   For details on the underlying implementation of these functions please
   consult the \GSL{} reference manual.
\end{seealso}



\section{Organization of the module}
\label{sec:stat:organization}

Individual parts of the \gsl{} functions names, providing artificial namespaces
in C, are mapped to modules and submodules in \pygsl{}.  That is,
\cfunction{gsl_stats_mean} can be found as \function{pygsl.statistics.mean} and
\cfunction{gsl_stats_long_mean} as \function{pygsl.statistics.long.mean}.

The functions in the module are available in versions for datasets in the
standard and \numpy{} floating-point and integer types. The generic versions
available in the \module{pygsl.statistics} module are using the generic \gsl{}
\ctype{double} versions.  The submodules use \gsl{} functions according to the
submodule name, e.g. long for \module{pygsl.statistics.long}.

Implemented submodules are \module{char}, \module{uchar}, \module{short},
\module{int}, \module{long}, \module{float}, and \module{double}. The latter
one also serves as default and is used whenever you don't expclicitely state a
different datatype. In most cases it is appropriate to simply use the default
implementation as it covers the widest range of the real space, offers high
precision, and as such is simple to use. If you have a sequence of all integer
values it is straightforward to use \module{pygsl.statistics.long} functions as
these use an implementation corresponding to Pythons \class{Float}-type. These
implemented submodules represent all numeric datatypes available in Python
(\class{Int}, \class{Float}) besides \class{Long Int} which has no
representation in standard C, as well as all numeric datatypes available in
\numpy{} that have corresponding implementations in \gsl{} (on 32 bit systems
these are: Character, UnsigendInt8, Int16, Int32, Int, Float32, Float).



\section{Available functions}
\label{sec:stat:available-functions}

\subsection{Mean, Standard Deviation, and Variance}
\label{sec:stat:mean-stddev-var}

\begin{funcdesc}{mean}{x}\index{mean}
   Arithmetic mean (\emph{sample mean}) of \var{x}:
   \begin{equation}
      \hat\mu = \frac{1}{N} \sum x_i
   \end{equation}
\end{funcdesc}

\begin{funcdesc}{variance}{x}\index{variance}
   Estimated (\emph{sample}) variance of \var{x}:
   \begin{equation}
      \hat\sigma^2 = \frac{1}{N-1} \sum (x_i - \hat\mu)^2
   \end{equation}
   This function computes the mean via a call to \function{mean}.  If you have
   already computed the mean then you can pass it directly to
   \function{variance_m}.
\end{funcdesc}

\begin{funcdesc}{variance_m}{x, mean}\index{variance}
   Estimated (\emph{sample}) variance of \var{x} relative to \var{mean}:
   \begin{equation}
      \hat\sigma^2 = \frac{1}{N-1} \sum (x_i - mean)^2
   \end{equation}
\end{funcdesc}

\begin{funcdesc}{sd}{x}
\end{funcdesc}
\begin{funcdesc}{sd_m}{x, mean}\index{sd}\index{mean}
   The standard deviation is defined as the square root of the variance of
   \var{x}.  These functions returns the square root of the respective
   variance-functions above.
\end{funcdesc}

\begin{funcdesc}{variance_with_fixed_mean}{x, mean}\index{variance}\index{mean}
   Compute an unbiased estimate of the variance of \var{x} when the population
   mean \var{mean} of the underlying distribution is known \emph{a priori}.  In
   this case the estimator for the variance uses the factor $1/N$ and the
   sample mean $\hat\mu$ is replaced by the known population mean $\mu$:
   \begin{equation}
      \hat\sigma^2 = \frac{1}{N} \sum (x_i - \mu)^2
   \end{equation}
\end{funcdesc}


\subsection{Absolute deviation}
\label{sec:stat:absolute-deviation}

\begin{funcdesc}{absdev}{data}
   Compute the absolute deviation from the mean of \var{data} The absolute
   deviation from the mean is defined as
   \begin{equation}
      absdev  = (1/N) \sum |x_i - \hat\mu|
   \end{equation}
   where $x_i$ are the elements of the dataset \var{data}.  The absolute
   deviation from the mean provides a more robust measure of the width of a
   distribution than the variance.  This function computes the mean of
   \var{data} via a call to \function{mean}.
\end{funcdesc}

\begin{funcdesc}{absdev_m}{data, mean}
   Compute the absolute deviation of the dataset \var{data} relative to the
   given value of \var{mean}
   \begin{equation}
      absdev  = (1/N) \sum |x_i - mean|
   \end{equation}
   This function is useful if you have already computed the mean of \var{data}
   (and want to avoid recomputing it), or wish to calculate the absolute
   deviation relative to another value (such as zero, or the median).
\end{funcdesc}


\subsection{Higher moments (skewness and kurtosis)}
\label{sec:stat:higher-moments}

\begin{funcdesc}{skew}{data}
   Compute the skewness of \var{data}.  The skewness is defined as
   \begin{equation}
      skew = (1/N) \sum ((x_i - \hat\mu)/\hat\sigma)^3
   \end{equation}
   where $x_i$ are the elements of the dataset \var{data}.  The skewness
   measures the asymmetry of the tails of a distribution.
   
   The function computes the mean and estimated standard deviation of
   \var{data} via calls to \function{mean} and \function{sd}.
\end{funcdesc}


\begin{funcdesc}{skew_m_sd}{data, mean, sd}
   Compute the skewness of the dataset \var{data} using the given values of the
   mean \var{mean} and standard deviation var{sd}
   \begin{equation}
      skew = (1/N) \sum ((x_i - mean)/sd)^3
   \end{equation}
   These functions are useful if you have already computed the mean and
   standard deviation of \var{data} and want to avoid recomputing them.
\end{funcdesc}


\begin{funcdesc}{kurtosis}{data}
   Compute the kurtosis of \var{data}.  The kurtosis is defined as
   \begin{equation}
      kurtosis = ((1/N) \sum ((x_i - \hat\mu)/\hat\sigma)^4) - 3
   \end{equation}
   The kurtosis measures how sharply peaked a distribution is, relative to its
   width.  The kurtosis is normalized to zero for a gaussian distribution.
\end{funcdesc}


\begin{funcdesc}{kurtosis_m_sd}{data, mean, sd}
   This function computes the kurtosis of the dataset \var{data} using the
   given values of the mean \var{mean} and standard deviation \var{sd}
   \begin{equation}
      kurtosis = ((1/N) \sum ((x_i - mean)/sd)^4) - 3
   \end{equation}
   This function is useful if you have already computed the mean and standard
   deviation of \var{data} and want to avoid recomputing them.
\end{funcdesc}



\subsection{Autocorrelation}
\label{sec:stat:autocorrelation}

\begin{funcdesc}{lag1_autocorrelation}{x}
   Computes the lag-1 autocorrelation of the dataset \var{x}
   \begin{equation}
      a_1 = \frac{\sum^{n}_{i = 1} (x_{i} - \hat\mu) (x_{i-1} - \hat\mu)}{
         \sum^{n}_{i = 1} (x_{i} - \hat\mu) (x_{i} - \hat\mu)}
   \end{equation}
 \end{funcdesc}

\begin{funcdesc}{lag1_autocorrelation_m}{x, mean}
   Computes the lag-1 autocorrelation of the dataset \var{x} using the given
   value of the mean \var{mean}.
   \begin{equation}
      a_1 = \frac{\sum_{i = 1}^{n} (x_{i} - \var{mean}) (x_{i-1} - \var{mean})}{
         \sum^{n}_{i = 1} (x_{i} - \var{mean}) (x_{i} - \var{mean})}
   \end{equation}
\end{funcdesc}



\subsection{Covariance}
\label{sec:stat:covariance}

\begin{funcdesc}{covariance}{x, y}
   Computes the covariance of the datasets \var{x} and \var{y} which must be of
   same length.
   \begin{equation}
      c = \frac{1}{n-1} \sum^{n}_{i=1} (x_i - \hat x) (y_i - \hat y)
   \end{equation}
\end{funcdesc}

\begin{funcdesc}{lag1_autocorrelation_m}{x, y, mean\_x, mean\_y}
   Computes the covariance of the datasets \var{x} and \var{y} using the given
   values of the means \var{mean\_x} and \var{mean\_y}. The datasets \var{x}
   and \var{y} must be of equal length.
   \begin{equation}
      c = \frac{1}{n-1} \sum^{n}_{i=1} (x_i - \var{mean\_x}) (y_i -
      \var{mean\_y})
   \end{equation}
\end{funcdesc}




\subsection{Maximum and Minimum values}
\label{sec:stat:max-min-value}


\begin{funcdesc}{max}{data}
   This function returns the maximum value in \var{data}.  The maximum value is
   defined as the value of the element $x_i$ which satisfies $x_i \ge x_j$ for
   all $j$.
   
   If you want instead to find the element with the largest absolute magnitude
   you will need to apply `fabs' or `abs' to your data before calling this
   function.
\end{funcdesc}

\begin{funcdesc}{min}{data}
   This function returns the minimum value in \var{data}. The maximum value is
   defined as the value of the element $x_i$ which satisfies $x_i \le x_j$ for
   all $j$.
   
   If you want instead to find the element with the smallest absolute magnitude
   you will need to apply `fabs' or `abs' to your data before calling this
   function.
\end{funcdesc}

\begin{funcdesc}{minmax}{data}
   This function returns both the minimum and maximum values of \var{data},
   determined in a single pass.
\end{funcdesc}

\begin{funcdesc}{max_index}{data}
   This function returns the index of the maximum value in \var{data}.  The
   maximum value is defined as the value of the element $x_i$ which satisfies
   $x_i \ge x_j$ for all $j$.  When there are several equal maximum elements
   then the first one is chosen.
\end{funcdesc}

\begin{funcdesc}{min_index}{data}
   This function returns the index of the minimum value in \var{data}.  The
   minimum value is defined as the value of the element $x_i$ which satisfies
   $x_i \le x_j$ for all $j$.  When there are several equal minimum elements
   then the first one is chosen.
\end{funcdesc}

\begin{funcdesc}{minmax_index}{data}
   This function returns the indexes of the minimum and maximum values of
   \var{data}, determined in a single pass.
\end{funcdesc}



\subsection{Median and Percentiles}
\label{sec:stat:median-percentiles}

The median and percentile functions described in this section operate on sorted
data.  For convenience we use "quantiles", measured on a scale of 0 to 1,
instead of percentiles (which use a scale of 0 to 100).

\begin{funcdesc}{median_from_sorted_data}{data}
   This function returns the median value of \var{data}.  The elements of the
   array must be in ascending numerical order.  There are no checks to see
   whether the data are sorted, so the function \function{sort} should always
   be used first.
   
   When the dataset has an odd number of elements the median is the value of
   element (n-1)/2.  When the dataset has an even number of elements the median
   is the mean of the two nearest middle values, elements (n-1)/2 and n/2.
   Since the algorithm for computing the median involves interpolation this
   function always returns a floating-point number, even for integer data
   types.
\end{funcdesc}

\begin{funcdesc}{quantile_from_sorted_data}{data, F}
   This function returns a quantile value of \var{data}.  The elements of the
   array must be in ascending numerical order.  The quantile is determined by
   the \var{F}, a fraction between 0 and 1.  For example, to compute the value
   of the 75th percentile \var{F} should have the value 0.75.
   
   There are no checks to see whether the data are sorted, so the function
   \function{sort} should always be used first.
   
   The quantile is found by interpolation, using the formula
   \begin{equation}
      quantile = (1 - \delta) x_i + \delta x_{i+1}
   \end{equation}
   where $i$ is $floor((n - 1)f)$ and $\delta$ is $(n-1)f - i$.
   
   Thus the minimum value of the array (\var{data[0]}) is given by \var{F}
   equal to zero, the maximum value (\var{data[-1]}) is given by \var{F} equal
   to one and the median value is given by \var{F} equal to 0.5.  Since the
   algorithm for computing quantiles involves interpolation this function
   always returns a floating-point number, even for integer data types.
\end{funcdesc}


\subsection{Weighted Samples}
\label{sec:weighted-samples}

The functions described in this section allow the computation of statistics for
weighted samples.  The functions accept an array of samples, $x_i$, with
associated weights, $w_i$.  Each sample $x_i$ is considered as having been
drawn from a Gaussian distribution with variance $\sigma_i^2$.  The sample
weight $w_i$ is defined as the reciprocal of this variance, $w_i =
1/\sigma_i^2$.  Setting a weight to zero corresponds to removing a sample from
a dataset.

\begin{funcdesc}{wmean}{w, data}
   This function returns the weighted mean of the dataset \var{data} using the
   set of weights \var{w}.  The weighted mean is defined as
   \begin{equation}
      \hat\mu = (\sum w_i x_i) / (\sum w_i)
   \end{equation}
\end{funcdesc}

\begin{funcdesc}{wvariance }{w, data}
   This function returns the estimated variance of the dataset \var{data},
   using the set of weights \var{w}.  The estimated variance of a weighted
   dataset is defined as
   \begin{equation}
      \hat\sigma^2 = ((\sum w_i)/((\sum w_i)^2 - \sum (w_i^2))) \sum w_i (x_i - \hat\mu)^2
   \end{equation}
   Note that this expression reduces to an unweighted variance with the
   familiar $1/(N-1)$ factor when there are $N$ equal non-zero weights.
\end{funcdesc}

\begin{funcdesc}{wvariance_m}{w, data, wmean}
   This function returns the estimated variance of the weighted dataset
   \var{data} using the given weighted mean \var{wmean}.
\end{funcdesc}

\begin{funcdesc}{wsd}{w, data}
   The standard deviation is defined as the square root of the variance.  This
   function returns the square root of the corresponding variance function
   \function{wvariance} above.
\end{funcdesc}

\begin{funcdesc}{wsd_m}{w, data, wmean}
   This function returns the square root of the corresponding variance function
   \function{wvariance_m} above.
\end{funcdesc}

\begin{funcdesc}{wvariance_with_fixed_mean}{w, data, mean}
   This function computes an unbiased estimate of the variance of weighted
   dataset \var{data} when the population mean \var{mean} of the underlying
   distribution is known _a priori_.  In this case the estimator for the
   variance replaces the sample mean $\hat\mu$ by the known population mean
   $\mu$,
   \begin{equation}
      \hat\sigma^2 = (\sum w_i (x_i - \mu)^2) / (\sum w_i)
   \end{equation}
\end{funcdesc}

\begin{funcdesc}{wsd_with_fixed_mean}{w, data, mean}
   The standard deviation is defined as the square root of the variance.  This
   function returns the square root of the corresponding variance function
   above.
\end{funcdesc}

\begin{funcdesc}{wabsdev}{w, data}
   This function computes the weighted absolute deviation from the weighted
   mean of \var{data}.  The absolute deviation from the mean is defined as
   \begin{equation}
      absdev = (\sum w_i |x_i - \hat\mu|) / (\sum w_i)
   \end{equation}
\end{funcdesc}

\begin{funcdesc}{wabsdev_m}{w, data, wmean}
   This function computes the absolute deviation of the weighted dataset DATA
   about the given weighted mean WMEAN.
\end{funcdesc}

\begin{funcdesc}{wskew}{w, data}
   This function computes the weighted skewness of the dataset DATA.
   \begin{equation}
      skew = (\sum w_i ((x_i - xbar)/\sigma)^3) / (\sum w_i)
   \end{equation}
\end{funcdesc}

\begin{funcdesc}{wskew_m_sd}{w, data, mean, wsd}
   This function computes the weighted skewness of the dataset \var{data} using
   the given values of the weighted mean and weighted standard deviation,
   \var{wmean} and \var{wsd}.
\end{funcdesc}

\begin{funcdesc}{wkurtosis}{w, data}
   This function computes the weighted kurtosis of the dataset \var{data}. The
   kurtosis is defined as 
   \begin{equation}
      kurtosis = ((\sum w_i ((x_i - xbar)/sigma)^4) / (\sum w_i)) - 3
   \end{equation}
\end{funcdesc}

\begin{funcdesc}{wkurtosis_m_sd}{w, data, mean, wsd}
   This function computes the weighted kurtosis of the dataset \var{data} using
   the given values of the weighted mean and weighted standard deviation,
   \var{wmean} and \var{wsd}.
\end{funcdesc}





\section{Further Reading}
\label{sec:stat:further-reading}

See the \gsl{} reference manual for a description of all available functions
and the calculations they perform.

The standard reference for almost any topic in statistics is the multi-volume
\emph{Advanced Theory of Statistics} by Kendall and Stuart.  Many statistical
concepts can be more easily understood by a Bayesian approach.  The book by
Gelman, Carlin, Stern and Rubin gives a comprehensive coverage of the subject.
For physicists the Particle Data Group provides useful reviews of Probability
and Statistics in the "Mathematical Tools" section of its Annual Review of
Particle Physics.
   
\begin{seealso}
   \seetext{Maurice Kendall, Alan Stuart, and J.\ Keith Ord: \emph{The Advanced
         Theory of Statistics} (multiple volumes) reprinted as \emph{Kendall's
         Advanced Theory of Statistics}.  Wiley, ISBN 047023380X.}
   
   \seetext{Andrew Gelman, John B.\ Carlin, Hal S.\ Stern, Donald B.\ Rubin:
      \emph{Bayesian Data Analysis}.  Chapman \& Hall, ISBN 0412039915.}
   
   \seetext{R.M.\ Barnett et al: Review of Particle Properties. \emph{Physical
         Review} \textbf{D54}, 1 (1996).}
   
   \seetext{D.E.\ Groom et al., \emph{The European Physical Journal}
      \textbf{C15}, 1 (2000) and \emph{2001 off-year partial update for the
         2002 edition} available on the PDG WWW pages (URL:
      \url{http://pdg.lbl.gov/}).}
   
   \seetext{Siegmund Brandt: \emph{Datenanalyse}, 4th ed. 1999, Spektrum,
      Heidelberg, ISBN 3827401585.}  
   
   \seetext{Siegmund Brandt: \emph{Data Analysis}. 3rd ed. 1998, Springer,
      Berlin, ISBN 0387984984.}
\end{seealso}


%% Local Variables:
%% mode: LaTeX
%% mode: auto-fill
%% fill-column: 79
%% indent-tabs-mode: nil
%% ispell-dictionary: "british"
%% reftex-fref-is-default: nil
%% TeX-auto-save: t
%% TeX-command-default: "pdfeLaTeX"
%% TeX-master: "pygsl"
%% TeX-parse-self: t
%% End:


[common]
sensorid = default

[virtual_file_system]
data_fs_url = default
fs_url = default

[session]
timeout = 30

[daemon]
;user = conpot
;group = conpot

[json]
enabled = False
filename = /var/log/conpot.json

[sqlite]
enabled = False

[mysql]
enabled = False
device = /tmp/mysql.sock
host = localhost
port = 3306
db = conpot
username = conpot
passphrase = conpot
socket = tcp        ; tcp (sends to host:port), dev (sends to mysql device/socket file)

[syslog]
enabled = False
device = /dev/log
host = localhost
port = 514
facility = local0
socket = dev        ; udp (sends to host:port), dev (sends to device)

[hpfriends]
enabled = False
host = hpfriends.honeycloud.net
port = 20000
ident = 3Ykf9Znv
secret = 4nFRhpm44QkG9cvD
channels = ["conpot.events", ]

[taxii]
enabled = False
host = taxiitest.mitre.org
port = 80
inbox_path = /services/inbox/default/
use_https = False

[fetch_public_ip]
enabled = True
urls = ["http://whatismyip.akamai.com/", "http://wgetip.com/"]

[change_mac_addr]
enabled = False
iface = eth0
addr = 00:de:ad:be:ef:00


\appendix

\chapter[\protect\module{pygsl.ieee} --- Floating Point Unit Support]
{\protect\module{pygsl.ieee} \\ Floating Point Unit Support}
\label{cha:ieee-module}
\declaremodule{extension}{pygsl.ieee}
\moduleauthor{Achim G\"adke}{achimgaedke@users.sourceforge.net}

This chapter lists features to configure the ``Floating Point Unit'' of your machine.
The exact behaviour of your Floating Point Unit can't be discussed here in general --- its just machine type dependent.

\begin{funcdesc} {set_mode}{int precision, int rounding, int exception\_mask}
the mode has effect on the behaviour during calcualtion, e.g. division by zero or rounding.

The following constants are used as precision argument:
\begin{tableii}{l|l}{constant}{mode value}{definition via gsl}
\lineii{single\_precision}{\code{GSL\_IEEE\_SINGLE\_PRECISION}}
\lineii{double\_precision}{\code{GSL\_IEEE\_DOUBLE\_PRECISION}}
\lineii{extended\_precision}{\code{GSL\_IEEE\_EXTENDED\_PRECISION}}
\end{tableii}
Possible round arguments are:
\begin{tableii}{l|l}{constant}{mode value}{definition via gsl}
\lineii{round\_to\_nearest}{\code{GSL\_IEEE\_ROUND\_TO\_NEAREST}}
\lineii{round\_down}{\code{GSL\_IEEE\_ROUND\_DOWN}}
\lineii{round\_up}{\code{GSL\_IEEE\_ROUND\_UP}}
\lineii{round\_to\_zero}{\code{GSL\_IEEE\_ROUND\_TO\_ZERO}}
\end{tableii}
These exception arguments can be added.
\constant{mask\_all} is the sum of all 5 \constant{mask\_*} constants.
\begin{tableii}{l|l}{constant}{mode value}{definition via gsl}
\lineii{mask\_invalid}{\code{GSL\_IEEE\_MASK\_INVALID}}
\lineii{mask\_denormalized}{\code{GSL\_IEEE\_MASK\_DENORMALIZED}}
\lineii{mask\_division\_by\_zero}{\code{GSL\_IEEE\_MASK\_DIVISION\_BY\_ZERO}}
\lineii{mask\_overflow}{\code{GSL\_IEEE\_MASK\_OVERFLOW}}
\lineii{mask\_underflow}{\code{GSL\_IEEE\_MASK\_UNDERFLOW}}
\lineii{mask\_all}{\code{GSL\_IEEE\_MASK\_ALL}}
\lineii{trap\_inexact}{\code{GSL\_IEEE\_TRAP\_INEXACT}}
\end{tableii}
\end{funcdesc}

\begin{funcdesc} {env\_setup}{}
sets the ieee mode from \envvar{GSL\_IEEE\_MODE}. This is not called any more
automatically  when importing the  \module{pygsl}.
\end{funcdesc}

\begin{funcdesc} {bin\_repr}{float value}
%\cfunction{gsl_ieee_double_to_rep}
returns the binary representation as tuple with the following contents:
\code{(int sign, string mantissa, int exponent, int type)}
These values are used as \constant{type} in \function{bin\_repr}:
\begin{tableii}{l|l}{constant}{type value}{definition via gsl}
\lineii{type\_nan}{\code{GSL\_IEEE\_TYPE\_NAN}}
\lineii{type\_inf}{\code{GSL\_IEEE\_TYPE\_INF}}
\lineii{type\_normal}{\code{GSL\_IEEE\_TYPE\_NORMAL}}
\lineii{type\_denormal}{\code{GSL\_IEEE\_TYPE\_DENORMAL}}
\lineii{type\_zero}{\code{GSL\_IEEE\_TYPE\_ZERO}}
\end{tableii}
\end{funcdesc}

\begin{funcdesc}{isnan}{float value}
determines if the argument is not a valid number
\end{funcdesc}

\begin{funcdesc}{nan}{}
generates a ``not-a-number'' value. This is implemented as function, because of the potential exception generation by your floating-point unit.
\end{funcdesc}

\begin{funcdesc}{isinf}{float value}
returns -1 if the argument represents a negative infinite value and +1 if positive, 0 otherwise
\end{funcdesc}

\begin{funcdesc}{posinf}{}
gives you the representation of ``positive infinity''
\end{funcdesc}

\begin{funcdesc}{neginf}{}
the same as posinf, but negative
\end{funcdesc}

\begin{funcdesc}{finite}{float value}
results in 1 if the value is finite, 0 if it is not a number or infinite
\end{funcdesc}


%\chapter[\protect\module{pygsl.init} --- Library initialisation]
%{\protect\module{pygsl.init} \\ Library initialisation}
%\label{cha:library-initialisation}
%\declaremodule{extension}{pygsl.init}
\moduleauthor{Pierre Schnizer}{schnizer@users.sourceforge.net}
\moduleauthor{Achim G\"adke}{achimgaedke@users.sourceforge.net}

This module is called the first time when loading \module{pygsl}.
All following procedures are called once and before everything other.

\section{Exception handling}
\index{exception handling!initialisation} GSL provides a selectable error
handler, that is called for occuring errors (like domain errors, division by
zero, etc. ).  This is switched off. Instead each wrapper function will check
the error return value and in case of error an python exception is created. 

Here is a python level example:
\begin{verbatim}
import pygsl.histogram
import pygsl.errors
hist=pygsl.histogram.histogram2d(100,100)
try:
   hist[-1,-1]=0
except pygsl.errors.gsl_Error,err:
   print err
\end{verbatim}
Will result
\begin{verbatim}
input domain error: index i lies outside valid range of 0 .. nx - 1
\end{verbatim}


An exception are ufuncs in the testings.sf module (see section\ref{sec:ufuncs}).

%\module{pygsl.init} installs a handler by calling
%\cfunction{gsl_set_error_handler} to set an appropiate exception from
%\module{pygsl.errors}.  A \module{pygsl} interface function should return
%\code{NULL} in case of an error, so the exception is raised.  If this handler
%is called more than once before returning to python, only the first set
%exception is raised.
%
%
% 
% \section{IEEE-mode}
% \index{ieee-mode!initialisation}
% The IEEE mode is set from the environment variable
%  \envvar{GSL_IEEE_MODE} via \cfunction{gsl_ieee_env_setup()}.
% After the initialisation use \module{pygsl.ieee} for manipulation.
% 
% \section{random number generators}
% \index{random number generator!initialisation}
% Also the random number generator can be initialised from the environment variables
%  \envvar{GSL_RNG_TYPE}
% and \envvar{GSL_RNG_SEED} using the gsl function \cfunction{gsl_rng_env_setup()}.
% Each random number generators are initialised with \envvar{GSL_RNG_SEED}.
% 
% The default generator can be created by:\nopagebreak
% \begin{verbatim}
% import pygsl.rng
% my_rng=pygsl.rng.rng()
% print my_rng.name()
% \end{verbatim}




\chapter{GNU Free Documentation License}
\label{cha:free-documentation-license}

Version 1.1, March 2000\\

 Copyright \copyright\ 2000  Free Software Foundation, Inc.\\
     59 Temple Place, Suite 330, Boston, MA  02111-1307  USA\\
 Everyone is permitted to copy and distribute verbatim copies
 of this license document, but changing it is not allowed.

\section*{Preamble}

The purpose of this License is to make a manual, textbook, or other
written document ``free'' in the sense of freedom: to assure everyone
the effective freedom to copy and redistribute it, with or without
modifying it, either commercially or noncommercially.  Secondarily,
this License preserves for the author and publisher a way to get
credit for their work, while not being considered responsible for
modifications made by others.

This License is a kind of ``copyleft'', which means that derivative
works of the document must themselves be free in the same sense.  It
complements the GNU General Public License, which is a copyleft
license designed for free software.

We have designed this License in order to use it for manuals for free
software, because free software needs free documentation: a free
program should come with manuals providing the same freedoms that the
software does.  But this License is not limited to software manuals;
it can be used for any textual work, regardless of subject matter or
whether it is published as a printed book.  We recommend this License
principally for works whose purpose is instruction or reference.

\section{Applicability and Definitions}

This License applies to any manual or other work that contains a
notice placed by the copyright holder saying it can be distributed
under the terms of this License.  The ``Document'', below, refers to any
such manual or work.  Any member of the public is a licensee, and is
addressed as ``you''.

A ``Modified Version'' of the Document means any work containing the
Document or a portion of it, either copied verbatim, or with
modifications and/or translated into another language.

A ``Secondary Section'' is a named appendix or a front-matter section of
the Document that deals exclusively with the relationship of the
publishers or authors of the Document to the Document's overall subject
(or to related matters) and contains nothing that could fall directly
within that overall subject.  (For example, if the Document is in part a
textbook of mathematics, a Secondary Section may not explain any
mathematics.)  The relationship could be a matter of historical
connection with the subject or with related matters, or of legal,
commercial, philosophical, ethical or political position regarding
them.

The ``Invariant Sections'' are certain Secondary Sections whose titles
are designated, as being those of Invariant Sections, in the notice
that says that the Document is released under this License.

The ``Cover Texts'' are certain short passages of text that are listed,
as Front-Cover Texts or Back-Cover Texts, in the notice that says that
the Document is released under this License.

A ``Transparent'' copy of the Document means a machine-readable copy,
represented in a format whose specification is available to the
general public, whose contents can be viewed and edited directly and
straightforwardly with generic text editors or (for images composed of
pixels) generic paint programs or (for drawings) some widely available
drawing editor, and that is suitable for input to text formatters or
for automatic translation to a variety of formats suitable for input
to text formatters.  A copy made in an otherwise Transparent file
format whose markup has been designed to thwart or discourage
subsequent modification by readers is not Transparent.  A copy that is
not ``Transparent'' is called ``Opaque''.

Examples of suitable formats for Transparent copies include plain
ASCII without markup, Texinfo input format, \LaTeX~input format, SGML
or XML using a publicly available DTD, and standard-conforming simple
HTML designed for human modification.  Opaque formats include
PostScript, PDF, proprietary formats that can be read and edited only
by proprietary word processors, SGML or XML for which the DTD and/or
processing tools are not generally available, and the
machine-generated HTML produced by some word processors for output
purposes only.

The ``Title Page'' means, for a printed book, the title page itself,
plus such following pages as are needed to hold, legibly, the material
this License requires to appear in the title page.  For works in
formats which do not have any title page as such, ``Title Page'' means
the text near the most prominent appearance of the work's title,
preceding the beginning of the body of the text.


\section{Verbatim Copying}

You may copy and distribute the Document in any medium, either
commercially or noncommercially, provided that this License, the
copyright notices, and the license notice saying this License applies
to the Document are reproduced in all copies, and that you add no other
conditions whatsoever to those of this License.  You may not use
technical measures to obstruct or control the reading or further
copying of the copies you make or distribute.  However, you may accept
compensation in exchange for copies.  If you distribute a large enough
number of copies you must also follow the conditions in section 3.

You may also lend copies, under the same conditions stated above, and
you may publicly display copies.


\section{Copying in Quantity}

If you publish printed copies of the Document numbering more than 100,
and the Document's license notice requires Cover Texts, you must enclose
the copies in covers that carry, clearly and legibly, all these Cover
Texts: Front-Cover Texts on the front cover, and Back-Cover Texts on
the back cover.  Both covers must also clearly and legibly identify
you as the publisher of these copies.  The front cover must present
the full title with all words of the title equally prominent and
visible.  You may add other material on the covers in addition.
Copying with changes limited to the covers, as long as they preserve
the title of the Document and satisfy these conditions, can be treated
as verbatim copying in other respects.

If the required texts for either cover are too voluminous to fit
legibly, you should put the first ones listed (as many as fit
reasonably) on the actual cover, and continue the rest onto adjacent
pages.

If you publish or distribute Opaque copies of the Document numbering
more than 100, you must either include a machine-readable Transparent
copy along with each Opaque copy, or state in or with each Opaque copy
a publicly-accessible computer-network location containing a complete
Transparent copy of the Document, free of added material, which the
general network-using public has access to download anonymously at no
charge using public-standard network protocols.  If you use the latter
option, you must take reasonably prudent steps, when you begin
distribution of Opaque copies in quantity, to ensure that this
Transparent copy will remain thus accessible at the stated location
until at least one year after the last time you distribute an Opaque
copy (directly or through your agents or retailers) of that edition to
the public.

It is requested, but not required, that you contact the authors of the
Document well before redistributing any large number of copies, to give
them a chance to provide you with an updated version of the Document.


\section{Modifications}

You may copy and distribute a Modified Version of the Document under
the conditions of sections 2 and 3 above, provided that you release
the Modified Version under precisely this License, with the Modified
Version filling the role of the Document, thus licensing distribution
and modification of the Modified Version to whoever possesses a copy
of it.  In addition, you must do these things in the Modified Version:

\begin{itemize}

\item Use in the Title Page (and on the covers, if any) a title distinct
   from that of the Document, and from those of previous versions
   (which should, if there were any, be listed in the History section
   of the Document).  You may use the same title as a previous version
   if the original publisher of that version gives permission.
\item List on the Title Page, as authors, one or more persons or entities
   responsible for authorship of the modifications in the Modified
   Version, together with at least five of the principal authors of the
   Document (all of its principal authors, if it has less than five).
\item State on the Title page the name of the publisher of the
   Modified Version, as the publisher.
\item Preserve all the copyright notices of the Document.
\item Add an appropriate copyright notice for your modifications
   adjacent to the other copyright notices.
\item Include, immediately after the copyright notices, a license notice
   giving the public permission to use the Modified Version under the
   terms of this License, in the form shown in the Addendum below.
\item Preserve in that license notice the full lists of Invariant Sections
   and required Cover Texts given in the Document's license notice.
\item Include an unaltered copy of this License.
\item Preserve the section entitled ``History'', and its title, and add to
   it an item stating at least the title, year, new authors, and
   publisher of the Modified Version as given on the Title Page.  If
   there is no section entitled ``History'' in the Document, create one
   stating the title, year, authors, and publisher of the Document as
   given on its Title Page, then add an item describing the Modified
   Version as stated in the previous sentence.
\item Preserve the network location, if any, given in the Document for
   public access to a Transparent copy of the Document, and likewise
   the network locations given in the Document for previous versions
   it was based on.  These may be placed in the ``History'' section.
   You may omit a network location for a work that was published at
   least four years before the Document itself, or if the original
   publisher of the version it refers to gives permission.
\item In any section entitled ``Acknowledgements'' or ``Dedications'',
   preserve the section's title, and preserve in the section all the
   substance and tone of each of the contributor acknowledgements
   and/or dedications given therein.
\item Preserve all the Invariant Sections of the Document,
   unaltered in their text and in their titles.  Section numbers
   or the equivalent are not considered part of the section titles.
\item Delete any section entitled ``Endorsements''.  Such a section
   may not be included in the Modified Version.
\item Do not retitle any existing section as ``Endorsements''
   or to conflict in title with any Invariant Section.

\end{itemize}

If the Modified Version includes new front-matter sections or
appendices that qualify as Secondary Sections and contain no material
copied from the Document, you may at your option designate some or all
of these sections as invariant.  To do this, add their titles to the
list of Invariant Sections in the Modified Version's license notice.
These titles must be distinct from any other section titles.

You may add a section entitled ``Endorsements'', provided it contains
nothing but endorsements of your Modified Version by various
parties -- for example, statements of peer review or that the text has
been approved by an organization as the authoritative definition of a
standard.

You may add a passage of up to five words as a Front-Cover Text, and a
passage of up to 25 words as a Back-Cover Text, to the end of the list
of Cover Texts in the Modified Version.  Only one passage of
Front-Cover Text and one of Back-Cover Text may be added by (or
through arrangements made by) any one entity.  If the Document already
includes a cover text for the same cover, previously added by you or
by arrangement made by the same entity you are acting on behalf of,
you may not add another; but you may replace the old one, on explicit
permission from the previous publisher that added the old one.

The author(s) and publisher(s) of the Document do not by this License
give permission to use their names for publicity for or to assert or
imply endorsement of any Modified Version.


\section{Combining Documents}

You may combine the Document with other documents released under this
License, under the terms defined in section 4 above for modified
versions, provided that you include in the combination all of the
Invariant Sections of all of the original documents, unmodified, and
list them all as Invariant Sections of your combined work in its
license notice.

The combined work need only contain one copy of this License, and
multiple identical Invariant Sections may be replaced with a single
copy.  If there are multiple Invariant Sections with the same name but
different contents, make the title of each such section unique by
adding at the end of it, in parentheses, the name of the original
author or publisher of that section if known, or else a unique number.
Make the same adjustment to the section titles in the list of
Invariant Sections in the license notice of the combined work.

In the combination, you must combine any sections entitled ``History''
in the various original documents, forming one section entitled
``History''; likewise combine any sections entitled ``Acknowledgements'',
and any sections entitled ``Dedications''.  You must delete all sections
entitled ``Endorsements.''


\section{Collections of Documents}

You may make a collection consisting of the Document and other documents
released under this License, and replace the individual copies of this
License in the various documents with a single copy that is included in
the collection, provided that you follow the rules of this License for
verbatim copying of each of the documents in all other respects.

You may extract a single document from such a collection, and distribute
it individually under this License, provided you insert a copy of this
License into the extracted document, and follow this License in all
other respects regarding verbatim copying of that document.



\section{Aggregation With Independent Works}

A compilation of the Document or its derivatives with other separate
and independent documents or works, in or on a volume of a storage or
distribution medium, does not as a whole count as a Modified Version
of the Document, provided no compilation copyright is claimed for the
compilation.  Such a compilation is called an ``aggregate'', and this
License does not apply to the other self-contained works thus compiled
with the Document, on account of their being thus compiled, if they
are not themselves derivative works of the Document.

If the Cover Text requirement of section 3 is applicable to these
copies of the Document, then if the Document is less than one quarter
of the entire aggregate, the Document's Cover Texts may be placed on
covers that surround only the Document within the aggregate.
Otherwise they must appear on covers around the whole aggregate.


\section{Translation}

Translation is considered a kind of modification, so you may
distribute translations of the Document under the terms of section 4.
Replacing Invariant Sections with translations requires special
permission from their copyright holders, but you may include
translations of some or all Invariant Sections in addition to the
original versions of these Invariant Sections.  You may include a
translation of this License provided that you also include the
original English version of this License.  In case of a disagreement
between the translation and the original English version of this
License, the original English version will prevail.


\section{Termination}

You may not copy, modify, sublicense, or distribute the Document except
as expressly provided for under this License.  Any other attempt to
copy, modify, sublicense or distribute the Document is void, and will
automatically terminate your rights under this License.  However,
parties who have received copies, or rights, from you under this
License will not have their licenses terminated so long as such
parties remain in full compliance.


\section{Future Revisions of This License}

The Free Software Foundation may publish new, revised versions
of the GNU Free Documentation License from time to time.  Such new
versions will be similar in spirit to the present version, but may
differ in detail to address new problems or concerns. See
http://www.gnu.org/copyleft/.

Each version of the License is given a distinguishing version number.
If the Document specifies that a particular numbered version of this
License "or any later version" applies to it, you have the option of
following the terms and conditions either of that specified version or
of any later version that has been published (not as a draft) by the
Free Software Foundation.  If the Document does not specify a version
number of this License, you may choose any version ever published (not
as a draft) by the Free Software Foundation.


% Complete documentation on the extended LaTeX markup used for Python
% documentation is available in ``Documenting Python'', which is part
% of the standard documentation for Python.  It may be found online
% at:
%
%     http://www.python.org/doc/current/doc/doc.html

\documentclass[hyperref]{manual}

% latex2html doesn't know [T1]{fontenc}, so we cannot use that:(
\usepackage{amsmath}
\usepackage[latin1]{inputenc}
\usepackage{textcomp}
\usepackage{hyperref}

% this version does not reset module names at section level
%begin{latexonly}
\makeatletter
\let\py@OldOldChapter=\chapter
\renewcommand{\chapter}{\py@reset%
                        \py@OldOldChapter}
\renewcommand{\section}{\@startsection{section}{1}{\z@}%
   {-3.5ex \@plus -1ex \@minus -.2ex}%
   {2.3ex \@plus.2ex}%
   {\reset@font\Large\py@HeaderFamily}}
\makeatother
%end{latexonly}


% some convenience declarations
\newcommand{\gsl}{GSL}
\newcommand{\GSL}{GNU Scientific Library}
\newcommand{\numpy}{NumPy}
\newcommand{\NUMPY}{Numerical Python}
\newcommand{\pygsl}{PyGSL}
\newcommand{\PYGSL}{PyGSL: Python wrapper of the GNU Scientific Library}

\makeatletter
\newenvironment{pytypedesc}[2]{
  % Using \renewcommand doesn't work for this, for unknown reasons:
  \global\def\py@thisclass{#1}
  \begin{fulllineitems}
    \py@sigline{\strong{pytype }\bfcode{#1}}{#2}%
    \index{#1@{\py@idxcode{#1}} (pytype in \py@thismodule)}
}{\end{fulllineitems}}
\makeatother


\title{PyGSL Reference Manual}

\ifhtml
\author{
  \ulink{Achim G\"adke}{mailto:achimgaedke@users.sourceforge.net}\\
  Technische Universit\"at Darmstadt, Darmstadt, Germany
}
\author{
  \ulink{Pierre Schnizer}{mailto:schnizer@users.sourceforge.net}\\
  Gesellschaft f\"ur Schwerionenforschung, Darmstadt, Germany
}
%\author{
%  \ulink{Jochen K\"upper}{mailto:jochen@jochen-kuepper.de}\\
%  Fritz-Haber-Institut der MPG, Berlin, Germany
%}
%\author{
%  \ulink{S\'ebastien Maret}{mailto:schnizer@users.sourceforge.net}\\
%  Department of Astronomy, University of Michigan, Ann Arbor, USA
%}
\else
%begin{latexonly}
%% pdfelatex (TeXLive 7) doesn't handle \footnotemark in here...
\author{Achim G\"adke \\ 
          Jochen K\"upper \\ 
         %S\'ebastien Maret \\
        Pierre Schnizer}
% Please at least include a long-lived email address!
\authoraddress{
   Technische Universit\"at Darmstadt, Darmstadt, Germany \\
   \email{achimgaedke@users.sourceforge.net} \\
   Gesellschaft f\"ur Schwerionenforschung, Darmstadt, Germany \\
   \email{schnizer@users.sourceforge.net} \\
%   Fritz-Haber-Institut der MPG, Berlin, Germany \\
%   \email{jochen@jochen-kuepper.de} \\
%   Department of Astronomy, University of Michigan, Ann Arbor, USA \\
%   \email{bmaret@users.sourceforge.et} \\
}
%end{latexonly}
\fi

\date{October, 2008}            % update before release!
                                % Use an explicit date so that reformatting
                                % doesn't cause a new date to be used.  Setting
                                % the date to \today can be used during draft
                                % stages to make it easier to handle versions.
\release{0.9}                   % release version; this is used to define the
\setshortversion{0.9}           % \version macro
\makeindex                      % tell \index to actually write the .idx file


\begin{document}

\maketitle

% This makes the contents more accessible from the front page of the HTML.
\ifhtml
\chapter*{Front Matter}
\label{front}
\fi

\input{copyright}

\begin{abstract}
   \noindent
   pygsl grants python users access to the GNU scientific library.  The latest
   version can be found at the project homepage, \url{http://pygsl.sf.net}.

   \textbf{Implemented features:} \\
   \begin{tabular}{ll}
     \module{pygsl.blas}                & basic linear algebra system\\
     \module{pygsl.chebyshev}           & chebyshev approximations\\
     \module{pygsl.combination}         & combinations  \\
     \module{pygsl.const}               & $>200$ often used mathematical and
                                          scientific constants. \\
     \module{pygsl.diff}                & (Deprecated. Use pygsl.deriv instead). \\
     \module{pygsl.deriv}               & Numerical differentiation. \\
     \module{pygsl.eigen}               &\\
     \module{pygsl.fit}                 &\\
     \module{pygsl.histogram}          & 1d and 2d histograms and operations
                                          on histograms. \\
     \module{pygsl.ieee}                & Access to the ieee-arithmetics layer
                                          of gsl. \\ 
     \module{pygsl.integrate}           &\\
     \module{pygsl.interpolation}       &\\ 
     \module{pygsl.linalg}              &\\
     \module{pygsl.math}                &\\
     \module{pygsl.monte}               &\\
     \module{pygsl.minimize}            &\\
     \module{pygsl.multifit}            &\\
     \module{pygsl.multifit_nlin}       &\\    
     \module{pygsl.multimin}            &\\
     \module{pygsl.multiroots}          &\\ 
     \module{pygsl.odeiv}               &\\
     \module{pygsl.permutation}         &\\  
     \module{pygsl.poly}                &\\
     \module{pygsl.qrng}                &\\     
     \module{pygsl.rng}                 & random number generators and probability densities. \\
     \module{pygsl.roots}               &\\
     \module{pygsl.siman}               &Simulated anealing\\
     \module{pygsl.sum}                 & \\
     \module{pygsl.sf}                  & $>200$ special functions. \\
     \module{pygsl.statistics}          & Statistical functions. \\
   \end{tabular}
\end{abstract}


\tableofcontents


\chapter{System Requirements, Installation}
\label{cha:system-req-installation}
\input{install}
\input{installadvanced.tex}
\input{interfacedesign.tex}
\paragraph*{Acknowledgment}
\label{sec:acknowledgment}
Parts of this this manual are based on the \GSL{} reference manual.
The authors want to thank all for contribution of code,
support material for generating distribution packages, bug reports
and example scripts.


\input{errors}

\chapter[\protect\module{pygsl.const} --- Mathematical and scientific
constants]{\protect\module{pygsl.const} \\ Mathematical and scientific
constants} 
\label{cha:const-module}
\input{const}
\input{chebyshev}
\input{differentiation}

\chapter[\protect\module{pygsl.histogram} --- Histogram Types]
{\protect\module{pygsl.histogram} \\ Histogram Types}
\label{cha:histogram-module}
\input{histogram}

\chapter[\protect\module{pygsl.rng} --- Random Number Generators]
{\protect\module{pygsl.rng} \\ Random Number Generators}
\label{cha:rng-module}
\input{rng}

%\chapter[\protect\module{pygsl.sf} --- Special Functions]
%{\protect\module{pygsl.sf} \\ Special Functions}
%\label{cha:sf-module}
%\input{sf}

\input{sum}
\input{statistics}

\input{testing}

\appendix

\chapter[\protect\module{pygsl.ieee} --- Floating Point Unit Support]
{\protect\module{pygsl.ieee} \\ Floating Point Unit Support}
\label{cha:ieee-module}
\input{ieee}

%\chapter[\protect\module{pygsl.init} --- Library initialisation]
%{\protect\module{pygsl.init} \\ Library initialisation}
%\label{cha:library-initialisation}
%\input{init}



\input{freedoc}

\input{ref.ind}                    % Index

\end{document}


%% Local Variables:
%% mode: LaTeX
%% mode: auto-fill
%% fill-column: 79
%% indent-tabs-mode: nil
%% ispell-dictionary: "american"
%% reftex-fref-is-default: nil
%% TeX-auto-save: t
%% TeX-command-default: "pdfeLaTeX"
%% TeX-parse-self: t
%% End:
                    % Index

\end{document}


%% Local Variables:
%% mode: LaTeX
%% mode: auto-fill
%% fill-column: 79
%% indent-tabs-mode: nil
%% ispell-dictionary: "american"
%% reftex-fref-is-default: nil
%% TeX-auto-save: t
%% TeX-command-default: "pdfeLaTeX"
%% TeX-parse-self: t
%% End:
                    % Index

\end{document}


%% Local Variables:
%% mode: LaTeX
%% mode: auto-fill
%% fill-column: 79
%% indent-tabs-mode: nil
%% ispell-dictionary: "american"
%% reftex-fref-is-default: nil
%% TeX-auto-save: t
%% TeX-command-default: "pdfeLaTeX"
%% TeX-parse-self: t
%% End:
                    % Index

\end{document}


%% Local Variables:
%% mode: LaTeX
%% mode: auto-fill
%% fill-column: 79
%% indent-tabs-mode: nil
%% ispell-dictionary: "american"
%% reftex-fref-is-default: nil
%% TeX-auto-save: t
%% TeX-command-default: "pdfeLaTeX"
%% TeX-parse-self: t
%% End:
