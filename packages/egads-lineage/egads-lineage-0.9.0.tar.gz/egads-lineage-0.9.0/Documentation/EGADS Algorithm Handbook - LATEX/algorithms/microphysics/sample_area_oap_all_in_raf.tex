%% $Date: 2012-07-06 17:42:54#$
%% $Revision: 168 $
\index{sample\_area\_oap\_all\_in\_raf}
\algdesc{Sample area for imaging probes (All in)}
{ %%%%%% Algorithm name %%%%%%
sample\_area\_oap\_all\_in\_raf
}
{ %%%%%% Algorithm summary %%%%%%
Calculation of 'all in' sample area size for OAP probes such as the 2DC, 2DP, CIP, etc. This sample area varies by
number of shadowed diodes. This routine calculates a sample area per bin.
}
{ %%%%%% Category %%%%%%
Microphysics
}
{ %%%%%% Inputs %%%%%%
$\lambda$ & Coeff. & Laser wavelength [nm] \\
$D_{arms}$ & Coeff. & Distance between probe arm tips [mm] \\
dD & Coeff. & Diode diameter [$\mu$m] \\
M & Coeff. & Probe magnification factor \\
N & Coeff. & Number of diodes in array
}
{ %%%%%% Outputs %%%%%%
SA & Vector & Sample area [m$^2$]
}
{ %%%%%% Formula %%%%%%
\begin{eqnarray}
DOF_i = \frac{6 R_i^2}{\lambda} \\ \nonumber
R_i = i \frac{dD}{2} \\ \nonumber
X = {1...N-1} \nonumber
\end{eqnarray}
where $DOF$ must be less than $D_{arms}$. The parameter $i$ ranges from 1 to $N-1$, since particles touching either edge are rejected as they are not considered 'all-in'. 


\begin{displaymath}
ESW_i = \frac{dD(N-X_i-1)}{M}
\end{displaymath}
A value for $ESW_i$ (effective sample width) is calculated for each $X$.

\begin{displaymath}
 SA_i = (DOF_i)(ESW_i) 
\end{displaymath}
}
{ %%%%%% Author %%%%%%
NCAR-RAF
}
{ %%%%%% References %%%%%% 
NCAR-RAF Bulletin No. 24. \cite{NCAR24}
}

