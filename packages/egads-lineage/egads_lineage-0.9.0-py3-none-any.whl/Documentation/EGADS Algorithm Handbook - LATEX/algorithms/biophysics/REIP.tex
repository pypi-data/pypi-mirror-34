%% $Date: 2012-07-06 17:42:54#$
%% $Revision: 168 $
\index{REIP}
\algdescgeneral{REIP}
%
{ %%%%%% Algorithm name %%%%%%
biophys\_indices (REIP is one index calculated within the overall program)
}
%
{ %%%%%% Algorithm summary %%%%%%
Calculation of red edge inflection point (REIP), method 1
}
%
{ %%%%%% Category %%%%%%
Biophysics - red edge parametrisation
}
%
{ %%%%%% Inputs %%%%%%
Narrow band multi- or hyperspectral imagery (ENVI standard image data) including channels close to the wavelengths of 671nm, 701nm, 740nm and 780nm.\bigskip
}
%
{ %%%%%% Outputs %%%%%%
Single band with REIP values
}
%
{ %%%%%% Formula %%%%%%
\begin{displaymath}
REIP =  700+40*\frac{0.5*(R_{671}+R_{780})-R_{701}}{R_{740}-R_{701}}
\end{displaymath}
}
%
{ %%%%%% Author %%%%%%
DLR-DFD
}
%
{ %%%%%% References %%%%%% 
Guyot, G., Baret, F. and Major, D. J. (1988). High spectral resolution: determination of spectral shifts between the red and the near infrared. In: International Archives of Photogrammetry and Remote Sensing 11, p. 750–760
}
